\documentclass[12pt,reqno]{article}
\usepackage[margin=3cm]{geometry}
\usepackage[utf8]{inputenc}


\usepackage{xfp}
\usepackage{natbib}
\usepackage{graphicx}
\usepackage{amsthm}
\usepackage{amsmath}
\usepackage{amssymb}
\usepackage{cases}
\usepackage{microtype}
\usepackage{hyperref}
\usepackage{mathrsfs}

\newtheorem{theorem}{Theorem}[section]
\newtheorem{proposition}[theorem]{Proposition}
\newtheorem{corollary}[theorem]{Corollary}
\newtheorem{lemma}[theorem]{Lemma}
\newtheorem{conjecture}[theorem]{Conjecture}
\newtheorem{example}[theorem]{Example}
\theoremstyle{definition}
\newtheorem{definition}[theorem]{Definition}
\theoremstyle{remark}
\newtheorem{remark}[theorem]{Remark}
\theoremstyle{exercise}
\newtheorem{exercise}[section]{Exercise}
\theoremstyle{subExercise}
\newtheorem{subExercise}{Part}[section]
\newtheorem{subSubExercise}{Sub Part}[subExercise]
\numberwithin{equation}{section}
\newcommand{\acl}[2]{\operatorname{acl}^{#2}\left(#1\right)}
\newcommand{\rank}[2]{\operatorname{rank}_{#2}\left(#1\right)}
\newcommand{\dcl}[2]{\operatorname{dcl}^{#2}\left(#1\right)}
\newcommand{\Aut}[1]{\operatorname{Aut}\left(#1\right)}
\newcommand{\Av}[1]{\operatorname{Av}\left(#1\right)}
\newcommand{\comment}[2]{#2}
\newcommand{\cof}[1]{\operatorname{cof}\left(#1\right)}
\renewcommand{\cal}[1]{\mathcal{#1}}
\newcommand{\D}[2][{}]{\text{Diag}_{#1}\left(#2\right)}
\renewcommand{\phi}{\varphi}
\newcommand{\code}[1]{\left\lceil#1\right\rceil}
\setcounter{MaxMatrixCols}{20}
\sloppy

\newcommand{\ex}[2][\fpeval{\value{section}+1}]{\setcounter{section}{\fpeval{#1 - 1}}
	\begin{exercise}
		#2
\end{exercise}}
\newcommand{\sub}[2][\fpeval{\value{subExercise}+1}]{\setcounter{subExercise}{\fpeval{#1 - 1}}
	\begin{subExercise}
		#2
\end{subExercise}}
\newcommand{\subb}[2][\fpeval{\value{subSubExercise}+1}]{\setcounter{subSubExercise}{\fpeval{#1 - 1}}
	\begin{subSubExercise}
		#2
\end{subSubExercise}}
\newcommand{\cl}{\mbox{cl}}
\newcommand{\CB}[1]{\operatorname{CB}\left(#1\right)}
\newcommand{\MR}[1]{\operatorname{MR}\left(#1\right)}
\newcommand{\MD}[1]{\operatorname{MD}\left(#1\right)}
\newcommand{\monster}{{\mathfrak C}}
\newcommand{\tp}[1]{\operatorname{tp}\left(#1\right)}
\newcommand{\hull}[3]{\operatorname{Hull}^{#2}_{#3}\left(#1\right)}
\newcommand{\stp}[1]{\operatorname{stp}\left(#1\right)}
\newcommand{\dom}[1]{\operatorname{dom}\left(#1\right)}
\newcommand{\range}[1]{\operatorname{range}\left(#1\right)}
\newcommand{\cnst}[2]{\operatorname{const}_{#1}\left(#2\right)}
\renewcommand{\c}{{\mathfrak c}}
\newcommand{\club}[1]{\operatorname{club}\left(#1\right)}
\newcommand{\Lev}[1]{\operatorname{Lev}\left(#1\right)}
\newcommand{\height}[1]{\operatorname{ht}\left(#1\right)}
\newcommand{\emptyseq}{\Lambda}

\newcommand{\N}{\mathbb N}
\newcommand{\Q}{\mathbb Q}
\newcommand{\R}{\mathbb R}
\newcommand{\C}{\mathbb C}
\newcommand{\F}{\mathbb F}




\usepackage{xparse}

\ExplSyntaxOn
\tl_new:N \l_septatrix_env_tl
\NewDocumentCommand \getenv { o m }
{
	\sys_get_shell:nnN { kpsewhich ~ --var-value ~ #2 }
	{ \int_set:Nn \tex_endlinechar:D { -1 } }
	\l_septatrix_env_tl
	\IfNoValueTF {#1}
	{ \tl_use:N \l_septatrix_env_tl }
	{ \tl_set_eq:NN #1 \l_septatrix_env_tl }
}
\tl_const:Nn \c_getenv_par_tl { \par }

\NewDocumentCommand{\ifenvsetTF}{mmm}
{
	\sys_get_shell:nnN { kpsewhich ~ --var-value ~ #1 } { } \l_tmpa_tl
	\tl_if_eq:NNTF \l_tmpa_tl \c_getenv_par_tl { #3 } { #2 }
}
\ExplSyntaxOff

\usepackage{datetime}
% \newdate{date}{06}{09}{2012}
% \date{\displaydate{date}}
\date{\today}
\newcommand{\envOrDefault}[2]{\ifenvsetTF{#1}{\getenv{#1}}{#2}}

\author{\envOrDefault{au4thor}{Holo}}



\usepackage{skak}
\usepackage{relsize}
\usepackage{graphicx}
\usepackage{mathtools}

\usepackage{textcomp}
\usepackage{bbding}

\usepackage{soul}

\newcommand{\flower}{\text{\scalebox{0.75}{\raisebox{-0.7ex}{
				\rotatebox{90}{\textleaf}\hspace{-0.3em}
				\scalebox{0.7}{\textleaf}\hspace{-1.35em}
				\raisebox{1ex}{\scalebox{0.8}{\FiveFlowerOpen}}
}}}}
\title{Exercise 1}
\begin{document}
\maketitle
\begin{cExercise}[][][author][1]
	\begin{cPart}
		There is no identity element.
	\end{cPart}
	\begin{cPart}
		It is a group, whose identity is $1$ and inverse of $x$ is $\frac1x$.
	\end{cPart}
	\begin{cPart}
		There is no inverse element to any $x\ne 1,-1$.
	\end{cPart}
	\begin{cPart}
		This is not even a structure (addition is not total on the domain).
	\end{cPart}
	\begin{cPart}
		The identity is $1$.
	\end{cPart}
	\begin{cPart}
		The identity is $1$ and the inverse of $z$ is $\frac1z$.
	\end{cPart}
	\begin{cPart}
		This is again not a structure, as multiplication is not total on the domain ($\sqrt2\sqrt2=2\ne \frac ab+\frac cd\sqrt2$).
	\end{cPart}
	\begin{cPart}
		The identity is $I_2$ and the inverse of 
		$
		\begin{pmatrix}
			x & y\\
			-y & x
		\end{pmatrix}
		$ is $\frac1{x^2+y^2}
		\begin{pmatrix}
			x & -y\\
			y & x
		\end{pmatrix}
		$
	\end{cPart}
	\begin{cPart}
		The identity is the identity function, and the inverse of $f$ is the unique function $g$ such that $g(f(x))=x$
	\end{cPart}
\end{cExercise}
\begin{cExercise}
	\begin{cPart}
		$
		\begin{pmatrix}
			1 & 1\\
			0 & 0
		\end{pmatrix}
		$ is the left identity, indeed $
		\begin{pmatrix}
			1 & 1\\
			0 & 0
		\end{pmatrix}\cdot
		\begin{pmatrix}
			x & y\\
			0 & 0
		\end{pmatrix}=
		\begin{pmatrix}
			x\cdot1+0\cdot1 & 0\cdot1+y\cdot1\\
			0 & 0
		\end{pmatrix}=
		\begin{pmatrix}
			x & y\\
			0 & 0
		\end{pmatrix}
		$
	\end{cPart}
	\begin{cPart}
		The right inverse of $\begin{pmatrix}
			x & y\\
			0 & 0
		\end{pmatrix}$ is $\begin{pmatrix}
			\frac1x & \frac1x\\
			0 & 0
		\end{pmatrix}$
	\end{cPart}
	\begin{cPart}
		We saw at class that given a group $H$, the left identity is always the identity, so if $G$ were to be a group, then $\begin{pmatrix}
			x & y\\
			0 & 0
			\end{pmatrix}\cdot \begin{pmatrix}
			1 & 1\\
			0 & 0
			\end{pmatrix}=\begin{pmatrix}
			x & y\\
			0 & 0
			\end{pmatrix}$, but this is absurd.
	\end{cPart}
\end{cExercise}
\begin{cExercise}
	\begin{cPart}
		If $G$ is Abelian then $a^2b^2=aabb=abab=(ab)^2$.\\
		If $G$ satisfy this equality, then given $a,b$ we have $aabb=abab\implies a^{-1}aabb=a^{-1}abab\implies abbb^{-1}=babb^{-1}\implies ab=ba$
		
		To see an example, let $G=D_4$ and chose $a$ to be $90^\circ$ rotation and $b$ to be rotation through the $y$-axis.
		
		Then $a^2b^2=a^2=$rotation by $180^\circ$ but $(ab)^2=e$ 
	\end{cPart}
	\begin{cPart}
		If $G$ is such group then $(ab)^2=e=ee=a^2b^2$, and by exercise 3.1 it is an Abelian group.
	\end{cPart}
	\begin{cPart}
		The only non trivial vector space axioms are the distributivity axioms. (I will use $+$ for the group operator as accustomed in vector spaces) \\
		Let $g,h\in G$ then $0(g+h)=e=e+e=0g+0h$ and $1(g+h)=e+(g+h)=g+h=(e+g)+(e+h)=1g+1h$.\\
		Let $g\in G$, then $g=1g=(1+0)g=(1+0)g$ and $1g+0g=g+e=e+g=g$. Symmetric argument works for $0+1$. The $0+0$ case is trivial. Lastly $(1+1)g=0g=e$ and $e=g+g=1g+1g$
	\end{cPart}
\end{cExercise}
\begin{cExercise}
	\begin{cPart}
		Assume that $x^k=x^m$ for $0\le k<m<n, \ell=m-k$, then in particular $e=x^0=x^{k-k}=x^{m-k}=x^{\ell}$ because multiplying by $x^{-1}$ is a an injective function.
		But that means that $|x|=\ell<n$, contradiction. And clearly, if $y\in \langle x\rangle$, then $y=x^p$ for some $p\in \mathbb Z$, because $x^{-1}=x^{n-1}$, we can assume $p\in\mathbb N$, but we have that $x^p=x^{p\mod n}$, so $y\in\{e,x,\ldots,x^{n-1}\}$, hence $|\langle x\rangle|=|x|$.
	\end{cPart}
	\begin{cPart}
		Given $n<m$, assume $n\ge 0$ then if $x^n=x^m$ then, just like before, we can multiple by $x^{-1}$ $n$-times to show that $e=x^{m-n}\implies x=x^{m-n+1}$, contradict the assumption.
		
		If $n<0$ then we do the same by instead of multiplying by $x^{-1}$ we multiply by $x$.
	\end{cPart}
	\begin{cPart}
		Assume that there is no element of order 2.
		
		For each $x\in G\setminus\{e\}$ we look at $\tilde x=\{x,x^{-1}\}$, then $G\setminus\{e\}=\bigcup_{x\in G\setminus\{e\}}\tilde x$. Because multiplication is injective, those sets are all disjoint then $2n=|G|=1+\sum_{0\le i <|G\setminus\{e\}|}|\tilde x|=1+2k$, contradiction.
	\end{cPart}
\end{cExercise}
\begin{cExercise}
	\begin{cPart}
		Because the determinate is multiplicative, $SL_n(\mathbb Z)$ is clearly closed under matrix multiplication, which inherent the associativity.\\
		Clearly $\det (I_n)=1$, so $SL_n(\mathbb Z)$ has an identity.\\
		Now if $A\in SL_n(\mathbb Z)$ then $1=\det(I_n)=\det (AA^{-1})=\det(A)\det(A^{-1})=\det(A^{-1})$, hence every $A\in SL_n(\mathbb Z)$ has an inverse.
	\end{cPart}
	\begin{cPart}
		For a matrix $A\in GL_n(\mathbb Z)$ with determinate $1$, we know that $\det(A)=\det(A^{-1})$, for $A$ to be invertible in $\mathbb{Z}$ it means that it's inverse also must have only integer values, in particular if we consider a minimal path in Gaussian elimination at no point we multiply a row by anything other than $-1$, otherwise we would have non-integer enteries in the inverse.
		
		Furthermore, after multiplying a row by $-1$ or switching $2$ rows we must either multiply another row by $-1$ or switch another $2$ rows.
		
		Now let $E_+$ be the set of elementary matrices that adds 2 rows, and let $E_\times$ be the elementary  matrices that multiply a row by $-1$, and let $E_s$ be the matrices that swap $2$ rows, from the argument above the set $E_+\cup\{ab\mid a,b\in E_\times\cup E_s\}$ generates $SL_n(\mathbb Z)$, and it is clearly finite.
	\end{cPart}
	\begin{cPart}
		Let $\tilde S_n$ be the set of matrices such that each row and column contains exactly one $1$. $|\tilde{S}_n|$ is clearly $n!$.\\
		The obviously the inverse of $A\in \tilde{S}_n$ is $A^\in \tilde{S}_n$, and a simply plugging in $A,B\in \tilde{S}_n$ to calculate $AB$ shows that $AB\in \tilde{S}_n$.
	\end{cPart}
	\begin{cPart}
		Given a field $\mathbb F_p$, a matrix $A\in M_n(\mathbb F_p)$ has an inverse if and only if all of it's rows are independent.\\
		In particular we can calculate all possible ways to choose $n$ independent vectors:\\
		The first vector can be anything but $0_{\mathbb F_p^n}$, so anything from a choice of $p^n-1$ elements.\\
		The $(k+1)^{\text{th}}$ vector can be anything that is not a linear combination of the previous rows, there are $p^k$ many possible linear combinations, so there are $p^n-p^k$ possible choices.
		
		Multiplying this all together and we get that $|GL_n(\mathbb F_p)|=\prod_{i=0}^{n-1}(p^n-p^i)$
	\end{cPart}
\end{cExercise}
\end{document}