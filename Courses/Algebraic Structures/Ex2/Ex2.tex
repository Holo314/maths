\documentclass[12pt,reqno]{article}
\usepackage[margin=3cm]{geometry}
\usepackage[utf8]{inputenc}


\usepackage{xfp}
\usepackage{natbib}
\usepackage{graphicx}
\usepackage{amsthm}
\usepackage{amsmath}
\usepackage{amssymb}
\usepackage{cases}
\usepackage{microtype}
\usepackage{hyperref}
\usepackage{mathrsfs}

\newtheorem{theorem}{Theorem}[section]
\newtheorem{proposition}[theorem]{Proposition}
\newtheorem{corollary}[theorem]{Corollary}
\newtheorem{lemma}[theorem]{Lemma}
\newtheorem{conjecture}[theorem]{Conjecture}
\newtheorem{example}[theorem]{Example}
\theoremstyle{definition}
\newtheorem{definition}[theorem]{Definition}
\theoremstyle{remark}
\newtheorem{remark}[theorem]{Remark}
\theoremstyle{exercise}
\newtheorem{exercise}[section]{Exercise}
\theoremstyle{subExercise}
\newtheorem{subExercise}{Part}[section]
\newtheorem{subSubExercise}{Sub Part}[subExercise]
\numberwithin{equation}{section}
\newcommand{\acl}[2]{\operatorname{acl}^{#2}\left(#1\right)}
\newcommand{\rank}[2]{\operatorname{rank}_{#2}\left(#1\right)}
\newcommand{\dcl}[2]{\operatorname{dcl}^{#2}\left(#1\right)}
\newcommand{\Aut}[1]{\operatorname{Aut}\left(#1\right)}
\newcommand{\Av}[1]{\operatorname{Av}\left(#1\right)}
\newcommand{\comment}[2]{#2}
\newcommand{\cof}[1]{\operatorname{cof}\left(#1\right)}
\renewcommand{\cal}[1]{\mathcal{#1}}
\newcommand{\D}[2][{}]{\text{Diag}_{#1}\left(#2\right)}
\renewcommand{\phi}{\varphi}
\newcommand{\code}[1]{\left\lceil#1\right\rceil}
\setcounter{MaxMatrixCols}{20}
\sloppy

\newcommand{\ex}[2][\fpeval{\value{section}+1}]{\setcounter{section}{\fpeval{#1 - 1}}
	\begin{exercise}
		#2
\end{exercise}}
\newcommand{\sub}[2][\fpeval{\value{subExercise}+1}]{\setcounter{subExercise}{\fpeval{#1 - 1}}
	\begin{subExercise}
		#2
\end{subExercise}}
\newcommand{\subb}[2][\fpeval{\value{subSubExercise}+1}]{\setcounter{subSubExercise}{\fpeval{#1 - 1}}
	\begin{subSubExercise}
		#2
\end{subSubExercise}}
\newcommand{\cl}{\mbox{cl}}
\newcommand{\CB}[1]{\operatorname{CB}\left(#1\right)}
\newcommand{\MR}[1]{\operatorname{MR}\left(#1\right)}
\newcommand{\MD}[1]{\operatorname{MD}\left(#1\right)}
\newcommand{\monster}{{\mathfrak C}}
\newcommand{\tp}[1]{\operatorname{tp}\left(#1\right)}
\newcommand{\hull}[3]{\operatorname{Hull}^{#2}_{#3}\left(#1\right)}
\newcommand{\stp}[1]{\operatorname{stp}\left(#1\right)}
\newcommand{\dom}[1]{\operatorname{dom}\left(#1\right)}
\newcommand{\range}[1]{\operatorname{range}\left(#1\right)}
\newcommand{\cnst}[2]{\operatorname{const}_{#1}\left(#2\right)}
\renewcommand{\c}{{\mathfrak c}}
\newcommand{\club}[1]{\operatorname{club}\left(#1\right)}
\newcommand{\Lev}[1]{\operatorname{Lev}\left(#1\right)}
\newcommand{\height}[1]{\operatorname{ht}\left(#1\right)}
\newcommand{\emptyseq}{\Lambda}

\newcommand{\N}{\mathbb N}
\newcommand{\Q}{\mathbb Q}
\newcommand{\R}{\mathbb R}
\newcommand{\C}{\mathbb C}
\newcommand{\F}{\mathbb F}




\usepackage{xparse}

\ExplSyntaxOn
\tl_new:N \l_septatrix_env_tl
\NewDocumentCommand \getenv { o m }
{
	\sys_get_shell:nnN { kpsewhich ~ --var-value ~ #2 }
	{ \int_set:Nn \tex_endlinechar:D { -1 } }
	\l_septatrix_env_tl
	\IfNoValueTF {#1}
	{ \tl_use:N \l_septatrix_env_tl }
	{ \tl_set_eq:NN #1 \l_septatrix_env_tl }
}
\tl_const:Nn \c_getenv_par_tl { \par }

\NewDocumentCommand{\ifenvsetTF}{mmm}
{
	\sys_get_shell:nnN { kpsewhich ~ --var-value ~ #1 } { } \l_tmpa_tl
	\tl_if_eq:NNTF \l_tmpa_tl \c_getenv_par_tl { #3 } { #2 }
}
\ExplSyntaxOff

\usepackage{datetime}
% \newdate{date}{06}{09}{2012}
% \date{\displaydate{date}}
\date{\today}
\newcommand{\envOrDefault}[2]{\ifenvsetTF{#1}{\getenv{#1}}{#2}}

\author{\envOrDefault{au4thor}{Holo}}



\usepackage{skak}
\usepackage{relsize}
\usepackage{graphicx}
\usepackage{mathtools}

\usepackage{textcomp}
\usepackage{bbding}

\usepackage{soul}

\newcommand{\flower}{\text{\scalebox{0.75}{\raisebox{-0.7ex}{
				\rotatebox{90}{\textleaf}\hspace{-0.3em}
				\scalebox{0.7}{\textleaf}\hspace{-1.35em}
				\raisebox{1ex}{\scalebox{0.8}{\FiveFlowerOpen}}
}}}}
\title{Exercise 2}
\begin{document}
\maketitle
\begin{cExercise}[][][author][1]
	\begin{cPart}
		Lets remember that a cycle is just a function $f$ for a finite set $X$ to itself such that $\operatorname{Supp}(f)=\{f^{(n)}(a)\}_{0\le n<|X|}$ for any $a\in \operatorname{Supp}(f)$.
		
		To see that the $2$-cycles, let's call this set $C_2$, generate the cycles (that we already saw generate the permutation group) we first note that $e\in\gen{C_2}$ because if $a,b\in X$ then $(a,b)(b,a)=e$. Now, let $f$ be a non-identity cycle, we know that $\operatorname{Supp}(f)\ne\emptyset$, let $a$ be such witness. Let $f_i$ be $(f^{i}(a),f^{i+1}(a))$, because $X$ is finite we must have $\{f_i\}_{i\in\omega}$ finite and clearly we have $f=f_0\circ f_1\circ\cdots\circ f_{|\{f_i\}_{i\in\omega}|-1}$.
	\end{cPart}
	\begin{cPart}
		Let the set of $(i,i+1)$ be $F_2$. To see that it generates $S_n$ we will show that $F_2$ generates $C_2$.
		
		Because $(a,b)^{-1}=(b,a)$ we may will always assume that when we write $(a,b)$ we have $a<b$, and because if $(g_i)$ is a sequence from $F_2$ that generates $(c-a,b-a)$ we can shift all of $g_i$ by $a$ to get $(c,b)$ it is enough to show that we generate $(1,k)$ to prove we generate $(a,b)$ for all $a,b$ with $b-a=k-1$.
		
		We will use induction on $k$ in $(1,k)$, starting with $2$. The base case is trivial so lets assume we generate $(1,k)$ and show $(1,k+1)$.
		
		We can compose $(1,k)$ with $(k,k+1)$ and then again with $(1,k)$ to get $g=(1,k)(k,k+1)(1,k)$. Clearly $g(i)=i$ for $i\notin\{1,k,k+1\}$, $g(k)=(1,k)(k,k+1)(1,k)k=(1,k)(k,k+1)1=(1,k)1=k$, $g(1)=(1,k)(k,k+1)(1,k)1=(1,k)(k,k+1)k=(1,k)k+1=k+1$ and $g(k+1)=(1,k)(k,k+1)(1,k)k+1=(1,k)(k,k+1)k+1=(1,k)k=1$, in other words $g=(1,k+1)$ and we are done.
	\end{cPart}
	\begin{cPart}
		First we notice that $(1,\ldots,n)k\equiv k+1\pmod{n}$, so let $1<p<n$ and look at $h=(1,\ldots,n)^{(p-1)}(1,2)(1,\ldots,n)^{(-(p-1))}$. Plugin the values of $p,p+1$ and $k\notin \{p,p+1\}$ we see that $h=(p,p+1)$.
	\end{cPart}
\end{cExercise}
\begin{cExercise}
	\begin{cPart}
		Clearly if $a$ is a multiply of $\operatorname{lcm}(d_1,d_2)$ then it is a multiply of both $d_1,d_2$, in other words $\gen{\operatorname{lcm}(d_1,d_2)}\subseteq \gen{d_1}\cap\gen{d_2}$.
		
		To see the other direction let $x\in \gen{d_1}\cap\gen{d_2}$, but this means that $x$ is a multiply of both $d_1$ and $d_2$, then $x=k\operatorname{lcm}(d_1,d_2)+r, r<\operatorname{lcm}(d_1,d_2)$, if $r\ne 0$ then it divides both $d_1,d_2$, contradiction to the minimality, so $x=k\operatorname{lcm}(d_1,d_2)\implies x\in \gen{\operatorname{lcm}(d_1,d_2)}$
	\end{cPart}
	\begin{cPart}
		Assume $\operatorname{lcm}(|g|,|h|)=k|gh|+r, r<|gh|$ and look at $(gh)^{\operatorname{lcm}(|g|,|h|)}=(gh)^{k|gh|+r}=(gh)^r$.
		
		But $(gh)^{\operatorname{lcm}(|g|,|h|)}=g^{\operatorname{lcm}(|g|,|h|)}h^{\operatorname{lcm}(|g|,|h|)}=0$, so $r$ must be divisible by both $|g|$ and $|h|$, which contradiction to the minimality of $\operatorname{lcm}(|g|,|h|)$ unless $r=0$.
	\end{cPart}
	\begin{cPart}
		We have that $C=AB=\begin{bmatrix}1&1\\0&1\end{bmatrix}$, with $C^n=\begin{bmatrix}1&n\\0&1\end{bmatrix}$.
		
		On the other hand, $A^4=I_2$ and $B^6=I_2$, in particular from the previous part, if $AB=BA$ then $(AB)^{\operatorname{lcm}(4,6)}=C^{\operatorname{lcm}(4,6)}=I_2\ne \begin{bmatrix}1&\operatorname{lcm}(4,6)\\0&1\end{bmatrix}$
	\end{cPart}
	\begin{cPart}
		It is enough to show that for 2 disjoint cycles $d,d'$ we have that $|dd'|=\operatorname{lcm}(|d|,|d'|)$.
		
		If $d^{|dd'|}$ or $d'^{|dd'|}$ is the identity we are done (as it implies that the other is the identity, and from the previous parts we get $|dd'|=\operatorname{lcm}(|d|,|d'|)$), but then $d^{|dd'|},d'^{|dd'|}$ are 2 disjoint nontrivial cycles, in particular $(dd')^{|dd'|}=d^{|dd'|}d'^{|dd'|}\ne e$, contradiction.
	\end{cPart}
\end{cExercise}
\begin{cExercise}
	\begin{cPart}
		If $|g|=n$ then the function $\mathbb Z_n\to G:k\mapsto g^{k}$ is clearly an isomorphism.
		
		Similarly, the same function with domain $\mathbb Z$ will be an isomorphism for $|g|=\infty$.
	\end{cPart}
	\begin{cPart}
		$(1,1), (1,1)+(1,1)=(0,2), (0,2)+(1,1)=(1,0), (1,0)+(1,1)=(0,1), (0,1)+(1,1)=(1,2), (1,2)+(1,1)=(0,0), (0,0)+(1,1)=(1,1)$ so $\gen{(1,1)}=\mathbb Z_2\times \mathbb Z_3$ and $|(1,1)|=6$, so from the previous part they are isomorphic.
	\end{cPart}
	\begin{cPart}
		$(a,b)+(a,b)=(0,0), (0,0)+(a,b)=(a,b)$ for all $(a,b)$, so $\mathbb Z_2\times\mathbb Z_2$ is not cyclic.
	\end{cPart}
\end{cExercise}
\begin{cExercise}
	If $H,K\ne H\cup K$, there exists $h\in H\setminus K, k\in K\setminus H$. If $hk\in H\cup K$ then it is in one of $H,K$, which is clearly a contradiction as we will have $h^{-1}hk\in H$ or $hkk^{-1}\in K$.
\end{cExercise}
\begin{cExercise}
	\begin{cPart}
		If $C\in \operatorname{SL}_n(\mathbb F_p)$ then clearly $\det(AC)=\det(A)\det(C)=\det(A)$ hence we have $A\cdot \operatorname{SL}_n(\mathbb F_p)\subseteq \{B\in \operatorname{GL}_n(\mathbb{F}_p)\mid \det(B)=\det(A)\}$.
		
		Take $B$ with $\det(B)=\det(A)$, then $\det(A^{-1}B)=\det(A^{-1})\det(B)=\det(A)^{-1}\det(B)=\det(B)^{-1}\det(B)=1$, hence $A^{-1}B\in\operatorname{SL}_n(\mathbb{F}_p)$.
	\end{cPart}
	\begin{cPart}
		For each $k\in[1,p]$ there exists a matrix $A_k$ with determinate $k$, all of which are in $\operatorname{GL}_n(\mathbb{F}_p)$ and those matrices bijects to $\operatorname{GL}_n(\mathbb{F}_p)/\operatorname{SL}_n(\mathbb{F}_p)$ by a natural map, composing $k\mapsto A_k\mapsto A_k\operatorname{SL}_n(\mathbb{F}_n)$ will finish the proof.
	\end{cPart}
\end{cExercise}
\begin{cExercise}
	\begin{cPart}
		First we will observe that $\sigma^{-1}=\sigma^{n-1},\tau^{-1}=\tau$.
		
		We will prove by induction on the length of the term that every $x\in D_n$ is either of the form $\sigma^k$ or $\tau\sigma^k$.
		
		To do this we notice that $\sigma\tau\sigma\tau=e\implies\sigma\tau=\tau^{-1}\sigma^{-1}=\tau\sigma^{n-1}$, which easily implies that $\sigma^{-1}\tau=\tau\sigma^{(n-1)^2}=\tau\sigma$. Indeed we can define an embedding $j:D_n\to S_n$ with $j(\tau)=(x\mapsto n-1-x)$ and $j(\sigma)=(x\mapsto x+1\pmod n)$, and then
		\begin{align*}
			&j(\sigma\tau\sigma\tau)&=&x\mapsto ((n-1-((n-1-x)+1))+1\pmod n)\\
			&&=&x\mapsto ((n-1-(n-x))+1\pmod n)\\
			&&=&x\mapsto ((x-1)+1\pmod n)\\
			&&=&x\mapsto x\\
			&&=&j(e)
		\end{align*}
		
		Now given $x\in D_n$ a term of length $p>2$, it is of the form $gh$ for $g\in\{\tau,\sigma\}$ and $h$ of length $p-1$, by the induction hypothesis $h$ is either of the form $\sigma^k$, in which case we are done, or of the form $\tau\sigma^k$. So $x=g(\tau\sigma^{k})=(g\tau)\sigma^k$, if $g=\tau$ we are done, otherwise $x=(\tau\sigma^{n-1})\sigma^k=\tau\sigma^{k-1}$.
	\end{cPart}
	\begin{cPart}
		Let $g,h\in D_n$, let's also assume neither of them is $e$.
		
		Let $g=\sigma^p,h=\sigma^q$, in this case $gh=\sigma^{p+q\pmod n}$.
		
		Let $g=\tau \sigma^p,h=\sigma^q$, in this case $gh=\tau\sigma^p\sigma^q=\tau\sigma^{p+q\pmod n}$.
		
		Let $g=\sigma^p,h=\tau\sigma^q$, in this case $gh=\sigma^p\tau\sigma^q$, from the observation we did in the previous part we can repeatedly move $\tau$ back using the identity $\sigma\tau=\tau\sigma^{-1}$, so $\sigma^p\tau=\tau\sigma^{-p}\implies gh=\sigma^p\tau\sigma^q=\tau\sigma^{q-p\pmod n}$
		
		Let $g=\tau \sigma^p,h=\tau\sigma^q$, in this case $gh=\tau\sigma^p\tau\sigma^q=\tau^2\sigma^{q-p\pmod n}=\sigma^{q-p\pmod n}$
	\end{cPart}
	\begin{cPart}
		From the previous part we can find for all $g=\sigma^k$ an $h$ such that $gh=\tau^i\sigma^{1-k},hg=\tau^i\sigma^{1+k}$, which are equal only for $k=n/2$. ($i\in\{0,1\}$)
		
		Similarly for $g=\tau\sigma^k$ we can find $h$ such that $gh=\tau^i\sigma^{k-1},hg=\tau^i\sigma^{k+1}$, which are never equal (unless $n=2,k=1$, in which case $\tau\sigma=e$).
		
		So all we need to check is $\sigma^{n/2}$, and quickly plugging it in the equations from the previous part we can see it works.
	\end{cPart}
\end{cExercise}
\end{document}