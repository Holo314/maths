\documentclass[12pt,reqno]{article}
\usepackage[margin=3cm]{geometry}
\usepackage[utf8]{inputenc}


\usepackage{xfp}
\usepackage{natbib}
\usepackage{graphicx}
\usepackage{amsthm}
\usepackage{amsmath}
\usepackage{amssymb}
\usepackage{cases}
\usepackage{microtype}
\usepackage{hyperref}
\usepackage{mathrsfs}
\usepackage{xparse}

\makeatletter
\DeclareRobustCommand\widecheck[1]{{\mathpalette\@widecheck{#1}}}
\def\@widecheck#1#2{%
	\setbox\z@\hbox{\m@th$#1#2$}%
	\setbox\tw@\hbox{\m@th$#1%
		\widehat{%
			\vrule\@width\z@\@height\ht\z@
			\vrule\@height\z@\@width\wd\z@}$}%
	\dp\tw@-\ht\z@
	\@tempdima\ht\z@ \advance\@tempdima2\ht\tw@ \divide\@tempdima\thr@@
	\setbox\tw@\hbox{%
		\raise\@tempdima\hbox{\scalebox{1}[-1]{\lower\@tempdima\box
				\tw@}}}%
	{\ooalign{\box\tw@ \cr \box\z@}}}
\makeatother

\newtheorem{theorem}{Theorem}[section]
\newtheorem{proposition}[theorem]{Proposition}
\newtheorem{corollary}[theorem]{Corollary}
\newtheorem{lemma}[theorem]{Lemma}
\newtheorem{conjecture}[theorem]{Conjecture}
\newtheorem{example}[theorem]{Example}
\theoremstyle{definition}
\newtheorem{definition}[theorem]{Definition}
\theoremstyle{remark}
\newtheorem{remark}[theorem]{Remark}
\theoremstyle{exercise}
\newtheorem{exercise}[section]{Exercise}
\theoremstyle{subExercise}
\newtheorem{subExercise}{Part}[section]
\newtheorem{subSubExercise}{Sub Part}[subExercise]
\numberwithin{equation}{section}
\newcommand{\acl}[2]{\operatorname{acl}^{#2}\left(#1\right)}
\newcommand{\rank}[2]{\operatorname{rank}_{#2}\left(#1\right)}
\newcommand{\dcl}[2]{\operatorname{dcl}^{#2}\left(#1\right)}
\newcommand{\Aut}[1]{\operatorname{Aut}\left(#1\right)}
\newcommand{\Av}[1]{\operatorname{Av}\left(#1\right)}
\newcommand{\comment}[2]{#2}
\newcommand{\cof}[1]{\operatorname{cof}\left(#1\right)}
\renewcommand{\cal}[1]{\mathcal{#1}}
\newcommand{\D}[2][{}]{\text{Diag}_{#1}\left(#2\right)}
\renewcommand{\phi}{\varphi}
\newcommand{\code}[1]{\left\lceil#1\right\rceil}
\setcounter{MaxMatrixCols}{20}
\sloppy

\usepackage{expl3}




\ExplSyntaxOn

\tl_new:N \l_septatrix_env_tl
\NewDocumentCommand \getenv { o m }
{
	\sys_get_shell:nnN { kpsewhich ~ --var-value ~ #2 }
	{ \int_set:Nn \tex_endlinechar:D { -1 } }
	\l_septatrix_env_tl
	\IfNoValueTF {#1}
	{ \l_septatrix_env_tl }
	{ #1 }
}

\NewDocumentCommand{\ifenvsetTF}{mmm}
{
	\sys_get_shell:nnN { kpsewhich ~ --var-value ~ #1 } { } \l_tmpa_tl
	\tl_if_eq:NNTF \l_tmpa_tl \c_getenv_par_tl { #3 } { #2 }
}

\tl_new:N \l_env_tl
\NewDocumentEnvironment{ifEnvEq}{ m m m +b}{
	\sys_get_shell:nnN { kpsewhich ~ --var-value ~ #1 }
		{ \int_set:Nn \tex_endlinechar:D { -1 } }
		\l_env_tl
	
	\cs_generate_variant:Nn \tl_if_eq:nnTF { o }
	\tl_if_eq:onTF { \l_env_tl } { #2 } { \stepcounter{#3} } { #4 }
}{}
\ExplSyntaxOff


% Optional parameters: Title, Value to blacklist, EnvVar, exercise number
\NewDocumentEnvironment{cExercise}{ O{} O{} O{author} O{\fpeval{\value{section}+1}} +b }{
	\setcounter{section}{\fpeval{#4 - 1}}
	\begin{ifEnvEq}{#3}{#2}{section}
		\begin{exercise}
			#1
		\end{exercise}
		#5
	\end{ifEnvEq}
}{}

% Optional parameters: Title, Value to blacklist, EnvVar, exercise number
\NewDocumentEnvironment{cPart}{ O{} O{} O{author} O{\fpeval{\value{subExercise}+1}} +b }{
	\setcounter{subExercise}{\fpeval{#4 - 1}}
	\begin{ifEnvEq}{#3}{#2}{subExercise}
		\begin{subExercise}
			#1
		\end{subExercise}
		#5
	\end{ifEnvEq}
}{}

% Optional parameters: Title, Value to blacklist, EnvVar, exercise number
\NewDocumentEnvironment{cSubPart}{ O{} O{} O{author} O{\fpeval{\value{subSubExercise}+1}} +b }{
	\setcounter{subSubExercise}{\fpeval{#4 - 1}}
	\begin{ifEnvEq}{#3}{#2}{subSubExercise}
		\begin{subSubExercise}
			#1
		\end{subSubExercise}
		#5
	\end{ifEnvEq}
}{}

% Legacy Environment Start
\newcommand{\ex}[2][\fpeval{\value{section}+1}]{\setcounter{section}{\fpeval{#1 - 1}}
	\begin{exercise}
		#2
	\end{exercise}
}
\newcommand{\sub}[2][\fpeval{\value{subExercise}+1}]{\setcounter{subExercise}{\fpeval{#1 - 1}}
	\begin{subExercise}
		#2
	\end{subExercise}
}
\newcommand{\subb}[2][\fpeval{\value{subSubExercise}+1}]{\setcounter{subSubExercise}{\fpeval{#1 - 1}}
	\begin{subSubExercise}
		#2
	\end{subSubExercise}
}
% Legacy Environment End
\newcommand{\cl}[2]{\mbox{cl}_{#2}\left(#1\right)}
\newcommand{\CB}[1]{\operatorname{CB}\left(#1\right)}
\newcommand{\MR}[1]{\operatorname{MR}\left(#1\right)}
\newcommand{\MD}[1]{\operatorname{MD}\left(#1\right)}
\newcommand{\monster}{{\mathfrak C}}
\newcommand{\tp}[1]{\operatorname{tp}\left(#1\right)}
\newcommand{\hull}[3]{\operatorname{Hull}^{#2}_{#3}\left(#1\right)}
\newcommand{\stp}[1]{\operatorname{stp}\left(#1\right)}
\newcommand{\dom}[1]{\operatorname{dom}\left(#1\right)}
\newcommand{\range}[1]{\operatorname{range}\left(#1\right)}
\newcommand{\cnst}[2]{\operatorname{const}_{#1}\left(#2\right)}
\renewcommand{\c}{{\mathfrak c}}
\newcommand{\club}[1]{\operatorname{club}\left(#1\right)}
\newcommand{\Lev}[1]{\operatorname{Lev}\left(#1\right)}
\newcommand{\height}[1]{\operatorname{ht}\left(#1\right)}
\newcommand{\emptyseq}{\Lambda}

\newcommand{\N}{\mathbb N}
\newcommand{\Q}{\mathbb Q}
\newcommand{\R}{\mathbb R}
\newcommand{\C}{\mathbb C}
\newcommand{\F}{\mathbb F}
\renewcommand{\P}{\mathbb P}
\renewcommand{\Bbb}[1]{\mathbb #1}
\newcommand{\incomp}{\operatorname{\bot}}
\newcommand{\comp}{\operatorname{\|}}
\newcommand{\force}{\Vdash}
\newcommand{\Add}[2]{\operatorname{Add}\left(#1,#2\right)}

\usepackage{xparse}


\ExplSyntaxOn
\NewExpandableDocumentCommand \randint { m m }
{ \int_rand:nn { #1 } { #2 } }
\ExplSyntaxOff
\newcounter{cntTherefore}
\newcommand\setTherefore[2]{%
	\csdef{Therefore#1}{#2}}
\newcommand\addTherefore[1]{%
	\stepcounter{cntTherefore}%
	\csdef{Therefore\thecntTherefore}{#1}}
\newcommand\getTherefore[1]{%
	\csuse{Therefore#1}}
\newcommand{\Therefore}{\getTherefore{\randint{1}{\thecntTherefore}} }
\addTherefore{therefore}
\addTherefore{hence}
\addTherefore{which implies}
\addTherefore{consequently}
\addTherefore{it follows that}




\newcommand{\envOrDefault}[2]{\ifenvsetTF{#1}{\getenv{#1}}{#2}}

\newcommand{\envWithDefault}[1]{\envOrDefault{#1}{Holo}}

\def\bonktxt#1{% 
	\quitvmode\hbox{% 
		\pdfliteral{q 1 0 .15 .4 0 0 cm}\rlap{#1}\pdfliteral{Q}\hphantom{#1}% 
		\pdfliteral{q .9063 .4226 -.4226 .9063 -3 6 cm .2 0 0 .2 0 0 cm}\llap\bonktext\pdfliteral{Q}\ % 
		\pdfliteral{q 
			.81914 .57356 -.57356 .81914 -3 4 cm 
			1.7 0 0 1.7 0 0 cm 1 j 1 J .7 w 
			0 0 m 0 .7 l 2 .7 l 2 0 l b 
			0 j .3 w .2 -.4 m .2 1.1 l s 
			1.8 -.4 m 1.8 1.1 l s 
			.8 1.05 m .85 1.15 1.15 1.15 1.2 1.05 c 
			1 0 m 1 -1.4 l s 
			1.1 -1.4 m 1.2 -2 1.1 -3 y s 
			.9 -1.4 m .8 -2 .9 -3 y s 
			Q}\kern3pt\relax 
	}% 
} 
\def\bonktext{bonk!}
\newcommand{\bonk}[1]{\bonktxt{$#1$}}

\usepackage{datetime}
% \newdate{date}{06}{09}{2012}
% \date{\displaydate{date}}
\date{\today}


\author{\envWithDefault{author}}



\usepackage{skak}
\usepackage{relsize}
\usepackage{graphicx}
\usepackage{mathtools}

\usepackage{textcomp}
\usepackage{bbding}

\usepackage{soul}

\newcommand{\flower}{\text{\scalebox{0.75}{\raisebox{-0.7ex}{
				\rotatebox{90}{\textleaf}\hspace{-0.3em}
				\scalebox{0.7}{\textleaf}\hspace{-1.35em}
				\raisebox{1ex}{\scalebox{0.8}{\FiveFlowerOpen}}
}}}}
\title{Exercise 2}
\begin{document}
\maketitle
\begin{cExercise}[][][author][1]
	\begin{cPart}
		Let $a_x=\begin{pmatrix}
			1 & x\\
			0 & 1
		\end{pmatrix}, a_y=\begin{pmatrix}
			1 & y\\
			0 & 1
		\end{pmatrix}\in G$, then we have $a_xa_y=\begin{pmatrix}
			1\cdot 1+x\cdot 0 & 1\cdot y + x\cdot 1\\
			0\cdot 1+0\cdot 1 & 0\cdot y + 1\cdot 1
		\end{pmatrix}=a_{x+y}=\begin{pmatrix}
			1 & x+y\\
			0 & 1
		\end{pmatrix}$.
		
		To see that $x\mapsto a_x$ is an isomorphism we need to show it is a bijection (which is obvious by the definition, alternatively, $\begin{pmatrix}
			1 & x\\
			0 & 1
		\end{pmatrix}\mapsto x$ is an inverse function, which exists iff the function is a bijection), that it sends $e_{\mathbb F^+}$ to $e_G=I_2$ (which is true because $e_{\mathbb F^+}=0_{\mathbb F}$), and that it preserves the group operator, which is shown to be true in the starting sentence.
	\end{cPart}
	\begin{cPart}
		We will show that $x\mapsto \exp(x)$ is an isomorphism from $\R^+$ to $\R_{>0}^\times$.
		
		Clearly $\exp(x+y)=\exp(x)\exp(y)$, and $\exp(0)=1$, and it is a strictly monotonic continuous function with $\lim_{x\to-\infty}\exp(x)=0, \lim_{x\to\infty}\exp(x)=\infty$, so it's injective range is $(0,\infty)$, hence bijective (to $\R_{>0}^\times$).
	\end{cPart}
	\begin{cPart}
		Let $f$ defined as:
		\begin{align*}
			&1\overset{f}{\mapsto} (0,0)\\
			&3\mapsto (0,1)\\
			&5\mapsto (1,0)\\
			&7\mapsto (1,1)\
		\end{align*}
		This is clearly a bijection and it sends the identity to the identity.
		
		We just need to check how $3,5,7$ interact under $f$ (as $1$ is sent to the identity)
		\begin{align*}
			&3,5:&f(7)=f(15)=f(3\cdot 5)=f(3)+f(5)=(1,1)\\
			&3,7:&f(5)=f(21)=f(3\cdot 7)=f(3)+f(7)=(1,0)\\
			&5,7:&f(3)=f(35)=f(5\cdot 7)=f(5)+f(7)=(0,1)
		\end{align*}
		Because the groups are Abelian, we are done.
	\end{cPart}
	\begin{cPart}
		The center $Z(S_4)$ is trivial, as if $p\in S_4$ moves $i\mapsto j$, and $k,\ell\ne i,j$ then $j=p\circ(i,k)(k)$ but $p(k)$ is one of $i,k,\ell$, non of which $(i,k)$ sends to $j$.
		
		On the other hand we saw that $Z(D_n)$ is not trivial for even $n$.
	\end{cPart}
	\begin{cPart}
		Every element of $\C^\times$ has a root, but not every element of $\R^\times$ has a root.
	\end{cPart}
	\begin{cPart}[Bonus]
		Let $(p_i)_{i\in\omega}$ be the prime numbers, and define $f:\{p_i\}_{i\in\omega}\to \Z[x]^+$ defined by $f(p_i)=x^i$.
		
		This function can be extend into $F$ a function on all of $\Q^\times_{>0}$ using $F(xy)=F(x)+F(y)$ and $F(p_i)=f(p_i)$.
		
		This function is surjective as $z_0\cdot x^i+z_1\cdot x^j=F(p_i^{z_0}p_j^{z_1})$, it is injective by the fundamental theorem of arithmetic, and it respects the operator by definition.
	\end{cPart}
\end{cExercise}
\begin{cExercise}
	\begin{cPart}
		Let $H=K\le D_3$ be the subgroups $\{e,\tau\}$ (this is a group as $\tau^2=e$), in this case $HK=H$ is a group.
		
		Let $K=\{e,\tau\sigma\}$ (this is a subgroup as $\tau\sigma\tau\sigma=e$), and let $H$ as before, then $HK=\{e,\sigma,\tau,\tau\sigma\}$, but this is not a group as $\sigma^2\notin HK$.
	\end{cPart}
	\begin{cPart}
		Assume $HK$ is a group, let  $h\in H,k\in K$, then $h^{-1}k^{-1}\in HK$, then $kh=(h^{-1}k^{-1})^{-1}\in HK$ so $KH\subseteq HK$.
		
		Now we want to show that $hk\in KH$, but from before $k^{-1}h^{-1}\in HK$ so $k^{-1}h^{-1}=pq$ for $p\in H,q\in K$ which implies $hk=(pq)^{-1}=q^{-1}p^{-1}\in KH$.
		
		Now assume $HK=KH$, clearly $e\in HK$ and $HK$ is closed under $(\text{---})^{-1}$, let $ab,xy\in HK$ with $a,y\in H,b,x\in K$, we have that $abx\in HK$, so it is in $KH$ and equal to $tr, t\in K,r\in H$, which gives $abxy=try\in KH=HK$. 
	\end{cPart}
	\begin{cPart}
		We have that for $ab\in HK$ and $x\in H\cap K$ (hence $x^{-1}$ is in there) we have $ab=axx^{-1}b$, and because multiplication by $a$ and multiplication by $b$ are bijections we don't have repetition.
		
		For each $g\in HK$ let $h_g\in H,k_g\in K$ such that $h_gk_g=g$.
		
		Let $hk=g$, so $hk=h_gk_g\implies h_g^{-1}h=k^{-1}k_g\in H\cap K$ and $(h_g^{-1}h)h_g=h$ and $k=k_g(k^{-1}k_g)^{-1}=k_g(h_g^{-1}h)^{-1}$.
		
		So the function $(h,k)\mapsto (hk,h_g^{-1}h)$ is a bijection from $|H||K|$ to $|HK||H\cap K|$
	\end{cPart}
	\begin{cPart}
		We have that $|HK|\le |G|$ so $|G|<(1+\sqrt{|G|})^2\le |H||K|=|HK||H\cap K|\le|G||H\cap K|\implies |H\cap K|>1$
	\end{cPart}
	\begin{cPart}
		Let $H<G$ be a subgroup of order $q$ and let $e\ne h\in H$, if $\gen{h}\ne H$ then the order of $h$ will divide $q$ but not be $1,q$, the same argument gives that $H$ has non non-trivial subgroups. If $K<G$ is another such group, it is generated from $k\in K$.
		
		From the previous part we have that for some $n,m<q$ we have $k^n=h^m\implies k=h^{m-n}\implies k\in \gen{h}\implies \gen{k}\le\gen{h}\implies \gen{k}=\gen{h}$
	\end{cPart}
\end{cExercise}
\begin{cExercise}
	\begin{cPart}
		\begin{itemize}
			\item It is faithful: given $g\in S_n$ if $gx=x$ for all $x\in [n]$ then it is the identity function by definition.
			
			\item It is transitive: given $x,y\in [n]$ we have that $(x,y)x=y$.
			
			\item It is not free for $n\ne 2$: $(1,2)$ moves $1$ and fixes $3$
			
			\item The orbit $O(1),O(n)$ are both $n$, as for every $k\in n$ we have $(1,k),(k,n)$ that witness that $k$ is in the orbit.
			
			\item The stabilizer $G_1=S_{[n]\setminus \{1\}}$ and $G_n=S_{[n-1]}=S_{n-1}$.
			
			\item The size of the orbits is $n$ and the size of the stabilizers is $|S_{n-1}|=(n-1)!=n!/n=|S_n|/|O(n)|$
		\end{itemize}
	\end{cPart}
	\begin{cPart}
		\begin{itemize}
			\item It is faithful: given $g\ne id$, and $gk=j$ for $k\ne j$, then $g\{k,j+1\}=\{j,g(j+1)\}\ne \{k,j+1\}$
			\item It is transitive: given $\{a,b\},\{c,d\}$, then $(a,c)(b,d)\{a,b\}=\{c,d\}$ where $d\ne a$, if they are equal use $(b,c)$ instead.
			\item It is never free: $(1,2)\{1,2\}=\{1,2\}$
			\item $O(\{1,n\})=[[n]]^2$ from transitivity
			\item $G_{\{1,n\}}=\{g\in G\mid \{g1,gn\}=\{1,n\}\}=S_{[n]\setminus \{1,n\}}\cup\{g\circ(1,n)\mid g\in S_{[n]\setminus \{1,n\}}\}$
			\item $|O(\{1,n\})|=|[[n]]^2|=\binom{n}{2}=n\cdot (n-1)/2$
			\item $|G_{\{1,n\}}|=(n-2)!+(n-2)!=2(n-2)!=|G|/|O(\{1,n\})|$
		\end{itemize}
	\end{cPart}
	\begin{cPart}
		\begin{itemize}
			\item It is faithful: Each vertex can any other vertex using only rotation
			\item It is transitive: It is a subgroup of a transitive group $S_n$
			\item It is not free: reflection that passes through a vertex will fix those and only those vertex, so it is not the identity and has fix points. 
			\item Like the previous examples, the orbit is the whole domain of the group action as the action is transitive.
			\item The stabilizer of a vertex is only the identity and the reflection that passes through this vertex
			\item The cardinality of the orbit is $n$
			\item The order of the stabilizer is $2$
		\end{itemize}
	\end{cPart}
\end{cExercise}
\begin{cExercise}
	Let $n$ be even.
	
	Every element is of the form $e,\sigma^k,\tau\sigma^k$.
	
	We shall calculate the conjugacy classes of $\sigma^k$ first: $\sigma^p\sigma^k\sigma^{-p}=\sigma^{p+k-p}=\sigma^k$, $\tau\sigma^p\sigma^k\sigma^{-p}\tau=\tau\sigma^k\tau=\tau\tau\sigma^{n-k}=\sigma^{n-k}$, so the conjugacy class is $\{\sigma^k,\sigma^{n-k}\}$.
	
	Moving on to $\tau$: $\sigma^p\tau\sigma^{-p}=\tau\sigma^{-2p}=\tau\sigma^{n-2p}$, $\tau\sigma^p\tau\sigma^{-p}\tau=\sigma^{n-2p}\tau=\tau\sigma^{2p}$, so the conjugacy class of $\tau$ is $\tau\sigma^{2k}$ ($0\le k<n/2$). (We use the fact that $n$ is even as we claim that $2p\pmod n=2k$ for $p<n$)
	
	Lastly, $\tau\sigma$:
	$\sigma^p\tau\sigma\sigma^{-p}=\tau\sigma^{1-2p}=\tau\sigma^{n-2p+1}$, $\tau\sigma^p\tau\sigma\sigma^{-p}\tau=\sigma^{n-2p+1}\tau=\tau\sigma^{2p-1}$, so the conjugacy class of $\tau$ is $\tau\sigma^{2k+1}$ ($0\le k<n/2$). (We use the fact that $n$ is even as we claim that $2p + 1\pmod n=2k + 1$ for $p<n$)
	
	For $n$ odd, the conjugacy classes we calculated for $\sigma^k$ don't change, but now $2k$ for $p<n$ will generate all of the values between $0$ and $n-1$, so the conjugacy class of $\tau$ is $\tau\sigma^k$ (for $0\le k<n$)
\end{cExercise}
\end{document}