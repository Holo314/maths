\documentclass[12pt,reqno]{article}
\usepackage[margin=3cm]{geometry}
\usepackage[utf8]{inputenc}


\usepackage{xfp}
\usepackage{natbib}
\usepackage{graphicx}
\usepackage{amsthm}
\usepackage{amsmath}
\usepackage{amssymb}
\usepackage{cases}
\usepackage{microtype}
\usepackage{hyperref}
\usepackage{mathrsfs}

\newtheorem{theorem}{Theorem}[section]
\newtheorem{proposition}[theorem]{Proposition}
\newtheorem{corollary}[theorem]{Corollary}
\newtheorem{lemma}[theorem]{Lemma}
\newtheorem{conjecture}[theorem]{Conjecture}
\newtheorem{example}[theorem]{Example}
\theoremstyle{definition}
\newtheorem{definition}[theorem]{Definition}
\theoremstyle{remark}
\newtheorem{remark}[theorem]{Remark}
\theoremstyle{exercise}
\newtheorem{exercise}[section]{Exercise}
\theoremstyle{subExercise}
\newtheorem{subExercise}{Part}[section]
\newtheorem{subSubExercise}{Sub Part}[subExercise]
\numberwithin{equation}{section}
\newcommand{\acl}[2]{\operatorname{acl}^{#2}\left(#1\right)}
\newcommand{\rank}[2]{\operatorname{rank}_{#2}\left(#1\right)}
\newcommand{\dcl}[2]{\operatorname{dcl}^{#2}\left(#1\right)}
\newcommand{\Aut}[1]{\operatorname{Aut}\left(#1\right)}
\newcommand{\Av}[1]{\operatorname{Av}\left(#1\right)}
\newcommand{\comment}[2]{#2}
\newcommand{\cof}[1]{\operatorname{cof}\left(#1\right)}
\renewcommand{\cal}[1]{\mathcal{#1}}
\newcommand{\D}[2][{}]{\text{Diag}_{#1}\left(#2\right)}
\renewcommand{\phi}{\varphi}
\newcommand{\code}[1]{\left\lceil#1\right\rceil}
\setcounter{MaxMatrixCols}{20}
\sloppy

\newcommand{\ex}[2][\fpeval{\value{section}+1}]{\setcounter{section}{\fpeval{#1 - 1}}
	\begin{exercise}
		#2
\end{exercise}}
\newcommand{\sub}[2][\fpeval{\value{subExercise}+1}]{\setcounter{subExercise}{\fpeval{#1 - 1}}
	\begin{subExercise}
		#2
\end{subExercise}}
\newcommand{\subb}[2][\fpeval{\value{subSubExercise}+1}]{\setcounter{subSubExercise}{\fpeval{#1 - 1}}
	\begin{subSubExercise}
		#2
\end{subSubExercise}}
\newcommand{\cl}{\mbox{cl}}
\newcommand{\CB}[1]{\operatorname{CB}\left(#1\right)}
\newcommand{\MR}[1]{\operatorname{MR}\left(#1\right)}
\newcommand{\MD}[1]{\operatorname{MD}\left(#1\right)}
\newcommand{\monster}{{\mathfrak C}}
\newcommand{\tp}[1]{\operatorname{tp}\left(#1\right)}
\newcommand{\hull}[3]{\operatorname{Hull}^{#2}_{#3}\left(#1\right)}
\newcommand{\stp}[1]{\operatorname{stp}\left(#1\right)}
\newcommand{\dom}[1]{\operatorname{dom}\left(#1\right)}
\newcommand{\range}[1]{\operatorname{range}\left(#1\right)}
\newcommand{\cnst}[2]{\operatorname{const}_{#1}\left(#2\right)}
\renewcommand{\c}{{\mathfrak c}}
\newcommand{\club}[1]{\operatorname{club}\left(#1\right)}
\newcommand{\Lev}[1]{\operatorname{Lev}\left(#1\right)}
\newcommand{\height}[1]{\operatorname{ht}\left(#1\right)}
\newcommand{\emptyseq}{\Lambda}

\newcommand{\N}{\mathbb N}
\newcommand{\Q}{\mathbb Q}
\newcommand{\R}{\mathbb R}
\newcommand{\C}{\mathbb C}
\newcommand{\F}{\mathbb F}




\usepackage{xparse}

\ExplSyntaxOn
\tl_new:N \l_septatrix_env_tl
\NewDocumentCommand \getenv { o m }
{
	\sys_get_shell:nnN { kpsewhich ~ --var-value ~ #2 }
	{ \int_set:Nn \tex_endlinechar:D { -1 } }
	\l_septatrix_env_tl
	\IfNoValueTF {#1}
	{ \tl_use:N \l_septatrix_env_tl }
	{ \tl_set_eq:NN #1 \l_septatrix_env_tl }
}
\tl_const:Nn \c_getenv_par_tl { \par }

\NewDocumentCommand{\ifenvsetTF}{mmm}
{
	\sys_get_shell:nnN { kpsewhich ~ --var-value ~ #1 } { } \l_tmpa_tl
	\tl_if_eq:NNTF \l_tmpa_tl \c_getenv_par_tl { #3 } { #2 }
}
\ExplSyntaxOff

\usepackage{datetime}
% \newdate{date}{06}{09}{2012}
% \date{\displaydate{date}}
\date{\today}
\newcommand{\envOrDefault}[2]{\ifenvsetTF{#1}{\getenv{#1}}{#2}}

\author{\envOrDefault{au4thor}{Holo}}



\usepackage{skak}
\usepackage{relsize}
\usepackage{graphicx}
\usepackage{mathtools}

\usepackage{textcomp}
\usepackage{bbding}

\usepackage{soul}

\newcommand{\flower}{\text{\scalebox{0.75}{\raisebox{-0.7ex}{
				\rotatebox{90}{\textleaf}\hspace{-0.3em}
				\scalebox{0.7}{\textleaf}\hspace{-1.35em}
				\raisebox{1ex}{\scalebox{0.8}{\FiveFlowerOpen}}
}}}}
\title{Exercise 5}
\begin{document}
\maketitle
\begin{cExercise}[][][author][1]
	\begin{cPart}
		Let $\phi:\Z^2\to\Z\times \Z_m$ defined as $\phi(a,b)=(a-b,b\pmod m)$.
		
		We have $(a-b,b\pmod m)+(x-y,y\pmod m)=(a-b+(x-y),(b\pmod m + y\pmod m)\pmod m)=((a+x)-(b+y), (b + y)\pmod m)$ so $\phi$ is an homomorphism.
		
		Given $(x,y)\in \Z\times \Z_m$ we have $\phi(x+y, y)=(x,y)$ so this is surjective.
		
		If $\phi(x,y)=e$ we have that $x=y$ and $y\equiv_m 0$, in particular $(x,y)=(nm,nm)$ for some integer $n$, hence $\ker(\phi)=\gen{(m,m)}$
	
		By the first iso-theorem we have that $\Z^2/\gen{(m,m)}\cong \Z\times\Z_m$.
	\end{cPart}
	\begin{cPart}
		Let $\phi:\R^2\to S_1\times S_1$ defined as $(x,y)\mapsto (\exp(2\pi x), \exp(2\pi y))$.
		Clearly this is a surjective homomorphism with kernel being the $\Z^2$, hence from the first iso-theorem the result follows.
	\end{cPart}
	\begin{cPart}
		We have that $\Q/\Z\cong \Q\cap [0,1)$ where the latter is with addition mod 1. This can be seen using the homomorphism $x\mapsto x-\lfloor x\rfloor$.
		
		Let $x\in \Q\cap [0,1)$, we have that $x=\frac ab$ for $b\in\N\setminus \{0\}$, in particular $x^b=0$, so the order of $x$ is at most $b\le\infty$.
		
		If we replace $\Q$ with $\R$ we still have the quotient isomorphic to $[0,1)\cap \R=[0,1)$, but then the order of $\frac1\pi$ is not finite, as it would imply that $\pi$ is rational.
	\end{cPart}
\end{cExercise}
\begin{cExercise}
	\begin{cPart}
		Let $k$ be the order of $g$, and $n$ the order of $gN$.
		
		We have that $g^k=e\implies (gN)^k=g^kN=eN=N=e_{G/N}\implies n|k$
	\end{cPart}
	\begin{cPart}
		We know that an order of an element divides the order of the group, so $n|[G:N]$, in particular $mn=[G:N]$ hence $g^{[G:N]}N=(gN)^{[G:N]}=e_{G/N}=N$.
		
		From that $g^{[G:N]}\in N$ follows.
	\end{cPart}
\end{cExercise}
\begin{cExercise}
	We know that $G/Z(G)$ must be either trivial, in which case $G=Z(G)$ and $G$ is Abelian, or $G$, in this case $Z(G)=1$, or $G/Z(G)$ has an order $p$, in which case it is cyclic hence $G$ is Abelian.
	
	We shall show that $Z(G)$ cannot be trivial, indeed if it was then we would get from the Conjugacy class equation that $p^2=|G|=|Z(G)|+\sum\cdots$ where each term of the sum is an order of a quotient, hence $p$ divides it. This gives us that $p$ must divides $|Z(G)|\implies |Z(G)|\ne 1$.
\end{cExercise}
\begin{cExercise}
	\begin{cPart}
		We have that $ijk=-1$ so $k=-jjk=j(-j)k=jiijk=(ji)(ijk)=-ji\implies ji=-k$.
		
		We have that $-kk=-(-1)=1$, hence $ij=(-i)(-j)=(-i)^{-1}(-j)^{-1}=((-j)(-i))^{-1}=(ji)^{-1}=(-k)^{-1}=k$.
	\end{cPart}
	\begin{cPart}
		In $D_4$ we have 6 elements of order $2$, $e,\sigma^2,\tau,\tau\sigma,\tau\sigma^2,\tau\sigma^3$, but in $\H$ we only have 2, $\pm1$, the rest of order 4.
	\end{cPart}
	\begin{cPart}
		We know that $\pm 1\in Z(\H)$, and that $i,j\notin Z(\H)$.
		
		If $k$ were in $Z(\H)$ then we would get $kij=-1$, and a symmetric argument of what we did in 4.1 would give that $ki=-ik$, and $k,i$ won't don't commute.
		
		Therefore $Z(\H)=\{1,-1\}$.
	\end{cPart}
	\begin{cPart}
		We have at least the subgroups $1,\gen{-1},\gen{i},\gen{j},\gen{k}$.
		
		Each of the latter 3 are of order $4$ hence cannot be extended to a different proper subgroup, and the first 2 are subgroups of the latter 3, hence those are the only proper subgroups.
		
		The first 2 subgroups are clearly normal as they are subgroups of the center, and the latter 3 are also normal because they have index 2.
	\end{cPart}
	\begin{cPart}
		We know that $1$ is one of the conjugacy classes and that $\{-1\}$ is another, as $1,-1\in Z(\H)$.
		
		We shall look at $\Cl(i)$, We have $jij^{-1}=-jij=-jk$, because $ijk=-1$ we have $jk=i$, hence $jij^{-1}=-i$. $kik^{-1}=-ijiij=ijj=-i$, so we get that $\Cl(i)=\{i,-i\}$.
		
		We notice that $ijk=-1\implies -jk=-i\implies -jki=1\implies jki=-1$, so from symmetry $kij=-1$ and $\Cl(k)=\{k,-k\},\Cl(j)=\{j,-j\}$
	\end{cPart}
\end{cExercise}
\begin{cExercise}
	We have that $D_5=\{e,\sigma,\sigma^2,\sigma^3,\sigma^4,\tau,\tau\sigma,\tau\sigma^2,\tau\sigma^3,\tau\sigma^4\}$.
	
	We have that $e$ is of order 1, $\sigma,\sigma^2,\sigma^3,\sigma^4$ of order 5, and $\tau,\tau\sigma,\tau\sigma^2,\tau\sigma^3,\tau\sigma^4$ of order 2.
	
	Furthermore, if $\phi:D_5\to D_5$ is automorphism, it completely determined from $\phi(\sigma),\phi(\tau)$.
	
	Because automorphism preserves order, it must sends $\sigma$ to an element of order 5, and $\tau$ to an element of order 2.
	
	Let $0<r,t<5$ such that $\phi(\sigma)=\sigma^r$ and $\phi(\tau)=\tau\sigma^t$, we shall show that this induces an automorphism and hence all possible functions of that form are automorphism and we shall achieve that $|\Aut{D_5}|=20$.
	
	We extend $\phi$ to the function $\phi(e)=e,\phi(\tau\sigma^j)=\tau\sigma^{jr+t\pmod 5},\phi(\sigma^i)=\sigma^{ir\pmod 5}$. The fact that $\phi$ preserve the group operation when multiplying $\sigma^j$ with anything is almost by definition, so we shall only check $\phi(\tau\sigma^j)\phi(\tau\sigma^i)=\tau\sigma^{jr+t\pmod 5}\tau\sigma^{ir+t\pmod 5}=\sigma^{-jr-t\pmod 5}\sigma^{ir+t\pmod 5}=\sigma^{(i-j)r\pmod 5}=\phi(\sigma^{i-j})=\phi(\tau\tau\sigma^{i-j})=\phi(\tau\sigma^{j}\tau\sigma^i)$
\end{cExercise}
\end{document}