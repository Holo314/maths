\documentclass[12pt,reqno]{article}
\usepackage[margin=3cm]{geometry}
\usepackage[utf8]{inputenc}


\usepackage{xfp}
\usepackage{natbib}
\usepackage{graphicx}
\usepackage{amsthm}
\usepackage{amsmath}
\usepackage{amssymb}
\usepackage{cases}
\usepackage{microtype}
\usepackage{hyperref}
\usepackage{mathrsfs}

\newtheorem{theorem}{Theorem}[section]
\newtheorem{proposition}[theorem]{Proposition}
\newtheorem{corollary}[theorem]{Corollary}
\newtheorem{lemma}[theorem]{Lemma}
\newtheorem{conjecture}[theorem]{Conjecture}
\newtheorem{example}[theorem]{Example}
\theoremstyle{definition}
\newtheorem{definition}[theorem]{Definition}
\theoremstyle{remark}
\newtheorem{remark}[theorem]{Remark}
\theoremstyle{exercise}
\newtheorem{exercise}[section]{Exercise}
\theoremstyle{subExercise}
\newtheorem{subExercise}{Part}[section]
\newtheorem{subSubExercise}{Sub Part}[subExercise]
\numberwithin{equation}{section}
\newcommand{\acl}[2]{\operatorname{acl}^{#2}\left(#1\right)}
\newcommand{\rank}[2]{\operatorname{rank}_{#2}\left(#1\right)}
\newcommand{\dcl}[2]{\operatorname{dcl}^{#2}\left(#1\right)}
\newcommand{\Aut}[1]{\operatorname{Aut}\left(#1\right)}
\newcommand{\Av}[1]{\operatorname{Av}\left(#1\right)}
\newcommand{\comment}[2]{#2}
\newcommand{\cof}[1]{\operatorname{cof}\left(#1\right)}
\renewcommand{\cal}[1]{\mathcal{#1}}
\newcommand{\D}[2][{}]{\text{Diag}_{#1}\left(#2\right)}
\renewcommand{\phi}{\varphi}
\newcommand{\code}[1]{\left\lceil#1\right\rceil}
\setcounter{MaxMatrixCols}{20}
\sloppy

\newcommand{\ex}[2][\fpeval{\value{section}+1}]{\setcounter{section}{\fpeval{#1 - 1}}
	\begin{exercise}
		#2
\end{exercise}}
\newcommand{\sub}[2][\fpeval{\value{subExercise}+1}]{\setcounter{subExercise}{\fpeval{#1 - 1}}
	\begin{subExercise}
		#2
\end{subExercise}}
\newcommand{\subb}[2][\fpeval{\value{subSubExercise}+1}]{\setcounter{subSubExercise}{\fpeval{#1 - 1}}
	\begin{subSubExercise}
		#2
\end{subSubExercise}}
\newcommand{\cl}{\mbox{cl}}
\newcommand{\CB}[1]{\operatorname{CB}\left(#1\right)}
\newcommand{\MR}[1]{\operatorname{MR}\left(#1\right)}
\newcommand{\MD}[1]{\operatorname{MD}\left(#1\right)}
\newcommand{\monster}{{\mathfrak C}}
\newcommand{\tp}[1]{\operatorname{tp}\left(#1\right)}
\newcommand{\hull}[3]{\operatorname{Hull}^{#2}_{#3}\left(#1\right)}
\newcommand{\stp}[1]{\operatorname{stp}\left(#1\right)}
\newcommand{\dom}[1]{\operatorname{dom}\left(#1\right)}
\newcommand{\range}[1]{\operatorname{range}\left(#1\right)}
\newcommand{\cnst}[2]{\operatorname{const}_{#1}\left(#2\right)}
\renewcommand{\c}{{\mathfrak c}}
\newcommand{\club}[1]{\operatorname{club}\left(#1\right)}
\newcommand{\Lev}[1]{\operatorname{Lev}\left(#1\right)}
\newcommand{\height}[1]{\operatorname{ht}\left(#1\right)}
\newcommand{\emptyseq}{\Lambda}

\newcommand{\N}{\mathbb N}
\newcommand{\Q}{\mathbb Q}
\newcommand{\R}{\mathbb R}
\newcommand{\C}{\mathbb C}
\newcommand{\F}{\mathbb F}




\usepackage{xparse}

\ExplSyntaxOn
\tl_new:N \l_septatrix_env_tl
\NewDocumentCommand \getenv { o m }
{
	\sys_get_shell:nnN { kpsewhich ~ --var-value ~ #2 }
	{ \int_set:Nn \tex_endlinechar:D { -1 } }
	\l_septatrix_env_tl
	\IfNoValueTF {#1}
	{ \tl_use:N \l_septatrix_env_tl }
	{ \tl_set_eq:NN #1 \l_septatrix_env_tl }
}
\tl_const:Nn \c_getenv_par_tl { \par }

\NewDocumentCommand{\ifenvsetTF}{mmm}
{
	\sys_get_shell:nnN { kpsewhich ~ --var-value ~ #1 } { } \l_tmpa_tl
	\tl_if_eq:NNTF \l_tmpa_tl \c_getenv_par_tl { #3 } { #2 }
}
\ExplSyntaxOff

\usepackage{datetime}
% \newdate{date}{06}{09}{2012}
% \date{\displaydate{date}}
\date{\today}
\newcommand{\envOrDefault}[2]{\ifenvsetTF{#1}{\getenv{#1}}{#2}}

\author{\envOrDefault{au4thor}{Holo}}



\usepackage{skak}
\usepackage{relsize}
\usepackage{graphicx}
\usepackage{mathtools}

\usepackage{textcomp}
\usepackage{bbding}

\usepackage{soul}

\newcommand{\flower}{\text{\scalebox{0.75}{\raisebox{-0.7ex}{
				\rotatebox{90}{\textleaf}\hspace{-0.3em}
				\scalebox{0.7}{\textleaf}\hspace{-1.35em}
				\raisebox{1ex}{\scalebox{0.8}{\FiveFlowerOpen}}
}}}}
\newcommand{\ei}[1]{e^{i#1}}


\title{Exercise 1}
\begin{document}
\maketitle
{\Large Complex Numbers}
\begin{cExercise}[][][author]
	We know that $(\rho \ei{\phi})^n=\rho^n \ei{n\phi}$, in particular we have $(\rho \ei{\phi})^n=r\ei{\theta}$ hence
	\begin{align*}
		&\rho^n=r&\implies&\rho=r^{1/n}\\
		&n\phi\equiv \theta\pmod {2\pi}&\implies& \phi=(\theta+2\pi k)/n,\ 0\le k<n
	\end{align*}
	(notice that the first row is legal because we know that $r,\rho$ both should be positive reals).
	
	In particular there are $n$ solutions.
\end{cExercise}

\begin{cExercise}[][][author][2]
	\begin{cPart}
		$1=1\ei{0}$, plugin in the solutions from (1) we get
		\begin{align*}
			&\rho=1\\
			&\phi=2\pi k/6,\ 0\le k<6
		\end{align*}		
	\end{cPart}
	\begin{cPart}
		$-1=1\ei{\pi}$, hence:
		\begin{align*}
			&\rho=1\\
			&\phi=(\pi+2\pi k)/4=3\pi k/4,\ 0\le k<4
		\end{align*}	
	\end{cPart}
	\begin{cPart}
		$-1+i\sqrt{3}=|-1+i\sqrt{3}|\ei{\arg(-1+i\sqrt{3})}=2\ei{2\pi/3}$ hence:
		\begin{align*}
			&\rho=2^{1/4}\\
			&\phi=(2\pi/3+2\pi k)/4,\ 0\le k<4
		\end{align*}
	\end{cPart}
\end{cExercise}
\begin{cExercise}[][][author]
	\begin{cPart}
		\begin{equation*}
			\frac1{6+2i}=\frac{\overline{6+2i}}{(6+2i)(\overline{6+2i})}=\frac{6}{36+4}+i\frac{-2}{36+4}=\frac{3}{20}+i\frac{-1}{20}
		\end{equation*}
	\end{cPart}
	\begin{cPart}
		\begin{equation*}
			\frac{(2+i)(3+2i)}{1-i}=\frac{(2+i)(3+2i)(\overline{1-i})}{(1-i)(\overline{1-i})}=\frac{-3+11i}{2}=-\frac32+i\frac{11}2
		\end{equation*}
	\end{cPart}
	\begin{cPart}
		\begin{equation*}
			-\frac12+i\frac{\sqrt{3}}2=1\ei{2\pi/3}\implies(-\frac12+i\frac{\sqrt{3}}2)^4=\ei{8\pi/3}=\ei{2\pi/3}=-\frac12+i\frac{\sqrt{3}}2
		\end{equation*}
	\end{cPart}
	\begin{cPart}
		\begin{equation*}
			-1+i0,\ 0+i(-1),\ 1+i0,\ 0+i1
		\end{equation*}
	\end{cPart}
\end{cExercise}
\begin{cExercise}[][][author][4]
	\begin{align*}
		&(a^2+b^2)(c^2+d^2)&=&a^2c^2+b^2d^2+a^2d^2+b^2c^2=a^2c^2-2abcd+b^2d^2+a^2d^2+2abcd+b^2c^2\\
		&&=&(ac-bd)^2+(ad+bc)^2
	\end{align*}
\end{cExercise}
{\Large Closed and Open Sets}
\begin{cExercise}[][][author][1]
	Let $\{\flower_i\}_{i\in I}$ be a family of open sets, and let $x\in \bigcup_{i\in I}\flower_i$.
	
	By definition there is some $i\in I$ such that $x\in \flower_i$, because $U_i$ is open there must be some $\epsilon>0$ such that $B_\epsilon(x)\subseteq \flower_i$, in particular $B_\epsilon(x)\subseteq \bigcup_{i\in I}\flower_i$, hence $\bigcup_{i\in I}\flower_i$ is open.
	
	Let $\{\flower_j\}_{j\in J}$ be a finite family of open sets, and let $x\in \bigcap_{j\in J}\flower_j$, by definition there exists $\epsilon_j$ for each $j\in J$ such that $B_{\epsilon_j}(x)\subseteq \flower_j$. Because $J$ is finite, the set $\{\epsilon_j\}$ has a minimum, let $\epsilon$ be this minimum and it is clear that $B_\epsilon(x)\subseteq \bigcap_{j\in J}\flower_j$
\end{cExercise}
\begin{cExercise}
	Assume the contrary, that $f^{-1}(U)$ is not open, in particular there exists $x\in f^{-1}(U)$ that witness it.
	
	Because $f$ is continuous we know that for every $\epsilon$ there is some $\delta$ such that $x'\in B_\delta(x)\implies f(x')\in B_\epsilon(f(x))$. Let $\epsilon$ be sufficiently small so that $B_\epsilon(f(x))\subseteq U$ (it exists because $U$ is open), and let $\delta$ be as above, but the above can be restated as $f''B_\delta(x)\subseteq B_\epsilon (f(x))$, in particular $B_\delta(x)\subseteq f^{-1}(U)$, but this contradict the fact that $x$ is witness of the failure of $f^{-1}(U)$ to be open.
\end{cExercise}
\begin{cExercise}
	Assume that $\lim x_n=x\notin C$, we must have then that $x\in \R^n\setminus C$, which is open.
	
	Let $\epsilon$ be such that $B_\epsilon(x)\subseteq \R^n\setminus C$, we know that $B_\epsilon(x)\cap C=\emptyset$, so for all $n\in\omega$ we have $x_n\notin B_\epsilon(x)$, which contradict the fact that they converge to $x$.
\end{cExercise}
\end{document}




