\documentclass[12pt,reqno]{article}
\usepackage[margin=3cm]{geometry}
\usepackage[utf8]{inputenc}


\usepackage{xfp}
\usepackage{natbib}
\usepackage{graphicx}
\usepackage{amsthm}
\usepackage{amsmath}
\usepackage{amssymb}
\usepackage{cases}
\usepackage{microtype}
\usepackage{hyperref}
\usepackage{mathrsfs}

\newtheorem{theorem}{Theorem}[section]
\newtheorem{proposition}[theorem]{Proposition}
\newtheorem{corollary}[theorem]{Corollary}
\newtheorem{lemma}[theorem]{Lemma}
\newtheorem{conjecture}[theorem]{Conjecture}
\newtheorem{example}[theorem]{Example}
\theoremstyle{definition}
\newtheorem{definition}[theorem]{Definition}
\theoremstyle{remark}
\newtheorem{remark}[theorem]{Remark}
\theoremstyle{exercise}
\newtheorem{exercise}[section]{Exercise}
\theoremstyle{subExercise}
\newtheorem{subExercise}{Part}[section]
\newtheorem{subSubExercise}{Sub Part}[subExercise]
\numberwithin{equation}{section}
\newcommand{\acl}[2]{\operatorname{acl}^{#2}\left(#1\right)}
\newcommand{\rank}[2]{\operatorname{rank}_{#2}\left(#1\right)}
\newcommand{\dcl}[2]{\operatorname{dcl}^{#2}\left(#1\right)}
\newcommand{\Aut}[1]{\operatorname{Aut}\left(#1\right)}
\newcommand{\Av}[1]{\operatorname{Av}\left(#1\right)}
\newcommand{\comment}[2]{#2}
\newcommand{\cof}[1]{\operatorname{cof}\left(#1\right)}
\renewcommand{\cal}[1]{\mathcal{#1}}
\newcommand{\D}[2][{}]{\text{Diag}_{#1}\left(#2\right)}
\renewcommand{\phi}{\varphi}
\newcommand{\code}[1]{\left\lceil#1\right\rceil}
\setcounter{MaxMatrixCols}{20}
\sloppy

\newcommand{\ex}[2][\fpeval{\value{section}+1}]{\setcounter{section}{\fpeval{#1 - 1}}
	\begin{exercise}
		#2
\end{exercise}}
\newcommand{\sub}[2][\fpeval{\value{subExercise}+1}]{\setcounter{subExercise}{\fpeval{#1 - 1}}
	\begin{subExercise}
		#2
\end{subExercise}}
\newcommand{\subb}[2][\fpeval{\value{subSubExercise}+1}]{\setcounter{subSubExercise}{\fpeval{#1 - 1}}
	\begin{subSubExercise}
		#2
\end{subSubExercise}}
\newcommand{\cl}{\mbox{cl}}
\newcommand{\CB}[1]{\operatorname{CB}\left(#1\right)}
\newcommand{\MR}[1]{\operatorname{MR}\left(#1\right)}
\newcommand{\MD}[1]{\operatorname{MD}\left(#1\right)}
\newcommand{\monster}{{\mathfrak C}}
\newcommand{\tp}[1]{\operatorname{tp}\left(#1\right)}
\newcommand{\hull}[3]{\operatorname{Hull}^{#2}_{#3}\left(#1\right)}
\newcommand{\stp}[1]{\operatorname{stp}\left(#1\right)}
\newcommand{\dom}[1]{\operatorname{dom}\left(#1\right)}
\newcommand{\range}[1]{\operatorname{range}\left(#1\right)}
\newcommand{\cnst}[2]{\operatorname{const}_{#1}\left(#2\right)}
\renewcommand{\c}{{\mathfrak c}}
\newcommand{\club}[1]{\operatorname{club}\left(#1\right)}
\newcommand{\Lev}[1]{\operatorname{Lev}\left(#1\right)}
\newcommand{\height}[1]{\operatorname{ht}\left(#1\right)}
\newcommand{\emptyseq}{\Lambda}

\newcommand{\N}{\mathbb N}
\newcommand{\Q}{\mathbb Q}
\newcommand{\R}{\mathbb R}
\newcommand{\C}{\mathbb C}
\newcommand{\F}{\mathbb F}




\usepackage{xparse}

\ExplSyntaxOn
\tl_new:N \l_septatrix_env_tl
\NewDocumentCommand \getenv { o m }
{
	\sys_get_shell:nnN { kpsewhich ~ --var-value ~ #2 }
	{ \int_set:Nn \tex_endlinechar:D { -1 } }
	\l_septatrix_env_tl
	\IfNoValueTF {#1}
	{ \tl_use:N \l_septatrix_env_tl }
	{ \tl_set_eq:NN #1 \l_septatrix_env_tl }
}
\tl_const:Nn \c_getenv_par_tl { \par }

\NewDocumentCommand{\ifenvsetTF}{mmm}
{
	\sys_get_shell:nnN { kpsewhich ~ --var-value ~ #1 } { } \l_tmpa_tl
	\tl_if_eq:NNTF \l_tmpa_tl \c_getenv_par_tl { #3 } { #2 }
}
\ExplSyntaxOff

\usepackage{datetime}
% \newdate{date}{06}{09}{2012}
% \date{\displaydate{date}}
\date{\today}
\newcommand{\envOrDefault}[2]{\ifenvsetTF{#1}{\getenv{#1}}{#2}}

\author{\envOrDefault{au4thor}{Holo}}



\usepackage{skak}
\usepackage{relsize}
\usepackage{graphicx}
\usepackage{mathtools}

\usepackage{textcomp}
\usepackage{bbding}

\usepackage{soul}

\newcommand{\flower}{\text{\scalebox{0.75}{\raisebox{-0.7ex}{
				\rotatebox{90}{\textleaf}\hspace{-0.3em}
				\scalebox{0.7}{\textleaf}\hspace{-1.35em}
				\raisebox{1ex}{\scalebox{0.8}{\FiveFlowerOpen}}
}}}}
\newcommand{\ei}[1]{e^{i#1}}


\title{Exercise 3}
\begin{document}
\maketitle
\begin{cExercise}[][][author]
	Assume that $\Omega$ is star domain at $\flower\in\mathbb C$, $T(z_1,z_2,z_3)^*\subseteq \Omega$, $z^\circ\in \Delta(z_1,z_2,z_3)\setminus T(z_1,z_2,z_3)^*$.
	
	Let $L$ be the line segment that starts at $\flower$ passes through $z^\circ$ (that is, continuing $I(\flower,z^\circ)$ in the direction of $z^\circ$).
	
	Because $z^\circ$ is inside of the circle, the line $L$ must intersect (uniquely) with $\partial \Delta(z_1,z_2,z_3)=T(z_1,z_2,z_3)^*$ \textit{after} passing through $z^\circ$ (and maybe one time before that), let $z^\bullet$ be that intersection.
	
	Because $\Omega$ is star domain at $\flower$ we have that $I(\flower, z^\bullet)^*\subseteq \Omega$, but of course we have that $z^\circ\in I(\flower, z^\bullet)^*$.
\end{cExercise}
\begin{cExercise}
	Because $\gamma$ is a finite sum of continuously differential curves, we can split the integral to finite sum of $f$ over continuously differential curves (connecting at the endpoints), so it is enough to prove the required for $\gamma$ a continuously differential curve.
	
	We know that $\int_\gamma f(z)dz=\int_0^1 f(\gamma(z))\gamma'(z)dz$.
	
	Because $\phi$ is strictly increasing, hence injective and we have that $\phi'$ is strictly positive, hence we can plug it in the formula for change of variables.
	$$...=\int_0^1f(\gamma(\phi(z)))\gamma'(\phi(z))\phi'(z)dz=\int_0^1f(\gamma\circ\phi(z))(\gamma\circ\phi)'(z)dz=\int_{\gamma\circ\phi}f(z)dz$$
\end{cExercise}
\begin{cExercise}
	We defined $L(\gamma)$ the supremum of the length of the possible polygonal chain approximation.
	
	For a given a polygonal chain characterized by the partition $P=[0=p_0,\ldots, p_n=1]$ we have $L(\gamma, P)=\sum_{i=0}^{n-1}|\gamma(p_{i+1})-\gamma(p_i)|\le \sum_{i=0}^{n-1}K|p_{i+1}-p_i|=K$, hence the supremum of all such approximations is $\le K$.
\end{cExercise}
\begin{cExercise}
	First we note that $\int_{T(x,y,w)}=\int_{3I(x,y)}+\int_{\frac13+3I(y,w)}+\int_{\frac23+3I(w,x)}$, where $s+v\times I(a,b)$ is $I(a,b)$ accelerated by $v$ starting at $s$.

	Hence $\int_{T(x,y,z)}\overline zdz=\int_{3I(x,y)}f(z)dz+\int_{\frac13+3I(y,w)}f(z)dz+\int_{\frac23+3I(w,x)}f(z)dz$ is
	\begin{align*}
		&\int_{3I(x,y)}f(z)dz&=&\int_0^{\frac13}f((3I(x,y))(z))(3I(x,y))'(z)dz\\
		&\int_{\frac13+\frac3I(y,w)}f(z)dz&=&\int_{\frac13}^{\frac23}f\left(\left(\frac13+3I(y,w)\right)(z)\right)\left(\frac13+3I(y,w)\right)'(z)dz\\
		&\int_{\frac23+\frac3I(w,x)}f(z)dz&=&\int_{\frac23}^{1}f\left(\left(\frac23+3I(w,x)\right)(z)\right)\left(\frac23+3I(w,x)\right)'(z)dz
	\end{align*}


	Let $x=0,y=a,z=ib, f(z)=\overline{z}$, and notice that $(I(0,a))'(t)=a, (I(a,ib))'(t)=ib-a, (I(ib, 0))'(t)=-ib$, and the rest is a simple polynomial integration to show that the integral is indeed $iab$.
\end{cExercise}
\begin{cExercise}
	\begin{cPart}
		We remember that $\frac{\partial f}{\partial x}=u_x+iv_x, \frac{\partial f}{\partial y}=u_y+iv_y$ so plugging this in the operator:
		\begin{align*}
			&\frac{\partial f}{\partial z}&=&\frac12\left(\frac{\partial f}{\partial x}-i\frac{\partial f}{\partial y}\right)&\\
			&&=&\frac12\left(u_x+iv_x-iu_y+v_y\right)&\\
			&&=&\frac12\left((u_x+v_y)+i(v_x-u_y)\right)&\\
			&&\overset{\text{CR}}{=}&\frac12\left((u_x+u_x)+i(v_x+v_x)\right)=u_x+iv_x&=f'(z)\\
			&\frac{\partial f}{\partial \overline z}&=&\frac12\left(\frac{\partial f}{\partial x}+i\frac{\partial f}{\partial y}\right)&\\
			&&=&\frac12\left(u_x+iv_x+iu_y-v_y\right)&\\
			&&=&\frac12\left((u_x-v_y)+i(v_x+u_y)\right)&\\
			&&\overset{\text{CR}}{=}&\frac12\left((u_x-u_x)+i(v_x-v_x)\right)&=0
		\end{align*}
	\end{cPart}
	\begin{cPart}
		Let $g=u'+iv'=u(x,-y)+iv(x,-y)$
		Calculating $\frac{\partial g}{\partial x}(x,y)=u_x'+v_x'=u_x(x,-y)+iv_x(x,-y)$ and $\frac{\partial g}{\partial y}(x,y)=u_y'+v_y'=-u_y(x,-y)-iv_x(x,-y)=-(u_y(x,-y)+iv_x(x,-y))$ immediately shows that the $2$ operators we define switch roles on $g$.
	\end{cPart}
\end{cExercise}
\begin{cExercise}
		\begin{cPart}
		Obviously $f'\equiv 0$ implies that $u'=v'\equiv 0$, from the result on $u,v$ as functions from $\mathbb R^2\to\mathbb R$ we get that $u,v$ are constants, but then $f=u+iv$ is also a constant.
	\end{cPart}
	\begin{cPart}
		If $f''\Omega\subseteq\mathbb R$, then in particular we have that $v\equiv 0$, hence $v_x=v_y\equiv 0$, but from Cauchy-Riemann we get that $u_x=v_y\equiv 0$ and that $u_y=-v_x\equiv 0$ which implies that $u'=v'\equiv 0$ and so $f'\equiv 0$.
	\end{cPart}
\end{cExercise}
\begin{cExercise}
	Let $\gamma_R:[0,1]\to\mathbb C:t\mapsto R\exp(i\pi\cdot t)$ and look at the integral $\int_{\gamma_R}f(z)dz=\int_0^1 f(R\exp(i\pi\cdot t))i\pi R\exp(i\pi\cdot t)dt$
	
	Setting $f(z)=\exp(iz)/z^2$ we get $$\int_{\gamma_R}\exp(iz)/z^2dz=\int_0^1 \frac{\exp(i(R\exp(i\pi\cdot t)))}{(R\exp(i\pi\cdot t))^2} i\pi R\exp(i\pi\cdot t)dt=\frac{i\pi}R\int_0^1 \frac{\exp(i(R\exp(i\pi\cdot t)))}{\exp(i\pi\cdot t)}dt$$
	
	Noticing that $|\exp(i(R\exp(i\pi\cdot t)))|$ is bounded by $M$ that is independent of $R$, so $\left|\int_{\gamma_R}\exp(iz)/z^2dz\right|\le \left|\frac{i\pi M}R\int_0^1 \frac{1}{\exp(i\pi\cdot t)}dt\right|$ which obviously goes to $0$ as $R$ grows.
\end{cExercise}
\end{document}




