\documentclass[12pt,reqno]{article}
\usepackage[margin=3cm]{geometry}
\usepackage[utf8]{inputenc}


\usepackage{xfp}
\usepackage{natbib}
\usepackage{graphicx}
\usepackage{amsthm}
\usepackage{amsmath}
\usepackage{amssymb}
\usepackage{cases}
\usepackage{microtype}
\usepackage{hyperref}
\usepackage{mathrsfs}

\newtheorem{theorem}{Theorem}[section]
\newtheorem{proposition}[theorem]{Proposition}
\newtheorem{corollary}[theorem]{Corollary}
\newtheorem{lemma}[theorem]{Lemma}
\newtheorem{conjecture}[theorem]{Conjecture}
\newtheorem{example}[theorem]{Example}
\theoremstyle{definition}
\newtheorem{definition}[theorem]{Definition}
\theoremstyle{remark}
\newtheorem{remark}[theorem]{Remark}
\theoremstyle{exercise}
\newtheorem{exercise}[section]{Exercise}
\theoremstyle{subExercise}
\newtheorem{subExercise}{Part}[section]
\newtheorem{subSubExercise}{Sub Part}[subExercise]
\numberwithin{equation}{section}
\newcommand{\acl}[2]{\operatorname{acl}^{#2}\left(#1\right)}
\newcommand{\rank}[2]{\operatorname{rank}_{#2}\left(#1\right)}
\newcommand{\dcl}[2]{\operatorname{dcl}^{#2}\left(#1\right)}
\newcommand{\Aut}[1]{\operatorname{Aut}\left(#1\right)}
\newcommand{\Av}[1]{\operatorname{Av}\left(#1\right)}
\newcommand{\comment}[2]{#2}
\newcommand{\cof}[1]{\operatorname{cof}\left(#1\right)}
\renewcommand{\cal}[1]{\mathcal{#1}}
\newcommand{\D}[2][{}]{\text{Diag}_{#1}\left(#2\right)}
\renewcommand{\phi}{\varphi}
\newcommand{\code}[1]{\left\lceil#1\right\rceil}
\setcounter{MaxMatrixCols}{20}
\sloppy

\newcommand{\ex}[2][\fpeval{\value{section}+1}]{\setcounter{section}{\fpeval{#1 - 1}}
	\begin{exercise}
		#2
\end{exercise}}
\newcommand{\sub}[2][\fpeval{\value{subExercise}+1}]{\setcounter{subExercise}{\fpeval{#1 - 1}}
	\begin{subExercise}
		#2
\end{subExercise}}
\newcommand{\subb}[2][\fpeval{\value{subSubExercise}+1}]{\setcounter{subSubExercise}{\fpeval{#1 - 1}}
	\begin{subSubExercise}
		#2
\end{subSubExercise}}
\newcommand{\cl}{\mbox{cl}}
\newcommand{\CB}[1]{\operatorname{CB}\left(#1\right)}
\newcommand{\MR}[1]{\operatorname{MR}\left(#1\right)}
\newcommand{\MD}[1]{\operatorname{MD}\left(#1\right)}
\newcommand{\monster}{{\mathfrak C}}
\newcommand{\tp}[1]{\operatorname{tp}\left(#1\right)}
\newcommand{\hull}[3]{\operatorname{Hull}^{#2}_{#3}\left(#1\right)}
\newcommand{\stp}[1]{\operatorname{stp}\left(#1\right)}
\newcommand{\dom}[1]{\operatorname{dom}\left(#1\right)}
\newcommand{\range}[1]{\operatorname{range}\left(#1\right)}
\newcommand{\cnst}[2]{\operatorname{const}_{#1}\left(#2\right)}
\renewcommand{\c}{{\mathfrak c}}
\newcommand{\club}[1]{\operatorname{club}\left(#1\right)}
\newcommand{\Lev}[1]{\operatorname{Lev}\left(#1\right)}
\newcommand{\height}[1]{\operatorname{ht}\left(#1\right)}
\newcommand{\emptyseq}{\Lambda}

\newcommand{\N}{\mathbb N}
\newcommand{\Q}{\mathbb Q}
\newcommand{\R}{\mathbb R}
\newcommand{\C}{\mathbb C}
\newcommand{\F}{\mathbb F}




\usepackage{xparse}

\ExplSyntaxOn
\tl_new:N \l_septatrix_env_tl
\NewDocumentCommand \getenv { o m }
{
	\sys_get_shell:nnN { kpsewhich ~ --var-value ~ #2 }
	{ \int_set:Nn \tex_endlinechar:D { -1 } }
	\l_septatrix_env_tl
	\IfNoValueTF {#1}
	{ \tl_use:N \l_septatrix_env_tl }
	{ \tl_set_eq:NN #1 \l_septatrix_env_tl }
}
\tl_const:Nn \c_getenv_par_tl { \par }

\NewDocumentCommand{\ifenvsetTF}{mmm}
{
	\sys_get_shell:nnN { kpsewhich ~ --var-value ~ #1 } { } \l_tmpa_tl
	\tl_if_eq:NNTF \l_tmpa_tl \c_getenv_par_tl { #3 } { #2 }
}
\ExplSyntaxOff

\usepackage{datetime}
% \newdate{date}{06}{09}{2012}
% \date{\displaydate{date}}
\date{\today}
\newcommand{\envOrDefault}[2]{\ifenvsetTF{#1}{\getenv{#1}}{#2}}

\author{\envOrDefault{au4thor}{Holo}}



\usepackage{skak}
\usepackage{relsize}
\usepackage{graphicx}
\usepackage{mathtools}

\usepackage{textcomp}
\usepackage{bbding}

\usepackage{soul}

\newcommand{\flower}{\text{\scalebox{0.75}{\raisebox{-0.7ex}{
				\rotatebox{90}{\textleaf}\hspace{-0.3em}
				\scalebox{0.7}{\textleaf}\hspace{-1.35em}
				\raisebox{1ex}{\scalebox{0.8}{\FiveFlowerOpen}}
}}}}
\newcommand{\ei}[1]{e^{i#1}}


\title{Exercise 4}
\begin{document}
\maketitle
\begin{cExercise}[][][author]
	\begin{cPart}
		Let $p$ be a polynomial of degree $k$ and large coefficient of $M$.
		
		We know that the radius of convergence is $\limsup\frac1{\sqrt[n]{|p(n)a_n|}}$ and we want to show it is equal to $\limsup\frac1{\sqrt[n]{|a_n|}}$, for that we will show that $\limsup \sqrt[n]{|p(n)a_n|}=\limsup \sqrt[n]{|a_n|}$.
		
		Indeed $\limsup \sqrt[n]{|p(n)a_n|}=\limsup \sqrt[n]{|p(n)||a_n|}=\limsup \sqrt[n]{|p(n)|}\sqrt[n]{|a_n|}$, but $\limsup \sqrt[n]{|p(n)|}=\limsup \sqrt[n]{|M|}\sqrt[n]{|n^k|}=\limsup \sqrt[n]{|M|}\limsup \sqrt[n]{|n^k|}=\limsup \sqrt[n]{|n|}^k=(\limsup \sqrt[n]{|n|})^k=1^k=1.$
	\end{cPart}
	\begin{cPart}
		We know that $|\frac{a_n}{\sqrt{n!}}|=\frac{|a_n|}{\sqrt{n!}}$.
		
		Furthermore, we have that the limsup of $\sqrt[n]{|a_n|}$ is bounded (as the series has positive convergence radius) and (because we know that $\exp(z)$ has a radius of convergence $\infty$) that $\sqrt[n]{\frac1{\sqrt{n!}}}$ goes to $0$, so the convergence radius of the new series is infinity.
		
		Let $t\in\omega$ be an index such that $\sup_{p>t}\sqrt[p]{|a_p|}<\infty$, the define the sequence $b_k=\sup_{p>t+k}\sqrt[p]{|a_p|}$.
		
		Clearly $b_k\underset{k\to\infty}{\longrightarrow}\frac1r$ and furthermore we have that $b_k^2=\sup_{p>t+k}\sqrt[p]{|a_p|}\sup_{p>t+k}\sqrt[p]{|a_p|}=\sup_{p>t+k}\sqrt[p]{|a_p|}\sqrt[p]{|a_p|}=\sup_{p>t+k}\sqrt[p]{|a_p|^2}$, so $\limsup{\sqrt[n]{|a_n|^2}}=\lim b_k^2$. From arithmetic of limits we have $b_k^2\underset{k\to\infty}{\longrightarrow}\frac1{r^2}$, so the convergence radius is $r^2$.
	\end{cPart}
\end{cExercise}
\begin{cExercise}
	\begin{cPart}
		Let's $F$ be as in the question, and calculate $F'(z)=\lim\limits_{z_1\to z}\frac{F(z_1)-F(z)}{z_1-z}=\lim\limits_{z_1\to z}\frac{\int_{C(z_0,r)}\frac{1}{w-z_1}-\frac{1}{w-z}dw}{z_1-z}$, looking at $\frac{1}{w-z_1}-\frac{1}{w-z}$ we have $\frac{1}{w-z_1}-\frac{1}{w-z}=\frac{(w-z)-(w-z_1)}{(w-z_1)(w-z)}=\frac{z_1-z}{(w-z_1)(w-z)}$ so:
		
		\begin{align*}
			&F'(z)&=&\lim\limits_{z_1\to z}\frac{\int_{C(z_0,r)}\frac{1}{w-z_1}-\frac{1}{w-z}dw}{z_1-z}&=&\lim\limits_{z_1\to z}\frac{\int_{C(z_0,r)}\frac{z_1-z}{(w-z_1)(w-z)}dw}{z_1-z}\\
			&&=&\lim\limits_{z_1\to z}\int_{C(z_0,r)}\frac{1}{(w-z_1)(w-z)}dw&=&\int_{C(z_0,r)}\frac{1}{(w-z)^2}dw
		\end{align*}
		
		And it is clear that $g(z,w)=\frac1{(w-z)^2}$ satisfy the conditions we want.
	\end{cPart}
	\begin{cPart}
		For a fixed $z$ we have that $D_w =-\frac1{w-z}=f(w,z)$ at $\C\setminus\{z\}$, so let $G(w,z)$ is an anti-derivative of $g$ for a fixed $z$. 
		
		Because $\C\setminus\{z\}$ is open and $g$ has a global anti-derivative in there, we know that the integral over any closed loop contained in $\C\setminus\{z\}$ is $0$, hence $F'(z)$, which is defined as such integral, is $0$ 
	\end{cPart}
	\begin{cPart}
		$F(z_0)=\int_{C(z_0,r)}\frac1{w-z_0}dw=\int_0^{2\pi}\frac{D_t(z_0+re^{it})}{(z_0+re^{it})-z_0}dt=\int_0^{2\pi}\frac{ire^{it}}{re^{it}}dt=i\int_0^{2\pi}dt=2\pi i$.
		
		Because $F'\equiv 0$, we know that $F$ is a constant on any connected component of it's domain, in particular $F\equiv F(z_0)$ in $B_r(z_0)$.
	\end{cPart}
\end{cExercise}
\begin{cExercise}
	We know that $\exp(z)=\sum_{n=0}^{\infty} \frac{\exp^{(n)}(z_0)}{n!}(z-z_0)^n=\overset{\text{Taylor Series}}{\sum_{n=0}^{\infty} \frac{\exp(z_0)}{n!}(z-z_0)^n}=\exp(z_0)\frac{\exp(z_0)}{n!}(z-z_0)^n=\exp(z_0)\exp(z-z_0)$.
	
	So given $w,w'\in \C$ and letting $z=w+w'$ and $z_0=w'$ we get $\exp(w+w')=\exp(z)=\exp(z_0)\exp(z-z_0)=\exp(w')\exp(w+w'-w')=\exp(w')\exp(w)$
\end{cExercise}
\begin{cExercise}
	\begin{cPart}
		The function is not defined precisely for $z\in\C$ such that $i(\exp(iz)+\exp(-iz))= 0\iff \exp(iz)+\exp(-iz)= 0\iff \exp(iz)= -\exp(-iz)$.
		
		Let $z=x+i y$: $\exp(iz)= -\exp(-iz)\iff \overset{\text{LHS}}{\exp(-y)\exp(ix)}=\overset{\text{RHS}}{-\exp(y)\exp(-ix)}$.
		
		If $\exp(-y)\ne \exp(y)$ then the absolute value of the LHS and RHS will be different, hence they will be different, but $\exp(-y)=\exp(y)\iff y=0$ for a real $y$, the function is not defined on $z$ iff $z\in \R$ and $\exp(iz)=-\exp(-iz)$.
		
		We know that $\exp(ix)=\cos(x)+i\sin(x)$, $-\exp(-ix)=-(\cos(-x)+i\sin(-x))=-(\cos(x)-i\sin(x))=i\sin(x)-\cos(x)$, comparing the $2$ sides gives us that $\tan$ is not defined precisely on $z\in\R$ such that $\cos(z)=-\cos(z)\iff \cos(z)=0\iff z=\frac\pi2+n\pi$ for some integer $n$.
	\end{cPart}
	\begin{cPart}
		Let $z\in\dom{\tan}$, and let $r>r'\in\R_{>0}$ such that $B_r(z)\subseteq \dom{\tan}$, on $B_{r'}(z)$, we have that $\tan$ is analytic, so it's Taylor series around $z$ has convergence radius $\ge r'$, so we have that the radius of convergence is $\sup\{r'\mid \exists r>r'\ s.t.\ B_r(z)\subseteq \dom{\tan}\}$, but this is exactly the minimum distance between $z$ and $\{\frac\pi2+n\pi\mid n\in\Z\}$ (to find the particular $n$ for a given $z$, we can reduce the problem to minimize the distance of $\Re(z)$ from that set)
	\end{cPart}
\end{cExercise}
\begin{cExercise}
	We know that $f$ is analytic over $\C$, so the integral $\int_{\gamma_R}f=0$ for all $R$, this integral is also equal to the sum of the integrals over the sides of the rectangle.
	
	Let's calculate $\lim_{R\to\infty}\int_{I(R,R+i\xi)}f$, we have that $|\int_{I(R,R+i\xi)}f|\le \int_{I(R,R+i\xi)}|f|\le\int_{I(R,R+i\xi)}\sup_{z\in {I(R,R+i\xi)}^*}|f|=|\xi| \sup_{z\in {I(R,R+i\xi)}^*}|f|$.
	
	Now for $z\in{I(R,R+i\xi)}^*$ we have $z=R+i\eta$ for $\eta\in[0,\xi]$, so $|f(z)|=|f(R+i\eta)|=|\exp(-(R+i\eta)^2/2)|=|\exp(-(R^2+2iR\eta-\eta^2)/2)|=|\exp((\eta^2-R^2)/2)\exp(iR\eta)|=|\exp((\eta^2-R^2)/2)|\le|\exp((\xi^2-R^2)/2)|\longrightarrow0$, so the integral goes to $0$.
	
	Similar calculations happens for the integral on $I(-R+i\xi, -R)$.
	
	So $0=\lim\limits_{R\to\infty}\int_{\gamma_R}f=\lim\limits_{R\to\infty}\int_{I(-R, R)}f+\lim\limits_{R\to\infty}\int_{I(R+i\xi, -R+i\xi)}f$, hence $-\sqrt{2\pi}=-\lim\limits_{R\to\infty}\int_{I(-R, R)}f=\lim\limits_{R\to\infty}\int_{I(R+i\xi, -R+i\xi)}f$
	
	Now we shall write $\int_{I(R+i\xi, -R+i\xi)}f$ explicitly: 
	
	$I(R+i\xi, -R+i\xi)(t)=\nu(t)=i\xi+R-tR$, $\nu'(t)=-R$, so $\int_{I(R+i\xi, -R+i\xi)}f=\int_{0}^2 f(\nu(t))\nu'(t)dt=\int_{0}^2-R\exp(-(i\xi+R-tR)^2/2)dt$
	
	Let $u=R-Rt$ and we get 
	\begin{align*}
		&-\sqrt{2\pi}&=&\lim_{R\to\infty}\int_\nu f\\
		&&=&\lim_{R\to\infty}\int_{0}^2-R\exp(-(i\xi+R-tR)^2/2)dt\\
		&&=&\int_{\infty}^{-\infty}\exp(-(i\xi+u)^2/2)du\\
		&&=&\int_{\infty}^{-\infty}\exp((\xi^2-2i\xi u-u^2)/2)du\\
		&&=&\int_{\infty}^{-\infty}\exp(\xi^2/2)\exp(-i\xi u)\exp((-u^2)/2)du\\
		&&=&-\exp(\xi^2/2)\int_{-\infty}^\infty\exp(i\xi u)\exp((-(-u)^2)/2)du\\
		&&=&-\exp(\xi^2/2)\int_{-\infty}^\infty\exp(i\xi u)f(u)du\\
		&&=&-\exp(\xi^2/2)\hat{f}(\xi)\\
		&&=&-\frac{\hat{f}(\xi)}{f(\xi)}
	\end{align*}
	Hence we have $\hat{f}(\xi)=\sqrt{2\pi}f(\xi)$
\end{cExercise}
\begin{cExercise}
	\begin{cPart}
		$f$ is analytic on the disk $\overline{B_1(0)}$ means by definition that there exists an open set $B\supseteq \overline{B_1(0)}$ on which $f$ is analytic.
		
		So by Cauchy formula we have $f'(0)=\frac1{2\pi i}\int_{C(0,1)}\frac{f(z)}{z^2}dz$.
		
		With $C(0,1)(t)=\exp(it)$, now we can define $R(0,1)=-\exp(it)=C(0,1)(t+\pi)$. The integral over $C(0,1)$ and $R(0,1)$ are equal, as they are just moving the starting point, so $f'(0)=\frac1{2\pi i}\int_{R(0,1)}\frac{f(z)}{z^2}dz=\int_0^{2\pi}\frac{f(-\exp(it))}{(-\exp(it))^2}\cdot(-i\exp(it))dt=\frac1{2\pi i}\int_{C(0,1)}\frac{-f(-z)}{(-z)^2}dz$
	\end{cPart}
\end{cExercise}
\end{document}




