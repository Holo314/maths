\documentclass[12pt,reqno]{article}
\usepackage[margin=3cm]{geometry}
\usepackage[utf8]{inputenc}


\usepackage{xfp}
\usepackage{natbib}
\usepackage{graphicx}
\usepackage{amsthm}
\usepackage{amsmath}
\usepackage{amssymb}
\usepackage{cases}
\usepackage{microtype}
\usepackage{hyperref}
\usepackage{mathrsfs}

\newtheorem{theorem}{Theorem}[section]
\newtheorem{proposition}[theorem]{Proposition}
\newtheorem{corollary}[theorem]{Corollary}
\newtheorem{lemma}[theorem]{Lemma}
\newtheorem{conjecture}[theorem]{Conjecture}
\newtheorem{example}[theorem]{Example}
\theoremstyle{definition}
\newtheorem{definition}[theorem]{Definition}
\theoremstyle{remark}
\newtheorem{remark}[theorem]{Remark}
\theoremstyle{exercise}
\newtheorem{exercise}[section]{Exercise}
\theoremstyle{subExercise}
\newtheorem{subExercise}{Part}[section]
\newtheorem{subSubExercise}{Sub Part}[subExercise]
\numberwithin{equation}{section}
\newcommand{\acl}[2]{\operatorname{acl}^{#2}\left(#1\right)}
\newcommand{\rank}[2]{\operatorname{rank}_{#2}\left(#1\right)}
\newcommand{\dcl}[2]{\operatorname{dcl}^{#2}\left(#1\right)}
\newcommand{\Aut}[1]{\operatorname{Aut}\left(#1\right)}
\newcommand{\Av}[1]{\operatorname{Av}\left(#1\right)}
\newcommand{\comment}[2]{#2}
\newcommand{\cof}[1]{\operatorname{cof}\left(#1\right)}
\renewcommand{\cal}[1]{\mathcal{#1}}
\newcommand{\D}[2][{}]{\text{Diag}_{#1}\left(#2\right)}
\renewcommand{\phi}{\varphi}
\newcommand{\code}[1]{\left\lceil#1\right\rceil}
\setcounter{MaxMatrixCols}{20}
\sloppy

\newcommand{\ex}[2][\fpeval{\value{section}+1}]{\setcounter{section}{\fpeval{#1 - 1}}
	\begin{exercise}
		#2
\end{exercise}}
\newcommand{\sub}[2][\fpeval{\value{subExercise}+1}]{\setcounter{subExercise}{\fpeval{#1 - 1}}
	\begin{subExercise}
		#2
\end{subExercise}}
\newcommand{\subb}[2][\fpeval{\value{subSubExercise}+1}]{\setcounter{subSubExercise}{\fpeval{#1 - 1}}
	\begin{subSubExercise}
		#2
\end{subSubExercise}}
\newcommand{\cl}{\mbox{cl}}
\newcommand{\CB}[1]{\operatorname{CB}\left(#1\right)}
\newcommand{\MR}[1]{\operatorname{MR}\left(#1\right)}
\newcommand{\MD}[1]{\operatorname{MD}\left(#1\right)}
\newcommand{\monster}{{\mathfrak C}}
\newcommand{\tp}[1]{\operatorname{tp}\left(#1\right)}
\newcommand{\hull}[3]{\operatorname{Hull}^{#2}_{#3}\left(#1\right)}
\newcommand{\stp}[1]{\operatorname{stp}\left(#1\right)}
\newcommand{\dom}[1]{\operatorname{dom}\left(#1\right)}
\newcommand{\range}[1]{\operatorname{range}\left(#1\right)}
\newcommand{\cnst}[2]{\operatorname{const}_{#1}\left(#2\right)}
\renewcommand{\c}{{\mathfrak c}}
\newcommand{\club}[1]{\operatorname{club}\left(#1\right)}
\newcommand{\Lev}[1]{\operatorname{Lev}\left(#1\right)}
\newcommand{\height}[1]{\operatorname{ht}\left(#1\right)}
\newcommand{\emptyseq}{\Lambda}

\newcommand{\N}{\mathbb N}
\newcommand{\Q}{\mathbb Q}
\newcommand{\R}{\mathbb R}
\newcommand{\C}{\mathbb C}
\newcommand{\F}{\mathbb F}




\usepackage{xparse}

\ExplSyntaxOn
\tl_new:N \l_septatrix_env_tl
\NewDocumentCommand \getenv { o m }
{
	\sys_get_shell:nnN { kpsewhich ~ --var-value ~ #2 }
	{ \int_set:Nn \tex_endlinechar:D { -1 } }
	\l_septatrix_env_tl
	\IfNoValueTF {#1}
	{ \tl_use:N \l_septatrix_env_tl }
	{ \tl_set_eq:NN #1 \l_septatrix_env_tl }
}
\tl_const:Nn \c_getenv_par_tl { \par }

\NewDocumentCommand{\ifenvsetTF}{mmm}
{
	\sys_get_shell:nnN { kpsewhich ~ --var-value ~ #1 } { } \l_tmpa_tl
	\tl_if_eq:NNTF \l_tmpa_tl \c_getenv_par_tl { #3 } { #2 }
}
\ExplSyntaxOff

\usepackage{datetime}
% \newdate{date}{06}{09}{2012}
% \date{\displaydate{date}}
\date{\today}
\newcommand{\envOrDefault}[2]{\ifenvsetTF{#1}{\getenv{#1}}{#2}}

\author{\envOrDefault{au4thor}{Holo}}



\usepackage{skak}
\usepackage{relsize}
\usepackage{graphicx}
\usepackage{mathtools}

\usepackage{textcomp}
\usepackage{bbding}

\usepackage{soul}

\newcommand{\flower}{\text{\scalebox{0.75}{\raisebox{-0.7ex}{
				\rotatebox{90}{\textleaf}\hspace{-0.3em}
				\scalebox{0.7}{\textleaf}\hspace{-1.35em}
				\raisebox{1ex}{\scalebox{0.8}{\FiveFlowerOpen}}
}}}}
\newcommand{\ei}[1]{e^{i#1}}


\title{Exercise 5}
\begin{document}
\maketitle
\begin{cExercise}[][][author]
	Notice that $-\cos(x)=\sin^{(3)}(x)$, in particular, from Cauchy formula, we have:
	\begin{equation*}
		-1=-\cos(0)=\sin^{(3)}(x)=\frac{3!}{2\pi i}\int_{C(0,r)}\frac{\sin(\xi)}{(\xi-0)^{3+1}}d\xi
	\end{equation*}
	Solving the equation we get that the integral is equal to $-\frac{\pi i}3$
\end{cExercise}
\begin{cExercise}
	Because $f$ is entire it satisfy Cauchy formula for any $z\in\C, r\in\R^+,n\in\N$:
	\begin{equation*}
		f^{(n)}(z)=\frac{n!}{2\pi i}\int_{C(z,r)}\frac{f(\xi)}{(\xi-z)^{n+1}}d\xi
	\end{equation*}
	Using our favourite inequality we get:
	\begin{align*}
		&|f^{(\lfloor d\rfloor)}(z)|&\le& \frac{\lfloor d\rfloor!}{2\pi }\int_{C(z,r)}\left|\frac{f(\xi)}{(\xi-z)^{\lfloor d\rfloor+1}}\right|d\xi&\\
		&&\le& \frac{|C|\lfloor d\rfloor!}{2\pi }\int_{C(z,r)}\frac{\left|z\right|^{\lfloor d\rfloor}+1}{r^{\lfloor d\rfloor+1}}d\xi&\\
		&&=&\frac{|C|\lfloor d\rfloor!}{2\pi r^{\lfloor d\rfloor+1}}\int_{C(z,r)}\left|z\right|^{\lfloor d\rfloor}+1d\xi&\\
		&&=&\frac{|C|\lfloor d\rfloor!}{ r^{\lfloor d\rfloor+1}}(\left|r+z\right|^{\lfloor d\rfloor}+1)&\overset{r\to\infty}{\longrightarrow}0&
	\end{align*}

	In particular $f^{(\lfloor d\rfloor)}\equiv 0$, taking the anti-derivative $\lfloor d\rfloor$ times gives the desired result
\end{cExercise}
\begin{cExercise}[][][][4]
	Let $f$ be an entire function satisfying $f(z)=f(z+i)=f(z+1)$, then $f''\{a+ib\mid a,b\in[0,1]\}=f''\C$, that is because given $z=x+iy$ then $f(x+iy)=f((x-\lfloor x\rfloor) + iy)=f((x-\lfloor x\rfloor) + i(y- \lfloor y\rfloor))$.
	
	But $f''\{a+ib\mid a,b\in[0,1]\}$ is bounded, hence $f$ is bounded, hence constant.
\end{cExercise}
\begin{cExercise}
	Assume both $f,g$ are not the identity $0$ and $\Omega$ connected.
	
	Let $x_0\in \Omega$ be such that $f(x_0)\ne 0$, in particular (as $f$ is continuous) we have that there exists $\epsilon>0$ such that $f(z)\ne 0$ on $B_{\epsilon}(x_0)$.
	
	By the assumption we have that on that ball we have $f(z)g(z)=0\implies g(z)=0$, but if $g$ is constant on an open set in a connected set, it is constant on the whole set, hence $g$ is constant everywhere (and because it is continuous, it is constant $0$).
	
	To see that the connected assumption is necessary, let $\Omega_0,\Omega_1$ be 2 disjoint open sets, and let:
	\begin{align*}
		f(z)=\begin{cases}
			0,&z\in\Omega_0\\
			1,&z\in\Omega_1
		\end{cases}\\
		g(z)=\begin{cases}
			1,&z\in\Omega_0\\
			0,&z\in\Omega_1
		\end{cases}
	\end{align*}
\end{cExercise}
\begin{cExercise}
	Let's rearrange our equation to get $(f'g-g'f)(1/n)=0$..
	
	Now assume that $h=f'g-g'f\not\equiv 0$ and let $k\in\N$ be the first such that $h^{(k)}(0)\ne 0$, it exists as $h$ analytic.
	
	Because $h$ is analytic, it equals to it's own Taylor series, in which the first $k$ terms disappear, so $h(z)=z^kp(z)$ for analytic $p$ with $p(0)\ne 0$.
	
	But because $p$ is continuous, it has a neighborhood $\Omega$ in around $0$ in which it is never $0$ in there, in particular $h(z)\ne 0$ for all $z\in\Omega\setminus\{0\}$, which is impossible as there exists a natural $n$ such that $\frac1n\in\Omega$.
	
	This means that $h\equiv 0$, which implies that $\left(\frac fg\right)'=0$, hence $\frac fg$ is constant.
\end{cExercise}
\begin{cExercise}
	Let $f,g$ be as in the question.
	
	We already saw that $g$ is analytic, it exhibit a local maxima on compact $\Omega\subseteq B_1(0)$ only on $\partial \Omega$.
	
	For $r\in(0,1)$, let $\Omega_r=\overline{B_r(0)}$, then $\max_{\Omega_r}(|g(z)|)=\max_{\partial\Omega_r}(|g(z)|)=\max_{|z|=r}(|g(z)|)=\max_{|z|=r}\left(\frac{|f(z)|}{|z|}\right)\le\frac1{r}$.
	
	Let $r\to 1$, and we get that $|g(z)|\le 1$ on the whole unit ball, and we are done.
\end{cExercise}
\begin{cExercise}
	We have that:
	\begin{align*}
		&\int_0^{2\pi}\frac1{a+b\cos(\theta)}d\theta&=&\int_0^{2\pi}\frac1{a+b\frac{\exp(iz)+\exp(-iz)}{2}}dz&\\
		&&=&2\int_0^{2\pi}\frac1{2a+b(\exp(iz)+\exp(-iz))}dz&\\
		&&=&2\int_0^{2\pi}\frac{\exp(iz)}{2a\exp(iz)+b\exp(2iz)+b}d\xi&\\
		&&=&\frac2i\int_{C(0,1)}\frac{1}{b\xi^2+2a\xi+b}d\xi&\\
		&&=&\frac2i\int_{C(0,1)}\frac{1}{(\xi-\xi_0)(\xi-\xi_1)}d\xi&\\
		&&&\text{Where $\xi_{0}=\frac{-a+\sqrt{a^2-b^2}}{b}, \xi_{1}=\frac{-a-\sqrt{a^2-b^2}}{b}$}&\\
		&&=&\frac2i\int_{C(0,1)}\frac{b}{2\sqrt{a^2-b^2}}\frac{1}{\xi-\xi_0}-\frac{b}{2\sqrt{a^2-b^2}}\frac{1}{\xi-\xi_1}d\xi&\\
		&&=&\frac{b}{\sqrt{a^2-b^2}i}\int_{C(0,1)}\frac{1}{\xi-\xi_0}-\frac{1}{\xi-\xi_1}d\xi&
	\end{align*}
	Only $x_0$ is in the unit circle, so the integral of the $x_1$ component is zero and the final answer is $\frac{2\pi b}{\sqrt{a^2-b^2}}$.
\end{cExercise}

\end{document}




