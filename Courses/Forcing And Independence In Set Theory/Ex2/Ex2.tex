\documentclass[12pt,reqno]{article}
\usepackage[margin=3cm]{geometry}
\usepackage[utf8]{inputenc}


\usepackage{xfp}
\usepackage{natbib}
\usepackage{graphicx}
\usepackage{amsthm}
\usepackage{amsmath}
\usepackage{amssymb}
\usepackage{cases}
\usepackage{microtype}
\usepackage{hyperref}
\usepackage{mathrsfs}

\newtheorem{theorem}{Theorem}[section]
\newtheorem{proposition}[theorem]{Proposition}
\newtheorem{corollary}[theorem]{Corollary}
\newtheorem{lemma}[theorem]{Lemma}
\newtheorem{conjecture}[theorem]{Conjecture}
\newtheorem{example}[theorem]{Example}
\theoremstyle{definition}
\newtheorem{definition}[theorem]{Definition}
\theoremstyle{remark}
\newtheorem{remark}[theorem]{Remark}
\theoremstyle{exercise}
\newtheorem{exercise}[section]{Exercise}
\theoremstyle{subExercise}
\newtheorem{subExercise}{Part}[section]
\newtheorem{subSubExercise}{Sub Part}[subExercise]
\numberwithin{equation}{section}
\newcommand{\acl}[2]{\operatorname{acl}^{#2}\left(#1\right)}
\newcommand{\rank}[2]{\operatorname{rank}_{#2}\left(#1\right)}
\newcommand{\dcl}[2]{\operatorname{dcl}^{#2}\left(#1\right)}
\newcommand{\Aut}[1]{\operatorname{Aut}\left(#1\right)}
\newcommand{\Av}[1]{\operatorname{Av}\left(#1\right)}
\newcommand{\comment}[2]{#2}
\newcommand{\cof}[1]{\operatorname{cof}\left(#1\right)}
\renewcommand{\cal}[1]{\mathcal{#1}}
\newcommand{\D}[2][{}]{\text{Diag}_{#1}\left(#2\right)}
\renewcommand{\phi}{\varphi}
\newcommand{\code}[1]{\left\lceil#1\right\rceil}
\setcounter{MaxMatrixCols}{20}
\sloppy

\newcommand{\ex}[2][\fpeval{\value{section}+1}]{\setcounter{section}{\fpeval{#1 - 1}}
	\begin{exercise}
		#2
\end{exercise}}
\newcommand{\sub}[2][\fpeval{\value{subExercise}+1}]{\setcounter{subExercise}{\fpeval{#1 - 1}}
	\begin{subExercise}
		#2
\end{subExercise}}
\newcommand{\subb}[2][\fpeval{\value{subSubExercise}+1}]{\setcounter{subSubExercise}{\fpeval{#1 - 1}}
	\begin{subSubExercise}
		#2
\end{subSubExercise}}
\newcommand{\cl}{\mbox{cl}}
\newcommand{\CB}[1]{\operatorname{CB}\left(#1\right)}
\newcommand{\MR}[1]{\operatorname{MR}\left(#1\right)}
\newcommand{\MD}[1]{\operatorname{MD}\left(#1\right)}
\newcommand{\monster}{{\mathfrak C}}
\newcommand{\tp}[1]{\operatorname{tp}\left(#1\right)}
\newcommand{\hull}[3]{\operatorname{Hull}^{#2}_{#3}\left(#1\right)}
\newcommand{\stp}[1]{\operatorname{stp}\left(#1\right)}
\newcommand{\dom}[1]{\operatorname{dom}\left(#1\right)}
\newcommand{\range}[1]{\operatorname{range}\left(#1\right)}
\newcommand{\cnst}[2]{\operatorname{const}_{#1}\left(#2\right)}
\renewcommand{\c}{{\mathfrak c}}
\newcommand{\club}[1]{\operatorname{club}\left(#1\right)}
\newcommand{\Lev}[1]{\operatorname{Lev}\left(#1\right)}
\newcommand{\height}[1]{\operatorname{ht}\left(#1\right)}
\newcommand{\emptyseq}{\Lambda}

\newcommand{\N}{\mathbb N}
\newcommand{\Q}{\mathbb Q}
\newcommand{\R}{\mathbb R}
\newcommand{\C}{\mathbb C}
\newcommand{\F}{\mathbb F}




\usepackage{xparse}

\ExplSyntaxOn
\tl_new:N \l_septatrix_env_tl
\NewDocumentCommand \getenv { o m }
{
	\sys_get_shell:nnN { kpsewhich ~ --var-value ~ #2 }
	{ \int_set:Nn \tex_endlinechar:D { -1 } }
	\l_septatrix_env_tl
	\IfNoValueTF {#1}
	{ \tl_use:N \l_septatrix_env_tl }
	{ \tl_set_eq:NN #1 \l_septatrix_env_tl }
}
\tl_const:Nn \c_getenv_par_tl { \par }

\NewDocumentCommand{\ifenvsetTF}{mmm}
{
	\sys_get_shell:nnN { kpsewhich ~ --var-value ~ #1 } { } \l_tmpa_tl
	\tl_if_eq:NNTF \l_tmpa_tl \c_getenv_par_tl { #3 } { #2 }
}
\ExplSyntaxOff

\usepackage{datetime}
% \newdate{date}{06}{09}{2012}
% \date{\displaydate{date}}
\date{\today}
\newcommand{\envOrDefault}[2]{\ifenvsetTF{#1}{\getenv{#1}}{#2}}

\author{\envOrDefault{au4thor}{Holo}}




\title{Exercise 2}
\begin{document}
\maketitle
\ex{}
Let $\Bbb P, A, f:A\to V^\Bbb P$ be as in the question, for each generic $G$ let $\bonk{G}$ be the unique $x\in A\cap G$, define $\operatorname{\uparrow}:\Bbb P\times A\to 2^\Bbb P$, $g:V^\Bbb P\times A\to V^\Bbb P$ and $\sigma_\cdot:A\to V^\P$ as:

\begin{itemize}
	\item $x\operatorname{\uparrow}a$ is the set of all common strengthening of $x$ and $a$
	\item $g(\tau, a)=\{(\pi, x)\mid\exists y (\pi, y)\in \tau\land x\in y\operatorname{\uparrow}a\}$
	\item $\sigma_a=g(f(a),a)$
\end{itemize}

Notice that $(f(\bonk{G}))_G=(\sigma_{\bonk{G}})_G$, the $\subseteq$ direction is follows from the fact that every 2 elements in $G$ have a common strengthening, indeed if $\tau\in f(\bonk{G})$ is don't discarded by $G$ then its right side is in $G$, in particular $\pi_2(\tau)\operatorname{\uparrow}\bonk{G}\ne \emptyset$ (where $\pi_i$ is the projection function) and so there is some $\pi\in \sigma_{\bonk{G}}$ such that $\pi_1(\pi)=\pi_1(\tau)$ and $\pi_2(\pi)\in G$.\\
The $\supseteq$ direction follows from the fact that $G$ is closed downwards, if $\pi\in \sigma_{\bonk{G}}$ is not discarded by $G$, then $\pi_2(\pi)$ comes from some $\tau\in f(\bonk{G})$ with $\pi_1(\tau)=\pi_1(\pi)$ and $\pi_2(\tau)\le\pi_2(\pi)$, which means that $\tau$ is also not discarded by $G$ because it is closed downwards.

In addition if $Q$ is a different generic ideal such that for some $x$ we have $x\in (\sigma_{\bonk{G}})_Q$ then $\bonk{G}=\bonk{Q}$ (in other words, $\sigma_{\bonk{G}}$ and $\sigma_{\bonk{H}}$ for $G,H$ generics that don't have a common $A$-member don't interfere with one another) because any right side  of an element of $\sigma_{\bonk{G}}$ must have stronger tag than $\bonk{G}$ and $Q$ is closed downwards which means that $\bonk{G}\in A\cap Q$ which by definition is equal to $\bonk{Q}$, so we can let $\sigma_f=\bigcup_{a\in A}\sigma_a$.

\ex[4]{}
\sub{}
For $(2)\to (1)$ in the book, notice that $(1)$ is a special case of $(2)$.
For $(2)\to (3)$ notice that assuming $(2)$ we have $\{r\mid r\Vdash^* \psi\}\supseteq \{r\mid r\ge p\}$, and the latter trivially dense above $p$.

For the atomic $(1)\to(2)$ direction, let $p\Vdash^*\tau_1=\tau_2$ and $r>p$, I want to show that for all $(\pi_1,s_1)\in\tau_1$ the set $\{q\ge r\mid q\ge s_1\implies \exists(\pi_2,s_2)\in\tau_2(q\ge s_2\land q\Vdash^*\pi_1=\pi_2)\}$ is dense above $r$, indeed take an element $s$ above $r$, then because $p\Vdash^*\tau_1=\tau_2$ and that $s\ge r\ge p$ there exists some $t\in \{q\ge p\mid q\ge s_1\implies \exists(\pi_2,s_2)\in\tau_2(q\ge s_2\land q\Vdash^*\pi_1=\pi_2)\}$ greater than $s$, but because $t\ge s\ge r$ we have that $t\in \{q\ge r\mid q\ge s_1\implies \exists(\pi_2,s_2)\in\tau_2(q\ge s_2\land q\Vdash^*\pi_1=\pi_2)\}$, so that set is indeed dense above $r$. A symmetric argument will finish this proof.

Now assume $p\Vdash^* \tau_1\in\tau_2$ and $r\ge p$, take $s\ge r\ge p$, there is some $t\in\{q\mid (\pi,x)\in\tau_2(q\ge x\land q\Vdash^*\pi=\tau_1)\}$ that is stronger than $s\ge r$, so the set is dense above $r$.

For $(3)\to(1)$ we just need to prove that if $\{r\mid D\text{ is dense above }r\}$ is dense above $p$, then $D$ is dense above $p$, indeed take $q\ge p$ and $r\in \{r\mid D\text{ is dense above }r\}$ greater than $q$, then let $q'>r$, because $D$ is dense above $r$ there exists $t\in D$ such that $t\ge q'>r\ge q\ge p$, hence $D$ is dense above $D$.

To finish all of the details of the proof of the lemma from the book we need to show $(1)\to(2)$ and $(3)\to(1)$ for $\lnot$ and $\exists x$.

For $(1)\to(2)$, assume $p\Vdash^*\lnot\phi$ and $r\ge p$, then every $t\ge r$ is also stronger than $p$, hence does not force$^*$ the sentence $\phi$, therefore $r\Vdash^* \lnot\phi$. Assume $p\Vdash^*\exists x \phi(x)$ and take $r\ge p$ and $t\ge r$, because $t\ge p$ there exists $q\ge t$ such that $\exists \sigma\in V^\Bbb P (q\Vdash^* \phi(\sigma))$, in particular $\{q\mid \exists \sigma\in V^\Bbb P (q\Vdash^* \phi(\sigma))\}$ is dense above $r$.

For $(3)\to(1)$ for the $\lnot\phi$ case, assume $p\not\Vdash^* \lnot\phi$ then there is some $r\ge p$ that forces$^*$ $\phi$, by the induction hypothesis we get that every $t\ge p$ also forces$^*$ $\phi$, a contradiction. The $\exists x\phi(x)$ case follows from the same fact as the atomic case, that if $\{r\mid D\text{ is dense above }r\}$ is dense above $p$, then $D$ is dense above $p$

\sub{}
Assume $p\Vdash^* \phi$, we want to prove $p\Vdash^*\lnot\lnot\phi$, in particular we want to show:
\begin{align*}
	&\forall q\ge p (\lnot q\Vdash\lnot \phi)\\
	\iff&\forall q\ge p (\lnot (\forall r\ge q (\lnot r\Vdash^*\phi)))\\
	\iff&\forall q\ge p  (\exists r\ge q (r\Vdash^*\phi))
\end{align*}

But any $r\ge q$ will witness it is true, as $r\ge q\ge p$ and from (4.1) $r\Vdash^* \phi$.
\sub{}
Let $D_\psi$ be as in the question and let $p\in \Bbb P$ be any term, if there is $q\ge p$ such that $q\Vdash^* \psi$ then we are done as $p\le q\in D_\psi$.
Otherwise we have that $p\Vdash^* \lnot\psi$ by definition, so $p\in D_\psi$.

So assume $p\Vdash^*\lnot\lnot\psi$, the set $D_\psi$ is dense above $p$, by definition there is no $q\ge p$ that forces$^*$ $\lnot\psi$, so $\{q\mid q\Vdash^* \psi\}=D_\psi\cap \{q\mid q\ge p\}$ is dense above $p$, by (4.1) this is equivalent to $p\Vdash\psi$.

\ex[6]{}
Assume $(p\Vdash^*\phi(\overline\tau))^M$, in particular we have that for every $G$ a generic containing $p$ we have $\exists q\in G (q\Vdash^* \phi(\overline{\tau}))^M$, by theorem 3.5 in the book we have $M[G]\models \phi(\overline{\tau}_G)$, hence by definition $p\Vdash_\Bbb P^M \phi(\overline{\tau})$

Now assume $p\Vdash_\Bbb P^M \phi(\overline{\tau})$ and let $r\ge p$ and $G$ be a $\Bbb P$-generic contains $r$, by definition $M[G]\models \phi(\overline{\tau}_G)$, by theorem 3.5 in the book there exists some $t\in G$ such that $(t\Vdash^*\phi(\overline\tau))^M$, let $q\in G$ be stronger than $r$ and $t$, by (4.1) we know that $(q\Vdash^*\phi(\overline\tau))^M$, hence $\{q\mid q\Vdash^*\phi(\overline\tau)\}$ is (dense above $p$)$^M$, which implies that $(p\Vdash^*\phi(\overline\tau))^M$

To get the form we had in class all we need to do is "concat" this result to question theorem 3.5, if $G$ is a generic and $M[G]\models \phi(\overline{\tau}_G)\iff \exists p\in G\; (p\force^*\phi(\overline{\tau}))^M\iff \exists p\in G\; p\force_\P^M \phi(\overline{\tau})$.

\newpage
\ex{}
\sub{}
Let $D$ be the set $\{p\mid \exists \sigma\in V^\Bbb P (p\Vdash \psi(\sigma, \overline{\tau}))\}$, we know this set is dense above $0_\Bbb P$, which means it is just dense.

Let $A\subseteq D$ be a maximal anti-chain, and for each $x\in A$ let $f(x)$ be a name that is a witness of $x\in D$, let $\sigma_f$ be the name from exercise 1.

For every generic ($=$ for every generic contains $0_\Bbb P$) we have that $M[G]\models \psi((\sigma_f)_G,\overline{\tau}_G)$, which by definition means $0_\Bbb P\Vdash \psi(\sigma_f, \overline{\tau})$.

\sub{}
Let $\psi(x,y,w,z)=(x \text{ is a function from $w$ to $z$})\land (y\text{ is a function from $w$ to $z$}\implies x=y)$, notice that for every $y',w',z'$ such that $(z'=\emptyset\implies w'=\emptyset)$ we have that $\exists x\psi(x,y',w',z')$ is tautology.

Let $G$ be a generic ideal and $f:\alpha\to\beta$ function in $M[G]$, let $\tau$ be the name that $G$ interpret as $f$, because $f$ is a function we have that either $\beta\ne 0$ or both $\alpha$ and $\beta$ are $0$, in particular for every $Q$ a generic ideal we have that $M[G]\models\exists x\psi(x,\tau_Q,\check{\alpha}_Q,\check{\beta}_Q)$, so $0_\Bbb P\Vdash \exists x\psi(x,\tau,\check{\alpha},\check{\beta})$, from (7.1) we have that there is some name $\sigma$ such that $0_\Bbb P\Vdash \psi(\sigma,\tau,\check{\alpha},\check{\beta})$.

Clearly $0_\Bbb P\Vdash \sigma:\check{\alpha}\to\check{\beta}$ and because $\tau_G=f$ is a function from $\alpha$ to $\beta$ we have that $f=\tau_G=\sigma_G$.
\end{document}




