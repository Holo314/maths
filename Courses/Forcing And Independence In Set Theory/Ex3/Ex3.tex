\documentclass[12pt,reqno]{article}
\usepackage[margin=3cm]{geometry}
\usepackage[utf8]{inputenc}


\usepackage{xfp}
\usepackage{natbib}
\usepackage{graphicx}
\usepackage{amsthm}
\usepackage{amsmath}
\usepackage{amssymb}
\usepackage{cases}
\usepackage{microtype}
\usepackage{hyperref}
\usepackage{mathrsfs}

\newtheorem{theorem}{Theorem}[section]
\newtheorem{proposition}[theorem]{Proposition}
\newtheorem{corollary}[theorem]{Corollary}
\newtheorem{lemma}[theorem]{Lemma}
\newtheorem{conjecture}[theorem]{Conjecture}
\newtheorem{example}[theorem]{Example}
\theoremstyle{definition}
\newtheorem{definition}[theorem]{Definition}
\theoremstyle{remark}
\newtheorem{remark}[theorem]{Remark}
\theoremstyle{exercise}
\newtheorem{exercise}[section]{Exercise}
\theoremstyle{subExercise}
\newtheorem{subExercise}{Part}[section]
\newtheorem{subSubExercise}{Sub Part}[subExercise]
\numberwithin{equation}{section}
\newcommand{\acl}[2]{\operatorname{acl}^{#2}\left(#1\right)}
\newcommand{\rank}[2]{\operatorname{rank}_{#2}\left(#1\right)}
\newcommand{\dcl}[2]{\operatorname{dcl}^{#2}\left(#1\right)}
\newcommand{\Aut}[1]{\operatorname{Aut}\left(#1\right)}
\newcommand{\Av}[1]{\operatorname{Av}\left(#1\right)}
\newcommand{\comment}[2]{#2}
\newcommand{\cof}[1]{\operatorname{cof}\left(#1\right)}
\renewcommand{\cal}[1]{\mathcal{#1}}
\newcommand{\D}[2][{}]{\text{Diag}_{#1}\left(#2\right)}
\renewcommand{\phi}{\varphi}
\newcommand{\code}[1]{\left\lceil#1\right\rceil}
\setcounter{MaxMatrixCols}{20}
\sloppy

\newcommand{\ex}[2][\fpeval{\value{section}+1}]{\setcounter{section}{\fpeval{#1 - 1}}
	\begin{exercise}
		#2
\end{exercise}}
\newcommand{\sub}[2][\fpeval{\value{subExercise}+1}]{\setcounter{subExercise}{\fpeval{#1 - 1}}
	\begin{subExercise}
		#2
\end{subExercise}}
\newcommand{\subb}[2][\fpeval{\value{subSubExercise}+1}]{\setcounter{subSubExercise}{\fpeval{#1 - 1}}
	\begin{subSubExercise}
		#2
\end{subSubExercise}}
\newcommand{\cl}{\mbox{cl}}
\newcommand{\CB}[1]{\operatorname{CB}\left(#1\right)}
\newcommand{\MR}[1]{\operatorname{MR}\left(#1\right)}
\newcommand{\MD}[1]{\operatorname{MD}\left(#1\right)}
\newcommand{\monster}{{\mathfrak C}}
\newcommand{\tp}[1]{\operatorname{tp}\left(#1\right)}
\newcommand{\hull}[3]{\operatorname{Hull}^{#2}_{#3}\left(#1\right)}
\newcommand{\stp}[1]{\operatorname{stp}\left(#1\right)}
\newcommand{\dom}[1]{\operatorname{dom}\left(#1\right)}
\newcommand{\range}[1]{\operatorname{range}\left(#1\right)}
\newcommand{\cnst}[2]{\operatorname{const}_{#1}\left(#2\right)}
\renewcommand{\c}{{\mathfrak c}}
\newcommand{\club}[1]{\operatorname{club}\left(#1\right)}
\newcommand{\Lev}[1]{\operatorname{Lev}\left(#1\right)}
\newcommand{\height}[1]{\operatorname{ht}\left(#1\right)}
\newcommand{\emptyseq}{\Lambda}

\newcommand{\N}{\mathbb N}
\newcommand{\Q}{\mathbb Q}
\newcommand{\R}{\mathbb R}
\newcommand{\C}{\mathbb C}
\newcommand{\F}{\mathbb F}




\usepackage{xparse}

\ExplSyntaxOn
\tl_new:N \l_septatrix_env_tl
\NewDocumentCommand \getenv { o m }
{
	\sys_get_shell:nnN { kpsewhich ~ --var-value ~ #2 }
	{ \int_set:Nn \tex_endlinechar:D { -1 } }
	\l_septatrix_env_tl
	\IfNoValueTF {#1}
	{ \tl_use:N \l_septatrix_env_tl }
	{ \tl_set_eq:NN #1 \l_septatrix_env_tl }
}
\tl_const:Nn \c_getenv_par_tl { \par }

\NewDocumentCommand{\ifenvsetTF}{mmm}
{
	\sys_get_shell:nnN { kpsewhich ~ --var-value ~ #1 } { } \l_tmpa_tl
	\tl_if_eq:NNTF \l_tmpa_tl \c_getenv_par_tl { #3 } { #2 }
}
\ExplSyntaxOff

\usepackage{datetime}
% \newdate{date}{06}{09}{2012}
% \date{\displaydate{date}}
\date{\today}
\newcommand{\envOrDefault}[2]{\ifenvsetTF{#1}{\getenv{#1}}{#2}}

\author{\envOrDefault{au4thor}{Holo}}



\usepackage{skak}
\usepackage{relsize}
\usepackage{graphicx}
\usepackage{mathtools}

\usepackage{textcomp}
\usepackage{bbding}

\usepackage{soul}

\newcommand{\flower}{\text{\scalebox{0.75}{\raisebox{-0.7ex}{
				\rotatebox{90}{\textleaf}\hspace{-0.3em}
				\scalebox{0.7}{\textleaf}\hspace{-1.35em}
				\raisebox{1ex}{\scalebox{0.8}{\FiveFlowerOpen}}
}}}}
\title{Exercise 3}
\begin{document}
\maketitle
\ex{Nice Names}
\sub{}
Let $f:\omega\to 2^\P$ be a function defined as $f(n)=\{p\in\P\mid p\force \check{n}\in\sigma\}$ and let $A_n$ be some maximal anti-chain of $f(n)$.

We will see that the nice name $\sigma^*=\bigcup_{n<\omega}\{\check{n}\}\times A_n$ is a nice name such that $0_\P\force \sigma=\sigma^*$, or equivalently that $M[\flower]\models \sigma_\flower=\sigma^*_\flower$ for all generic ideals $\flower$.

Fix some $\flower$ and $n\in\omega$ a natural, assume $M[\flower]\models n\in \sigma_\flower$, I claim that $\flower\cap A_n\ne\emptyset$, first remember that $g(n)=\{p\in \P\mid p\force \check{n}\in\sigma\lor p\force \check{n}\notin\sigma\}\supseteq f(n)$ is dense in $\P$, so extend $A_n$ into $B_n$ a maximal anti-chain in $g(n)$, because $B_n$ is maximal anti-chain in a dense set, it is also maximal anti-chain in $\P$.

Let $p\in \flower\cap B_n$, if $p\notin A_n$ it means that $p\force \check{n}\notin\sigma$ which is false, hence $p\in \flower\cap A_n$ which means by definition that $M[\flower]\models n\in\sigma^*_\flower$. 

The direction of $M[\flower]\models n\notin \sigma^*_\flower\implies M[\flower]\models n\notin \sigma_\flower$ is just the contrapositive of the previous case.

The directions of $M[\flower]\models n\in\sigma^*_\flower\implies M[\flower]\models n\in\sigma_\flower$ and the contrapositive $M[\flower]\models n\notin\sigma_\flower\implies M[\flower]\models n\notin\sigma^*_\flower$ are directly from the definition of $\sigma^*$.

\sub{}
Let $\P=\Add{\omega}{\omega_2}$, and note that $|\aleph_2|\le|\P|\le|[\aleph_0\times\aleph_2\times 2]^{<\omega}|=|[\aleph_2]^{<\omega}|=\aleph_2$.

Let $\cal A$ be the set of anti-chains of $\P$, because $\P$ is c.c.c. we have that $\aleph_2=|\P|\le|\cal A|\le |[\aleph_2]^{\le \omega}|=|[\aleph_2]^{< \omega}\cup [\aleph_2]^{\omega}|=\aleph_2+|[\aleph_2]^{\omega}|=|[\aleph_2]^{\omega}|\le \aleph_2^{\aleph_0}=\left(2^{\aleph_1}\right)^{\aleph_0}=2^{\aleph_1\times\aleph_0}=2^{\aleph_1}=\aleph_2$

Notice that a function that sends $f:\omega\to\cal A$ to $\bigcup_{n<\omega}\{\check{n}\}\times f(n)$ is a bijection from the nice names to $^{\aleph_0}\cal A$ so the cardinality of the set of nice names is exactly $|\cal A|^{\aleph_0}=\aleph_2^{\aleph_0}=\aleph_2$

\sub{}

Let $F$ be a bijection from the nice $\P$-names of $M$ to $\aleph_2$, because $F,\flower\in M[\flower]$ and $M[\flower]\models AC$ we can define inside of $M[\flower]$ a function that for each $f\in 2^{\aleph_0}$ chooses some $\sigma\in\dom{F}$ such that $\sigma_\flower=f$ and sends it to $F(\sigma)$, this is an injective function because $F$ is injective and given $a\ne b\in G[\flower]$ they are not evaluated from the same $\P$ name.
	
We have shown in class that $M[\flower]\models 2^{\aleph_0}\ge\aleph_2$ and so because $M[\flower]$ satisfy Cantor–Bernstein we have $M[\flower]\models 2^{\aleph_0}=\aleph_2$.

%\newpage
\ex[7]{Homogeneous Posets}
\sub{}
For 2 partial function $p,q$ let $K(p,q)=\dom{p}\cap \dom{q}$ and $p^{(q)}=p\cup (q\restriction \dom{q}\setminus K(p,q))$.

Notice that $p^{(q)}\ge p$ and that if $p,q$ are comparable then $p^{(q)}=\max(p,q)$.

Now let $\P$ be a poset of partial functions ordered by inclusion and $\pi:\P\to\P$ be a bijection that is bit-wise, that is $\dom{p}=\dom{\pi(p)}$ and $\pi(p)(n)$ depends only on $p(n)$ and $n$ for $n\in\dom{p}$, then we have that $\pi$ is an automorphism.

Indeed let $\tau(n,p(n))$ be $(n,\pi(p)(n))$, because $\pi$ is a bijection so is $\tau$, and let $p\subseteq q$, if $(a,b)\in p$ then $(a,b)\in q$ so $\tau(a,b)\in \pi(p)$ and $\tau(a,b)\in \pi(q)$, and if $(a,b)\in \pi(p)$ then $\tau^{-1}(a,b)\in p$ so $\tau(\tau^{-1}(a,b))=(a,b)\in \pi(q)$ so $\pi$ is order-preserving.

Now fix some $p,q\in\P$ and lets define the automorphism $\pi_p^q:\P\to\P$ that swaps $p^{(q)}$ and $q^{(p)}$ by swapping $(n,p(n))$ with $(n,q(n))$ for all $n\in K(p,q)$ and let it not change any other pair.

Indeed $\pi_p^q$ is bit-wise, $\pi_p^q(t)(n)=t(n)$ if $n\notin K(p,q)$ or $t(n)\notin\{p(n),q(n)\}$, otherwise if $t(n)=p(n)$ let $p(t)(n)=q(n)$ and vice versa.

To see it swaps $p^{(q)}$ with $q^{(p)}$ notice that $p^{(q)},q^{(p)}$ have the domain of of $\dom{p}\cup\dom{q}$ and they agree on their domain apart from (maybe) $K(p,q)$, so let $n\in K(p,q)$ and we get that $p^{(q)}(n)=p(n)$, so $\pi_p^q(p^{(q)})(n)=q(n)$ by definition, and vice versa.

Because $\Add{\kappa}{1}$ and $Col(\omega,\lambda)$ are posets of partial functions ordered by inclusion we are done.

\sub{}
Let $G$ be any generic we know that $\phi(\overline{\check{x}_G})$ either holds in $M[G]$ or its negation holds, WLOG assume it holds, and take $p\in G$ such that $p\force \phi(\overline{\check{x}})$.

Let $q\in\P$ be any element, and let $\pi$ be an automorphism that sends $r\ge p$ to $t\ge q$, because $r\ge p$, it also forces $\phi(\overline{\check{x}})$ and so from problem (6.2) we have that $t\force \phi(\overline{\pi(\check{x})})$ and from (6.1) we can conclude that $t\force \phi(\overline{\check{x}})$, so $\{p\in \P\mid p\force\phi(\overline{\check{x}})\}$ is dense above $0_\P$, hence $0_\P$ also forces that.



\end{document}




