\documentclass[12pt,reqno]{article}
\usepackage[margin=3cm]{geometry}
\usepackage[utf8]{inputenc}


\usepackage{xfp}
\usepackage{natbib}
\usepackage{graphicx}
\usepackage{amsthm}
\usepackage{amsmath}
\usepackage{amssymb}
\usepackage{cases}
\usepackage{microtype}
\usepackage{hyperref}
\usepackage{mathrsfs}

\newtheorem{theorem}{Theorem}[section]
\newtheorem{proposition}[theorem]{Proposition}
\newtheorem{corollary}[theorem]{Corollary}
\newtheorem{lemma}[theorem]{Lemma}
\newtheorem{conjecture}[theorem]{Conjecture}
\newtheorem{example}[theorem]{Example}
\theoremstyle{definition}
\newtheorem{definition}[theorem]{Definition}
\theoremstyle{remark}
\newtheorem{remark}[theorem]{Remark}
\theoremstyle{exercise}
\newtheorem{exercise}[section]{Exercise}
\theoremstyle{subExercise}
\newtheorem{subExercise}{Part}[section]
\newtheorem{subSubExercise}{Sub Part}[subExercise]
\numberwithin{equation}{section}
\newcommand{\acl}[2]{\operatorname{acl}^{#2}\left(#1\right)}
\newcommand{\rank}[2]{\operatorname{rank}_{#2}\left(#1\right)}
\newcommand{\dcl}[2]{\operatorname{dcl}^{#2}\left(#1\right)}
\newcommand{\Aut}[1]{\operatorname{Aut}\left(#1\right)}
\newcommand{\Av}[1]{\operatorname{Av}\left(#1\right)}
\newcommand{\comment}[2]{#2}
\newcommand{\cof}[1]{\operatorname{cof}\left(#1\right)}
\renewcommand{\cal}[1]{\mathcal{#1}}
\newcommand{\D}[2][{}]{\text{Diag}_{#1}\left(#2\right)}
\renewcommand{\phi}{\varphi}
\newcommand{\code}[1]{\left\lceil#1\right\rceil}
\setcounter{MaxMatrixCols}{20}
\sloppy

\newcommand{\ex}[2][\fpeval{\value{section}+1}]{\setcounter{section}{\fpeval{#1 - 1}}
	\begin{exercise}
		#2
\end{exercise}}
\newcommand{\sub}[2][\fpeval{\value{subExercise}+1}]{\setcounter{subExercise}{\fpeval{#1 - 1}}
	\begin{subExercise}
		#2
\end{subExercise}}
\newcommand{\subb}[2][\fpeval{\value{subSubExercise}+1}]{\setcounter{subSubExercise}{\fpeval{#1 - 1}}
	\begin{subSubExercise}
		#2
\end{subSubExercise}}
\newcommand{\cl}{\mbox{cl}}
\newcommand{\CB}[1]{\operatorname{CB}\left(#1\right)}
\newcommand{\MR}[1]{\operatorname{MR}\left(#1\right)}
\newcommand{\MD}[1]{\operatorname{MD}\left(#1\right)}
\newcommand{\monster}{{\mathfrak C}}
\newcommand{\tp}[1]{\operatorname{tp}\left(#1\right)}
\newcommand{\hull}[3]{\operatorname{Hull}^{#2}_{#3}\left(#1\right)}
\newcommand{\stp}[1]{\operatorname{stp}\left(#1\right)}
\newcommand{\dom}[1]{\operatorname{dom}\left(#1\right)}
\newcommand{\range}[1]{\operatorname{range}\left(#1\right)}
\newcommand{\cnst}[2]{\operatorname{const}_{#1}\left(#2\right)}
\renewcommand{\c}{{\mathfrak c}}
\newcommand{\club}[1]{\operatorname{club}\left(#1\right)}
\newcommand{\Lev}[1]{\operatorname{Lev}\left(#1\right)}
\newcommand{\height}[1]{\operatorname{ht}\left(#1\right)}
\newcommand{\emptyseq}{\Lambda}

\newcommand{\N}{\mathbb N}
\newcommand{\Q}{\mathbb Q}
\newcommand{\R}{\mathbb R}
\newcommand{\C}{\mathbb C}
\newcommand{\F}{\mathbb F}




\usepackage{xparse}

\ExplSyntaxOn
\tl_new:N \l_septatrix_env_tl
\NewDocumentCommand \getenv { o m }
{
	\sys_get_shell:nnN { kpsewhich ~ --var-value ~ #2 }
	{ \int_set:Nn \tex_endlinechar:D { -1 } }
	\l_septatrix_env_tl
	\IfNoValueTF {#1}
	{ \tl_use:N \l_septatrix_env_tl }
	{ \tl_set_eq:NN #1 \l_septatrix_env_tl }
}
\tl_const:Nn \c_getenv_par_tl { \par }

\NewDocumentCommand{\ifenvsetTF}{mmm}
{
	\sys_get_shell:nnN { kpsewhich ~ --var-value ~ #1 } { } \l_tmpa_tl
	\tl_if_eq:NNTF \l_tmpa_tl \c_getenv_par_tl { #3 } { #2 }
}
\ExplSyntaxOff

\usepackage{datetime}
% \newdate{date}{06}{09}{2012}
% \date{\displaydate{date}}
\date{\today}
\newcommand{\envOrDefault}[2]{\ifenvsetTF{#1}{\getenv{#1}}{#2}}

\author{\envOrDefault{au4thor}{Holo}}



\usepackage{skak}
\usepackage{relsize}
\usepackage{graphicx}
\usepackage{mathtools}

\usepackage{textcomp}
\usepackage{bbding}

\usepackage{soul}



\title{Exercise 4}
\begin{document}
\maketitle
\begin{cExercise}[Dense projections and Quotient posets]
	\begin{cPart}
		Let $\pi,\mathbb P, \mathbb Q$ and $\Gamma_\mathbb Q^\mathbb P$ be as in the question, let $G$ be any $M$-generic subset of $\mathbb P$ and let $\Gamma=(\Gamma_\mathbb Q^\mathbb P)_G, \Gamma'=(\check\pi''\Gamma_\mathbb P)_G$.
		
		Let $p,q\in\Gamma$ and $p',q'\in\Gamma'$ be strengthening of $p,q$ respectively, let $t,r\in G$ be such that $\pi(t)=p',\pi(r)=q'$, because $G$ is generic we have $t,r<w\in G$, and because $\pi$ is a weak homomorphism, $\pi(w)>q',p'$ hence $\pi(w)>q,p$, and almost by definition $\pi(w)\in \Gamma$.
		
		Because $\Gamma$ is closed downwards by definition, we only left to show that it intersect with every maximal anti-chain of $\mathbb Q$ that is in $M$, let $A$ be such maximal anti-chain. For each $x\in\mathbb Q$ let $x^\dagger$ be a maximal anti-chain of $\{y\in \pi''\mathbb P\mid y\ge x\}$, and let $A^\dagger=\bigcup\{y^\dagger\mid y\in A\}$, because $\pi''\mathbb P$ is dense, we have that $A^\dagger$ is a maximal anti-chain as well.
		
		Let $A^\ddagger$ be the inverse image of $A^\dagger$ by $\pi$, because $\pi$ preserve incompatibility we must have that $A^\ddagger$ contains a maximal anti-chain, let $t\in A^\ddagger\cap G$, $\pi(t)\in A^\dagger\cap \Gamma$. By construction of $A^\dagger$ there exists some $p\in A$ such that $\pi(t)>p$, and because $\Gamma$ is closed downwards we have that $p\in A\cap \Gamma$, hence $M[G]\vDash"\Gamma\text{ is generic over $M$ for $\mathbb Q$}"$.
		
		Finally because $G$ was arbitrary we can conclude that $0_\mathbb P$ forces that.
	\end{cPart}
	\begin{cPart}
		Let $G\subseteq \mathbb P$ be generic.
		
		Notice that $H$ is exactly $\Gamma$ from (1.1), so $H$ is generic over $M$ for $\mathbb Q$.
		
		Because being downwards closed is downwards absolute (from $\mathbb P$ to $\mathbb P/_\pi H$) and every 2 elements having a common extension in the subset is absolute (again from $\mathbb P$ to $\mathbb P/_\pi H$) we only need to care about it intersecting every maximal anti-chain of $\mathbb P/_\pi H$ in $M[H]$.
		
		Let $p,q\in\mathbb P/_\pi H$, by definition $\pi(p),\pi(q)\in H\implies \pi(p)\|\pi(q)\implies p\|q$, the last implication is because $\pi$ is homomorphism and the one before that is because $H$ is a generic filter.
		
		Let $A\subseteq \mathbb P/_\pi H$ be any maximal anti-chain in $M[H]$, that is $A=\{p\}$ for some $p\in\mathbb P/_\pi H$ (as any $2$ elements of the quotient poset are compatible), so $A\in M$. Extend $A$ to $A'$ a maximal anti-chain in $\mathbb P$, because $G$ is $M$ generic for $\mathbb P$ we know that $G\cap A'\ne\emptyset$, but $G\subseteq \mathbb P/_\pi H$ so $G\cap A'=A\implies G\cap A=A\ne\emptyset$.
		
		For the other direction, let $G$ be any subset of $\mathbb P$ such that $H$ is $M$-generic for $\mathbb Q$ and $G$ is $M[H]$-generic for $\mathbb P/_\pi H$.
		
		Similarly to before, every 2 elements in $\mathbb P/_\pi H$ are compatible, in particular any 2 elements in $G$ are compatible, as $G\subseteq \mathbb P/_\pi H$.
		
		Let $p\in G$ and $q<p$, because $\pi$ is a weak homomorphism we have that $\pi(q)<\pi(p)\in H$, so $\pi(q)\in H\implies q\in \mathbb P/_\pi H$, because $q<p\in \mathbb P/_\pi H$ and $G$ is closed downwards in $\mathbb P/_\pi H$ we must have that $q\in G$.
		
		Lastly, let $p,q\in \mathbb P/_\pi H$, let $D_{p,q}=\{r\in \mathbb P\mid r\ge p,q\lor r\bot p\lor r\bot q\}$ and $D'=\pi''D_{p,q}$, because $D_{p,q}$ is dense and $\pi$ is dense homomorphism we know that $D'$ is dense hence intersecting $H$, let $r\in D_{p,q}$ be such that $\pi(r)\in D'\cap H$. Because $\pi(r)\in H$ we have that $r\in \mathbb P/_\pi H$ but because every 2 elements of $\mathbb P/_\pi H$ are compatible in $\mathbb P$, we have that $r\| p$ and $r\| q$, by construction of $D_{p,q}$ we get that $r\ge p,q$.
		
		That means that every 2 elements in $\mathbb P/_\pi H$ are compatible in $\mathbb P/_\pi H$, so the only generic filter is $\mathbb P/_\pi H$, so $G=\mathbb P/_\pi H$.
		
		Now take $A$ to be any maximal anti-chain, $\pi''A$ is a maximal anti-chain as well, hence $\{\pi(p)\}=H\cap \pi''A$ hence $G\cap A=\mathbb P/_\pi H\cap A=\{p\}\ne\emptyset$
	\end{cPart}
\end{cExercise}

\begin{cExercise}[Collapse Criterion]
	First note that $\mathbb Q/q$ absorbs $\mathbb Q$, indeed for each $M$-generic for $\mathbb Q/q$ we can take the downwards closure and get an $M$-generic for $\mathbb Q$.
	
	Therefore $0_{\mathbb Q/q}\Vdash |\check{\omega}|=|\check{\delta}|$, so $\mathbb Q/q$ is not $\delta$.c.c. and has a maximal anti-chain of size $\delta$.
	
	Let $\dot g$ be a $\mathbb Q$ name that $0_\mathbb Q\Vdash"\dot g:\omega\to \Gamma_\mathbb Q\text{ is surjective}"$
	
	We will construct recursively an embedding from $\delta^{<\omega}$ to $\mathbb Q$ as a tree $T$:
	
	\begin{itemize}
		\item $T_\emptyseq=0_\mathbb Q$
		\item Let $t\in \delta^{n+1}$ and assume $T_t$ is defined, let $C_t=\{c_t^{(\alpha)}\mid \alpha\in\delta\}$ be a maximal anti-chain of size $\delta$ in $\mathbb Q/ T_t$ such that for each $c\in C_t$ we have that $c$ decides the value of $\dot g(\check n)$ and let $T_{t\frown\{i\}}=c_t^{(i)}$
	\end{itemize}
	
	Note that the anti-chain above is always well defined, let $p\in\mathbb Q$ be any element and let $A\subseteq \mathbb Q/p$ be any maximal anti-chain that decides $\dot g(\check n)$, let $q\in A$ be any element and let $B\subseteq \mathbb Q/q$ of cardinality $\delta$, then $A'=A\setminus q\cup B$ is a maximal anti-chain of size $\delta$ that decides $\dot g(\check n)$.
	
	This map is clearly an injective homomorphism, to see that it is dense let $p\in \mathbb Q$ be any element, remember that $p\Vdash "\dot g:\omega\to \Gamma_\mathbb Q\text{ is surjective and }\check p\in\Gamma_\mathbb Q=\range{\dot g}"$ which implies $p\Vdash\exists n\in\check\omega\;\dot g(n)=\check p$ in particular there exists a name $\dot n$ such that $p\Vdash \dot g(\dot n)=\check p$, let $m$ be a real natural such that $p\Vdash \check m=\dot n$, thus $p\Vdash \dot g(\check m)=\check p$.
	
	The set $\operatorname{Lev}_T(m+1)$ is clearly a maximal anti-chain, so either $p\in \operatorname{Lev}_T(m+1)$, in which case we are done, or there exists $q\in \operatorname{Lev}_T(m+1)$ that is compatible with $p$, because $p,q$ are compatible and both decide $\dot g(\check m)$ they must agree with one another, that is $q\Vdash \dot g(\check m)=\check p$, in particular $q\Vdash \check p\in \range{\dot g}=\Gamma_\mathbb Q$ therefore $p$ is inside every generic $G$ that $q$ is in, in other words $p<q$.
	
	Finally, let $G$ be $M$-generic for $\mathbb Q$, and let $H'$ be the intersection of $G$ with the image of the tree, it is simple to see that $H'$ is $M$-generic over $T$ let $H''$ be the pre-image of $H'$ into $\delta^{<\omega}$, and let $H$ be the downwards closure of $H''$ into an $M$-generic for $\operatorname{Coll}(\omega,\delta)$.
	
	To see that $M[G]=M[H]$ first note that because $H$ is definable in $M[G]$ (with parameters) it means that $M[H]\subseteq M[G]$, to see the other direction, let $H^\dagger$ be the restriction of $H$ to $\delta^{<\omega}$, let $H^{\ddagger}$ be the image of $H^{\dagger}$ under the embedding to $T$, we can verify that $G$ is exactly the downward closure of $H^\ddagger$.
	
	First because $H^\ddagger\subseteq G$ the downwards closure of $H^\ddagger$ is clearly a subset of $G$, and the downwards closure is $M$-generic for $\mathbb Q$ (as $T$ is dense).
	
	Let $p\in G$ be any element, extend $\{p\}$ to a maximal anti-chain $A$, $A$ intersects with the downwards closure of $H^\ddagger$, but $p$ is compatible with every element of the downwards closure of $H^\ddagger$, so the intersection must be $\{p\}$, so $p$ is in the downwards closure of $H^\ddagger$.
	
	So we got that $G$ is definable in $M[H]$ (with parameters) and hence $M[G]\subseteq M[H]$ and we are done.
\end{cExercise}
\begin{cExercise}[Finishing the proof of Solovay’s Theorem][Yuval Paz][author]
	\begin{cPart}
		Notice that $Coll(\omega, <\kappa)$ is exactly the forcing product of $Coll(\omega,\alpha)$ for $\alpha<\kappa$ with finite support.
		
		To see that the forcing is $\kappa$.c.c let $A$ be a family of conditions of cardinality $\kappa$, by the sunflower lemma we may assume that the places where 2 conditions has nontrivial condition is a constant finite $r$ with $m=\max(r)<\kappa$, but then we have that the compatibility of the conditions depends only on the product up to $m+1$, which has cardinality $<\kappa$ and hence $\kappa$.c.c. 
	\end{cPart}
	\begin{cPart}
		Clearly $Coll(\omega,<\alpha)\times Coll(\omega,[\alpha,\kappa))\cong Coll(\omega,<\kappa)$ as witness by concatenation, or in the other direction, as witness by splitting conditions at $\alpha$.
		
		Let $G$ be generic in $Coll(\omega,<\kappa)$, and let $G'\subseteq Coll(\omega,<\alpha)\times Coll(\omega,[\alpha,\kappa))$ be image of $G$ under the isomorphism above. We have shown in the lecture that a subset of a product forcing notion $H\times K\subseteq \P\times\Q$ is generic if and only if $H$ is $\P$-generic over $V$, and $K$ is $\Q$-generic over $M[H]$, which is exactly the situation the question asks for.
	\end{cPart}
	\begin{cPart}
		Let $G=G_{<\kappa}$ for ease.
		
		First we notice that if $X\in M[G]$ such that $X$ is a bounded subset of $\kappa$, then there exists some $\alpha<\kappa$ such that $X\in M[G_{<\alpha}]$. Indeed if $X$ is as such, let $\tau=\{\{\alpha\}\times A_\alpha\}_{\alpha\in \sup X}$ be a nice name of $X$, because $\kappa$ is regular and satisfy $\kappa$.c.c, we must have that $\tau$ is some $Coll(\omega,<\alpha)$ for some $\alpha<\kappa$, which means that $X=\tau_{G}=\tau_{G_{<\alpha}}\in M[G_{<\alpha}]$. In particular, if $X\in M$ is any set, we can encode biject it into an ordinal, and decode the bijection in $M[G]$, so any subset of $X$ of cardinality $<\kappa$ first appear in some bounded stage.
		
		Rewording the above we get it neatly: if $A\in M$ and $M[G]\models B \subseteq A\land |B|=\aleph_0$ then there exists $\alpha<\kappa$ such that $B\in M[G_{<\alpha}]$.
		
		Now let $\Q$ be as in the question and let $\alpha$ be the first such ordinal and let $\lambda=\max(|\alpha|, |\Q|)^+$. Clearly we have that $M[H]\models"\lambda\text{ is uncountable}"$. Because $|\Q\times (Coll(\omega,\lambda))^{M[H]}|=\lambda$ and it collapses $\lambda$ to $\aleph_0$, by exercise 2 we have in $M$ that $\Q\times (Coll(\omega,\lambda))^{M[H]}\cong Coll(\omega,\lambda)=Coll(\omega,<\lambda+1)$.
		
		Let $K\subseteq (Coll(\omega,\lambda))^{M[H]}$ such that $H\times K\cong G_{<\lambda+1}$, but this means that $K$ is $(Coll(\omega,\lambda))^{M[H]}$-generic over $M[H]$. From the previous part we have that $M[G]=M[H][K][G_{[\lambda+1,\kappa)}]$.
		
		Lastly we note that $(Coll(\omega,\lambda))^{M[H]}$ is a superset of $Coll(\omega,\lambda)=Coll(\omega,\lambda+1)$, so any generic on the former will be generic for the latter, in particular $K$ is such. So from the previous part again we get that there is a $Coll(\omega,<\kappa)$-generic $G^\Q$ (that comes from $K\times G_{[\lambda+1,\kappa)}$).
		
		To see that $G^\Q$ is really generic over $M[H]$ note that $K$ is generic over $M[H]$ and $G_{[\lambda+1,\kappa)}$ is generic over $M[H][K]$.
	\end{cPart}
\end{cExercise}
\end{document}




