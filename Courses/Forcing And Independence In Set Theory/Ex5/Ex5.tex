\documentclass[12pt,reqno]{article}
\usepackage[margin=3cm]{geometry}
\usepackage[utf8]{inputenc}


\usepackage{xfp}
\usepackage{natbib}
\usepackage{graphicx}
\usepackage{amsthm}
\usepackage{amsmath}
\usepackage{amssymb}
\usepackage{cases}
\usepackage{microtype}
\usepackage{hyperref}
\usepackage{mathrsfs}
\usepackage{xparse}

\makeatletter
\DeclareRobustCommand\widecheck[1]{{\mathpalette\@widecheck{#1}}}
\def\@widecheck#1#2{%
	\setbox\z@\hbox{\m@th$#1#2$}%
	\setbox\tw@\hbox{\m@th$#1%
		\widehat{%
			\vrule\@width\z@\@height\ht\z@
			\vrule\@height\z@\@width\wd\z@}$}%
	\dp\tw@-\ht\z@
	\@tempdima\ht\z@ \advance\@tempdima2\ht\tw@ \divide\@tempdima\thr@@
	\setbox\tw@\hbox{%
		\raise\@tempdima\hbox{\scalebox{1}[-1]{\lower\@tempdima\box
				\tw@}}}%
	{\ooalign{\box\tw@ \cr \box\z@}}}
\makeatother

\newtheorem{theorem}{Theorem}[section]
\newtheorem{proposition}[theorem]{Proposition}
\newtheorem{corollary}[theorem]{Corollary}
\newtheorem{lemma}[theorem]{Lemma}
\newtheorem{conjecture}[theorem]{Conjecture}
\newtheorem{example}[theorem]{Example}
\theoremstyle{definition}
\newtheorem{definition}[theorem]{Definition}
\theoremstyle{remark}
\newtheorem{remark}[theorem]{Remark}
\theoremstyle{exercise}
\newtheorem{exercise}[section]{Exercise}
\theoremstyle{subExercise}
\newtheorem{subExercise}{Part}[section]
\newtheorem{subSubExercise}{Sub Part}[subExercise]
\numberwithin{equation}{section}
\newcommand{\acl}[2]{\operatorname{acl}^{#2}\left(#1\right)}
\newcommand{\rank}[2]{\operatorname{rank}_{#2}\left(#1\right)}
\newcommand{\dcl}[2]{\operatorname{dcl}^{#2}\left(#1\right)}
\newcommand{\Aut}[1]{\operatorname{Aut}\left(#1\right)}
\newcommand{\Av}[1]{\operatorname{Av}\left(#1\right)}
\newcommand{\comment}[2]{#2}
\newcommand{\cof}[1]{\operatorname{cof}\left(#1\right)}
\renewcommand{\cal}[1]{\mathcal{#1}}
\newcommand{\D}[2][{}]{\text{Diag}_{#1}\left(#2\right)}
\renewcommand{\phi}{\varphi}
\newcommand{\code}[1]{\left\lceil#1\right\rceil}
\setcounter{MaxMatrixCols}{20}
\sloppy

\usepackage{expl3}




\ExplSyntaxOn

\tl_new:N \l_septatrix_env_tl
\NewDocumentCommand \getenv { o m }
{
	\sys_get_shell:nnN { kpsewhich ~ --var-value ~ #2 }
	{ \int_set:Nn \tex_endlinechar:D { -1 } }
	\l_septatrix_env_tl
	\IfNoValueTF {#1}
	{ \l_septatrix_env_tl }
	{ #1 }
}

\NewDocumentCommand{\ifenvsetTF}{mmm}
{
	\sys_get_shell:nnN { kpsewhich ~ --var-value ~ #1 } { } \l_tmpa_tl
	\tl_if_eq:NNTF \l_tmpa_tl \c_getenv_par_tl { #3 } { #2 }
}

\tl_new:N \l_env_tl
\NewDocumentEnvironment{ifEnvEq}{ m m m +b}{
	\sys_get_shell:nnN { kpsewhich ~ --var-value ~ #1 }
		{ \int_set:Nn \tex_endlinechar:D { -1 } }
		\l_env_tl
	
	\cs_generate_variant:Nn \tl_if_eq:nnTF { o }
	\tl_if_eq:onTF { \l_env_tl } { #2 } { \stepcounter{#3} } { #4 }
}{}
\ExplSyntaxOff


% Optional parameters: Title, Value to blacklist, EnvVar, exercise number
\NewDocumentEnvironment{cExercise}{ O{} O{} O{author} O{\fpeval{\value{section}+1}} +b }{
	\setcounter{section}{\fpeval{#4 - 1}}
	\begin{ifEnvEq}{#3}{#2}{section}
		\begin{exercise}
			#1
		\end{exercise}
		#5
	\end{ifEnvEq}
}{}

% Optional parameters: Title, Value to blacklist, EnvVar, exercise number
\NewDocumentEnvironment{cPart}{ O{} O{} O{author} O{\fpeval{\value{subExercise}+1}} +b }{
	\setcounter{subExercise}{\fpeval{#4 - 1}}
	\begin{ifEnvEq}{#3}{#2}{subExercise}
		\begin{subExercise}
			#1
		\end{subExercise}
		#5
	\end{ifEnvEq}
}{}

% Optional parameters: Title, Value to blacklist, EnvVar, exercise number
\NewDocumentEnvironment{cSubPart}{ O{} O{} O{author} O{\fpeval{\value{subSubExercise}+1}} +b }{
	\setcounter{subSubExercise}{\fpeval{#4 - 1}}
	\begin{ifEnvEq}{#3}{#2}{subSubExercise}
		\begin{subSubExercise}
			#1
		\end{subSubExercise}
		#5
	\end{ifEnvEq}
}{}

% Legacy Environment Start
\newcommand{\ex}[2][\fpeval{\value{section}+1}]{\setcounter{section}{\fpeval{#1 - 1}}
	\begin{exercise}
		#2
	\end{exercise}
}
\newcommand{\sub}[2][\fpeval{\value{subExercise}+1}]{\setcounter{subExercise}{\fpeval{#1 - 1}}
	\begin{subExercise}
		#2
	\end{subExercise}
}
\newcommand{\subb}[2][\fpeval{\value{subSubExercise}+1}]{\setcounter{subSubExercise}{\fpeval{#1 - 1}}
	\begin{subSubExercise}
		#2
	\end{subSubExercise}
}
% Legacy Environment End
\newcommand{\cl}[2]{\mbox{cl}_{#2}\left(#1\right)}
\newcommand{\CB}[1]{\operatorname{CB}\left(#1\right)}
\newcommand{\MR}[1]{\operatorname{MR}\left(#1\right)}
\newcommand{\MD}[1]{\operatorname{MD}\left(#1\right)}
\newcommand{\monster}{{\mathfrak C}}
\newcommand{\tp}[1]{\operatorname{tp}\left(#1\right)}
\newcommand{\hull}[3]{\operatorname{Hull}^{#2}_{#3}\left(#1\right)}
\newcommand{\stp}[1]{\operatorname{stp}\left(#1\right)}
\newcommand{\dom}[1]{\operatorname{dom}\left(#1\right)}
\newcommand{\range}[1]{\operatorname{range}\left(#1\right)}
\newcommand{\cnst}[2]{\operatorname{const}_{#1}\left(#2\right)}
\renewcommand{\c}{{\mathfrak c}}
\newcommand{\club}[1]{\operatorname{club}\left(#1\right)}
\newcommand{\Lev}[1]{\operatorname{Lev}\left(#1\right)}
\newcommand{\height}[1]{\operatorname{ht}\left(#1\right)}
\newcommand{\emptyseq}{\Lambda}

\newcommand{\N}{\mathbb N}
\newcommand{\Q}{\mathbb Q}
\newcommand{\R}{\mathbb R}
\newcommand{\C}{\mathbb C}
\newcommand{\F}{\mathbb F}
\renewcommand{\P}{\mathbb P}
\renewcommand{\Bbb}[1]{\mathbb #1}
\newcommand{\incomp}{\operatorname{\bot}}
\newcommand{\comp}{\operatorname{\|}}
\newcommand{\force}{\Vdash}
\newcommand{\Add}[2]{\operatorname{Add}\left(#1,#2\right)}

\usepackage{xparse}


\ExplSyntaxOn
\NewExpandableDocumentCommand \randint { m m }
{ \int_rand:nn { #1 } { #2 } }
\ExplSyntaxOff
\newcounter{cntTherefore}
\newcommand\setTherefore[2]{%
	\csdef{Therefore#1}{#2}}
\newcommand\addTherefore[1]{%
	\stepcounter{cntTherefore}%
	\csdef{Therefore\thecntTherefore}{#1}}
\newcommand\getTherefore[1]{%
	\csuse{Therefore#1}}
\newcommand{\Therefore}{\getTherefore{\randint{1}{\thecntTherefore}} }
\addTherefore{therefore}
\addTherefore{hence}
\addTherefore{which implies}
\addTherefore{consequently}
\addTherefore{it follows that}




\newcommand{\envOrDefault}[2]{\ifenvsetTF{#1}{\getenv{#1}}{#2}}

\newcommand{\envWithDefault}[1]{\envOrDefault{#1}{Holo}}

\def\bonktxt#1{% 
	\quitvmode\hbox{% 
		\pdfliteral{q 1 0 .15 .4 0 0 cm}\rlap{#1}\pdfliteral{Q}\hphantom{#1}% 
		\pdfliteral{q .9063 .4226 -.4226 .9063 -3 6 cm .2 0 0 .2 0 0 cm}\llap\bonktext\pdfliteral{Q}\ % 
		\pdfliteral{q 
			.81914 .57356 -.57356 .81914 -3 4 cm 
			1.7 0 0 1.7 0 0 cm 1 j 1 J .7 w 
			0 0 m 0 .7 l 2 .7 l 2 0 l b 
			0 j .3 w .2 -.4 m .2 1.1 l s 
			1.8 -.4 m 1.8 1.1 l s 
			.8 1.05 m .85 1.15 1.15 1.15 1.2 1.05 c 
			1 0 m 1 -1.4 l s 
			1.1 -1.4 m 1.2 -2 1.1 -3 y s 
			.9 -1.4 m .8 -2 .9 -3 y s 
			Q}\kern3pt\relax 
	}% 
} 
\def\bonktext{bonk!}
\newcommand{\bonk}[1]{\bonktxt{$#1$}}

\usepackage{datetime}
% \newdate{date}{06}{09}{2012}
% \date{\displaydate{date}}
\date{\today}


\author{\envWithDefault{author}}



\usepackage{skak}
\usepackage{relsize}
\usepackage{graphicx}
\usepackage{mathtools}

\usepackage{textcomp}
\usepackage{bbding}

\usepackage{soul}



\title{Exercise 5}
\begin{document}
\maketitle
\begin{cExercise}[Suslin trees][Yuval Paz][author]
	By definition Suslin trees satisfy c.c.c. 
	
	Let $T$ be a Suslin tree, for each $\alpha\in\omega_1$ let $D_\alpha$ be the set if nodes in $T$ whose rank is greater or equal to $\alpha$ (these sets are dense by the virtue of how we defined them, if we exclude the condition that each node has arbitrarily large extension we just add to $D_\alpha$ all of the leaves of smaller rank), and $\mathfrak{D}$ be the set of all $D_\alpha$, let $G$ be $\mathfrak{D}$-generic for $T$.
	
	Because $T$ is a tree, 2 elements are compatible if and only if they are comparable, in particular $G$ is a chain. Assume it is not maximal, let $t\in T$ be element that is comparable with all of $G$, because $G$ is closed downwards $t$ must be above all of $G$, let $\alpha$ be the rank of $t$ and $t'\in D_{\alpha+1}\cap G$, we have that $t,t'$ are comparable and the rank of $t'$ is greater so $t<t'$, contradiction.
	
	In addition $\{\bigcup G\restriction \alpha\mid \alpha\in\omega_1\}=G$, but it is clearly of cardinality $\aleph_1$, contradiction to the fact $T$ is Suslin.
\end{cExercise}

\begin{cExercise}[Almost disjoint families][][author][2]
	\begin{cPart}
		Let $A_i$ be the powers of the $i^\textbf{th}$ prime, this gives a collection of $\aleph_0$-many disjoint infinite subsets of $\N$.
		
		For each $\alpha\in\omega_1\setminus\omega$ let $f_\alpha:\alpha\to\omega$ be a bijection, assume that $A_\beta$ is defined for each $\beta<\alpha$ and that $A_\beta$ are all infinite and that $\{A_\beta\}$ is almost disjoint family.
		
		Let $\{B_i\}_{i\in\omega}$ be the set $\{A_\beta\}$ mapped to order type $\omega$ using $f_\alpha$.
		
		Let $a_i$ be an element from $B_i\setminus (\bigcup_{j<i}B_j)$, because $B_i$ is infinite for each $i\in\omega$ and $B_i\cap (\bigcup_{j<i}B_j)$ is finite, $a_i$ is well defined for each $i\in\omega$, furthermore for each $i\in\omega$ we have that $\{a_i\}_{i\in\omega}\cap B_j\subseteq \{a_i\}_{i\le j}$ a finite set, so letting $A_\alpha=\{a_i\}_{i\in\omega}$ will work.
	\end{cPart}
	\begin{cPart}
		Given $\vec{a},A$, let $(s,F),(s,F')\in\Q(\vec{a},A)$ we have that $(s,F\cup F')\ge (s,F),(s,F')$ trivially, and because $s\in[\omega]^{<\omega}$ which is countable, for given uncountably many conditions, there will be some elements with the same left-side, and any 2 such elements are compatible, so the forcing notion is c.c.c
		
		Let $a^G$ be as in the question, to see that $a^G$ is infinite we claim that if $(s,F)$ is any condition, then there exists $(s',F)>(s,F)$ such that $|s'|>|s|$, this would imply that $\{(s,F)\mid |s|>|a^G|\}$ is dense set in $M$ that is disjoint from $G$.
		
		Let $(s,F)$ by any condition, note that the set of finite subsets and cofinite subsets of $\omega$ is countable, so there exists $a_\alpha\in\vec{a}$ that is infinite-coinfinite, which has infinite intersection with each cofinite set, so each $a_\alpha$ is coinfinite.
		
		In addition if $\{b_i\}_{i\in I}$ is almost disjoint family and $J\subseteq I$ is finite we have that $\{b_i\}_{i\in I\setminus J}\cup\{\bigcup_{j\in J}b_j\}$ is also almost disjoint family.
		
		Hence we can conclude that $B=\{a_\alpha\}_{\alpha\in \omega_1\setminus F}\cup \{\bigcup_{\alpha\in F}a_\alpha\}$ is almost disjoint, from the previous fact we get that $\bigcup_{\alpha\in F}a_\alpha$ is coinfinite, let $k\in\omega\setminus \bigcup_{\alpha\in F}a_\alpha$ and we get that $(s\cup\{k\},F)>(s,F)$. We had used only $\aleph_0$ many dense sets to show this ($D_n=\{(s,F)\mid |s|>n\}$).
		
		Let $j\notin A$ and assume $a^G\cap a_j$ is finite, similarly to above, we can see that $\{(s,F)\mid |s\cap a_j|>|a^G\cap a_j|\}$ is dense, indeed we saw before that $B$ is almost disjoint, and because $F\subseteq A$ it means that $a_j\cap \bigcup_{\alpha\in F}a_\alpha$ is finite, in particular $a_j\setminus \bigcup_{\alpha\in F}a_\alpha$ is not empty, and we can choose a $k$ from there. We had used $|\omega_1\setminus A|\times\aleph_0\le\aleph_1\times\aleph_0=\aleph_1$ many dense sets for this (to be precise, we can remove the reference to $a^G$ and use the dense sets $D_{j,n}=\{(s,F)\mid |s\cap a_j|>n\}$).
		
		Lastly, let $j\in A$, we want to show that $a^G\cap a_j$ is finite, to do this we will show that there exists some finite $a\subseteq a_j$ such that for each $(s,F)\in G$ we have that $s\cap a_j\subseteq a$ and conclude that $a^G\cap a_j\subseteq a$ is finite.
		
		Indeed, if for every $(s,F)\in G$ we have that $s\cap a_j=\emptyset$ we are done, so let $(s',F')\in G$ be some fixed condition such that $s'\cap a_j\ne\emptyset$.
		
		Because $D^j=\{(s,F)\mid j\in F\}$ is dense (because we can always strengthen $(s,F)$ to $(s,F\cup\{j\})$) there must be some $(z,W)\in G$ with $j\in W$, WLOG assume $(z,W)>(s',F')$ (if not, just take the common strengthening), and let $a=z\cap a_j\supseteq s'\cap a_j\ne\emptyset$.
		
		Assume $(s,F)\in G$ such that there exists $k\in s\cap a_j\setminus a$ and let $(t,Q)$ be common strengthening of $(s,F),(z,W)$. Because $(t,Q)\ge (s,F)$ we must have that $k\in t$, in particular $k\in(t\setminus z)\cap a_j\ne\emptyset$, but $j\in W$, so $(t,Q)\not\ge (z,W)$, contradiction. For this argument we used $|A|\le\aleph_1$ many dense sets.
	\end{cPart}
	\begin{cPart}
		Assume $\sf MA_{\omega_1}$ and fix some $\vec a$ almost disjoint sequence as in the previous parts, for each $A\subseteq \omega_1$ let $\mathfrak{D}_A$ be the set of all dense sets we used in the previous part and let $G_A$ be $\mathfrak{D}_A$-generic for $\mathbb{Q}(\vec a,A)$ (note that $|\mathfrak{D}_A|\le \aleph_0+\aleph_1+\aleph_1=\aleph_1$ and that $\mathbb{Q}(\vec a,A)$ is c.c.c).
		
		Define the function $f:2^{\aleph_1}\to2^{\aleph_0}$ as $f(A)=a^{G_A}$, because $A$ is recoverable from $a^{G_A}$ alone (using $\vec a$ as a parameter, in particular we don't need to know what $G_A$ is), $f$ must be injective, hence $2^{\aleph_1}\le 2^{\aleph_0}$, and the other direction is trivial.
	\end{cPart}
\end{cExercise}



\end{document}




