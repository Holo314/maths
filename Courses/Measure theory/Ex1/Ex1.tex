\documentclass[12pt,reqno]{article}
\usepackage[margin=3cm]{geometry}
\usepackage[utf8]{inputenc}


\usepackage{xfp}
\usepackage{natbib}
\usepackage{graphicx}
\usepackage{amsthm}
\usepackage{amsmath}
\usepackage{amssymb}
\usepackage{cases}
\usepackage{microtype}
\usepackage{hyperref}
\usepackage{mathrsfs}

\newtheorem{theorem}{Theorem}[section]
\newtheorem{proposition}[theorem]{Proposition}
\newtheorem{corollary}[theorem]{Corollary}
\newtheorem{lemma}[theorem]{Lemma}
\newtheorem{conjecture}[theorem]{Conjecture}
\newtheorem{example}[theorem]{Example}
\theoremstyle{definition}
\newtheorem{definition}[theorem]{Definition}
\theoremstyle{remark}
\newtheorem{remark}[theorem]{Remark}
\theoremstyle{exercise}
\newtheorem{exercise}[section]{Exercise}
\theoremstyle{subExercise}
\newtheorem{subExercise}{Part}[section]
\newtheorem{subSubExercise}{Sub Part}[subExercise]
\numberwithin{equation}{section}
\newcommand{\acl}[2]{\operatorname{acl}^{#2}\left(#1\right)}
\newcommand{\rank}[2]{\operatorname{rank}_{#2}\left(#1\right)}
\newcommand{\dcl}[2]{\operatorname{dcl}^{#2}\left(#1\right)}
\newcommand{\Aut}[1]{\operatorname{Aut}\left(#1\right)}
\newcommand{\Av}[1]{\operatorname{Av}\left(#1\right)}
\newcommand{\comment}[2]{#2}
\newcommand{\cof}[1]{\operatorname{cof}\left(#1\right)}
\renewcommand{\cal}[1]{\mathcal{#1}}
\newcommand{\D}[2][{}]{\text{Diag}_{#1}\left(#2\right)}
\renewcommand{\phi}{\varphi}
\newcommand{\code}[1]{\left\lceil#1\right\rceil}
\setcounter{MaxMatrixCols}{20}
\sloppy

\newcommand{\ex}[2][\fpeval{\value{section}+1}]{\setcounter{section}{\fpeval{#1 - 1}}
	\begin{exercise}
		#2
\end{exercise}}
\newcommand{\sub}[2][\fpeval{\value{subExercise}+1}]{\setcounter{subExercise}{\fpeval{#1 - 1}}
	\begin{subExercise}
		#2
\end{subExercise}}
\newcommand{\subb}[2][\fpeval{\value{subSubExercise}+1}]{\setcounter{subSubExercise}{\fpeval{#1 - 1}}
	\begin{subSubExercise}
		#2
\end{subSubExercise}}
\newcommand{\cl}{\mbox{cl}}
\newcommand{\CB}[1]{\operatorname{CB}\left(#1\right)}
\newcommand{\MR}[1]{\operatorname{MR}\left(#1\right)}
\newcommand{\MD}[1]{\operatorname{MD}\left(#1\right)}
\newcommand{\monster}{{\mathfrak C}}
\newcommand{\tp}[1]{\operatorname{tp}\left(#1\right)}
\newcommand{\hull}[3]{\operatorname{Hull}^{#2}_{#3}\left(#1\right)}
\newcommand{\stp}[1]{\operatorname{stp}\left(#1\right)}
\newcommand{\dom}[1]{\operatorname{dom}\left(#1\right)}
\newcommand{\range}[1]{\operatorname{range}\left(#1\right)}
\newcommand{\cnst}[2]{\operatorname{const}_{#1}\left(#2\right)}
\renewcommand{\c}{{\mathfrak c}}
\newcommand{\club}[1]{\operatorname{club}\left(#1\right)}
\newcommand{\Lev}[1]{\operatorname{Lev}\left(#1\right)}
\newcommand{\height}[1]{\operatorname{ht}\left(#1\right)}
\newcommand{\emptyseq}{\Lambda}

\newcommand{\N}{\mathbb N}
\newcommand{\Q}{\mathbb Q}
\newcommand{\R}{\mathbb R}
\newcommand{\C}{\mathbb C}
\newcommand{\F}{\mathbb F}




\usepackage{xparse}

\ExplSyntaxOn
\tl_new:N \l_septatrix_env_tl
\NewDocumentCommand \getenv { o m }
{
	\sys_get_shell:nnN { kpsewhich ~ --var-value ~ #2 }
	{ \int_set:Nn \tex_endlinechar:D { -1 } }
	\l_septatrix_env_tl
	\IfNoValueTF {#1}
	{ \tl_use:N \l_septatrix_env_tl }
	{ \tl_set_eq:NN #1 \l_septatrix_env_tl }
}
\tl_const:Nn \c_getenv_par_tl { \par }

\NewDocumentCommand{\ifenvsetTF}{mmm}
{
	\sys_get_shell:nnN { kpsewhich ~ --var-value ~ #1 } { } \l_tmpa_tl
	\tl_if_eq:NNTF \l_tmpa_tl \c_getenv_par_tl { #3 } { #2 }
}
\ExplSyntaxOff

\usepackage{datetime}
% \newdate{date}{06}{09}{2012}
% \date{\displaydate{date}}
\date{\today}
\newcommand{\envOrDefault}[2]{\ifenvsetTF{#1}{\getenv{#1}}{#2}}

\author{\envOrDefault{au4thor}{Holo}}



\usepackage{skak}
\usepackage{relsize}
\usepackage{graphicx}
\usepackage{mathtools}

\usepackage{textcomp}
\usepackage{bbding}

\usepackage{soul}

\newcommand{\flower}{\text{\scalebox{0.75}{\raisebox{-0.7ex}{
				\rotatebox{90}{\textleaf}\hspace{-0.3em}
				\scalebox{0.7}{\textleaf}\hspace{-1.35em}
				\raisebox{1ex}{\scalebox{0.8}{\FiveFlowerOpen}}
}}}}
\title{Exercise 1}
\begin{document}
\maketitle
\begin{cExercise}
	\begin{cPart}
		We need to show that $\cal A$ is closed under finite unions, it is enough to show that $\cal A$ is closed under union of size $2$.
		
		Let $Z,Y\in\cal A$, then $Z\cup Y=X\setminus((X\setminus Y)\setminus Z)\in\cal A$
	\end{cPart}
	\begin{cPart}
		First let's note that $X\in \cal A_0$ hence $X\in \cal A$.
		
		Now given $Z,Y\in\cal A$ there exists some $n$ such that $Z,Y\in \cal A_n$ hence $Z\setminus Y\in\cal A_n$ and so $Z\setminus Y\in \cal A$ and we are done.
	\end{cPart}
	\begin{cPart}
		\comment{Note that given $A$ a family of sets over $X$, $|\sigma(A)|\le \max(|A|^{\aleph_0},\aleph_0)$. That is because every element of $\sigma(A)$ comes uniquely from a countable amount of sets from $A$ as well as finite amount of operators from $\{\bigcup,\bigcap,\setminus\}$ (where $\bigcup,\bigcap$ are treated as function from $(2^X)^{\N}$ to $2^X$ and $\setminus$ is from $(2^X)^{2}$ to $2^X$).
			
		In particular $|\sigma(A)|\le |A|^{\aleph_0}\times 3^{<\aleph_0}=\max(|A|^{\aleph_0},\aleph_0)$.}{}
		Note that given a set $X$ and a finite family $F$ over $X$, $\sigma(F)$ is finite.
		
		So let $X=\N$, and recursively define:
		
		\begin{itemize}
			\item $\cal A_0=\sigma(\emptyset)$
			\item $\cal A_{n+1}=\sigma(\cal A_n\cup\{n\})$
			\item $\cal A=\bigcup \cal A_n$
		\end{itemize}
		Because $\cal A$ contains all of the singletons, if it was a $\sigma$-algebra it would contains the set of even numbers, but clearly no $\cal A_n$ contains that set, so $\cal A$ is not a $\sigma$-algebra
	\end{cPart}
\end{cExercise}
\begin{cExercise}
	\begin{cPart}
		For every $x\in U$ let $I_x$ be an interval containing $x$ and a subset of $U$ (such interval exists by definition).
		
		Given $x,y\in U$, we say $x\sim y$ if there exists $\{x,y\}\subseteq J\subseteq U$ such that $\bigcup_{z\in J} I_z$ is an interval.
		
		This is clearly a equivalence relation, furthermore $\forall x\forall z\in I_x\;(x\sim z)$. Let $I$ be an equivalent class, and $z\in I$, then $I_z\subseteq I$, hence $I$ is open.
		
		Furthermore, if $x\sim y$ and $J\subseteq U$ witness that, then $x\sim w$ for every $w$ in $\bigcup_{z\in J} I_z$.
		
		If $I$ is not an interval it means that there exists $a<b<c$ such that $a,c\in I$ but $b\notin I$, but that is impossible, from the previous sentence.
	\end{cPart}
	\begin{cPart}
		Every family of disjoint intervals in $\R$ is at most countable, indeed every interval contains some rational number and there are only countably many rational numbers.
	\end{cPart}
	\begin{cPart}
		Let $\cal I$ be the collection of open intervals and $\tau$ the collection of open sets, then from the previous 2 parts we have that 
		$$\cal I\subseteq\tau\subseteq \sigma(\cal I)\implies \sigma(\cal I)\subseteq\sigma(\tau)=\cal B(\R)\subseteq \sigma(\sigma(\cal I))=\sigma(\cal I)$$
	\end{cPart}
\end{cExercise}
\begin{cExercise}
	$(a)\implies (b)$: trivial.
	
	$(b)\implies (c)$: assume $(b)$, and let $(A_n)$ be an $\subseteq$-increasing sequence of sets in $\cal M$, then $B_n=A_n\setminus(\bigcup_{k<n}A_k)$ is a sequence of disjoint subsets in $\cal M$ (it is in $\cal M$ because $\cal M$ is an algebra), from the assumption $\bigcup A_n=\bigcup B_n\in \cal M$
	
	$(c)\implies (a)$: assume $(c)$ and let $(A_n)$ be arbitrary sequence of elements from $\cal M$, then $B_n=\bigcup_{k<n}A_k$ is an increasing sequence of elements from $\cal M$ (it is in $\cal M$ because $\cal M$ is an algebra), from the assumption $\bigcup A_n=\bigcup B_n\in \cal M$
	
\end{cExercise}
\begin{cExercise}
	Given $Z,Y\in \cal M_1$, then $Z=f^{-1}(r), Y=f^{-1}(t)$ for $r,t\in\cal M_2$.
	
	\begin{itemize}
		\item $X_1=f^{-1}(X_2)\in \cal M_1$
		\item $Z\setminus Y=f^{-1}(r)\setminus f^{-1}(t)=f^{-1}(r\setminus t)\in\cal M_1$.
		\item Let $(A_n)=(f^{-1}(B_n))$ be a sequence of elements from $\cal M_1$, then $\bigcup A_n=f^{-1}(\bigcup B_n)\in\cal M_1$
	\end{itemize}
\end{cExercise}
\begin{cExercise}
	\begin{cPart}
		Let $A$ be the set of atoms in $\cal M$ (the set of non-empty sets that are $\subseteq$-minimal).
		
		Every element in $\cal M$ is either disjoint or a superset of every atom, in particular all of the atoms are disjoint.
		
		If $A$ is infinite we are done. Otherwise $X\setminus \bigcup A\in \cal M$. Because $\cal M$ is infinite and $2^A$ is finite, then $X\setminus \bigcup A\ne\emptyset$.
		
		Define the following binary tree:
		
		\begin{itemize}
			\item Define $T_{\Lambda}=X\setminus \bigcup A$ ($\Lambda$ is the empty sequence)
			\item Given a finite binary sequence $\tau$ for which $T_\tau$ is already defined, let $\emptyset\ne A\subsetneq T_\tau$ be an element of $\cal M$ (it exists because $T_\tau$ is not an atom), then define $T_{\tau\frown\{0\}}=A,T_{\tau\frown\{1\}}=T_\tau\setminus A$
			\item Define $T$ be the tree $(\{T_\tau\mid \tau\in 2^{<\N}\},\subseteq)$ 
		\end{itemize}
		
		Notice that given $\tau,\sigma\in 2^{<\N}$, if $\tau$ is not an initial segment of $\sigma$ and vice versa, then $T_\tau\cap T_\sigma=\emptyset$.
		
		So $\{T_\tau\mid \tau\in 1^{<\N}\times\{1\}\}$ ($1^{<\N}\times\{1\}=\{1, 01, 001, 0001,\ldots\}$) is an infinite set of disjoint elements from $\cal M$.
	\end{cPart}
	\begin{cPart}
		If $A$ is infinite set of disjoint elements from $\cal M$ then it contains a countable infinite subset $B$, so $\{\bigcup J\mid J\subseteq B\}\subseteq \cal M$ but the former has cardinality $2^{\aleph_0}=\frak c>\aleph_0$
	\end{cPart}
\end{cExercise}

\end{document}




