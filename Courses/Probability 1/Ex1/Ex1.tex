\documentclass[12pt,reqno]{article}
\usepackage[margin=3cm]{geometry}
\usepackage[utf8]{inputenc}


\usepackage{xfp}
\usepackage{natbib}
\usepackage{graphicx}
\usepackage{amsthm}
\usepackage{amsmath}
\usepackage{amssymb}
\usepackage{cases}
\usepackage{microtype}
\usepackage{hyperref}
\usepackage{mathrsfs}

\newtheorem{theorem}{Theorem}[section]
\newtheorem{proposition}[theorem]{Proposition}
\newtheorem{corollary}[theorem]{Corollary}
\newtheorem{lemma}[theorem]{Lemma}
\newtheorem{conjecture}[theorem]{Conjecture}
\newtheorem{example}[theorem]{Example}
\theoremstyle{definition}
\newtheorem{definition}[theorem]{Definition}
\theoremstyle{remark}
\newtheorem{remark}[theorem]{Remark}
\theoremstyle{exercise}
\newtheorem{exercise}[section]{Exercise}
\theoremstyle{subExercise}
\newtheorem{subExercise}{Part}[section]
\newtheorem{subSubExercise}{Sub Part}[subExercise]
\numberwithin{equation}{section}
\newcommand{\acl}[2]{\operatorname{acl}^{#2}\left(#1\right)}
\newcommand{\rank}[2]{\operatorname{rank}_{#2}\left(#1\right)}
\newcommand{\dcl}[2]{\operatorname{dcl}^{#2}\left(#1\right)}
\newcommand{\Aut}[1]{\operatorname{Aut}\left(#1\right)}
\newcommand{\Av}[1]{\operatorname{Av}\left(#1\right)}
\newcommand{\comment}[2]{#2}
\newcommand{\cof}[1]{\operatorname{cof}\left(#1\right)}
\renewcommand{\cal}[1]{\mathcal{#1}}
\newcommand{\D}[2][{}]{\text{Diag}_{#1}\left(#2\right)}
\renewcommand{\phi}{\varphi}
\newcommand{\code}[1]{\left\lceil#1\right\rceil}
\setcounter{MaxMatrixCols}{20}
\sloppy

\newcommand{\ex}[2][\fpeval{\value{section}+1}]{\setcounter{section}{\fpeval{#1 - 1}}
	\begin{exercise}
		#2
\end{exercise}}
\newcommand{\sub}[2][\fpeval{\value{subExercise}+1}]{\setcounter{subExercise}{\fpeval{#1 - 1}}
	\begin{subExercise}
		#2
\end{subExercise}}
\newcommand{\subb}[2][\fpeval{\value{subSubExercise}+1}]{\setcounter{subSubExercise}{\fpeval{#1 - 1}}
	\begin{subSubExercise}
		#2
\end{subSubExercise}}
\newcommand{\cl}{\mbox{cl}}
\newcommand{\CB}[1]{\operatorname{CB}\left(#1\right)}
\newcommand{\MR}[1]{\operatorname{MR}\left(#1\right)}
\newcommand{\MD}[1]{\operatorname{MD}\left(#1\right)}
\newcommand{\monster}{{\mathfrak C}}
\newcommand{\tp}[1]{\operatorname{tp}\left(#1\right)}
\newcommand{\hull}[3]{\operatorname{Hull}^{#2}_{#3}\left(#1\right)}
\newcommand{\stp}[1]{\operatorname{stp}\left(#1\right)}
\newcommand{\dom}[1]{\operatorname{dom}\left(#1\right)}
\newcommand{\range}[1]{\operatorname{range}\left(#1\right)}
\newcommand{\cnst}[2]{\operatorname{const}_{#1}\left(#2\right)}
\renewcommand{\c}{{\mathfrak c}}
\newcommand{\club}[1]{\operatorname{club}\left(#1\right)}
\newcommand{\Lev}[1]{\operatorname{Lev}\left(#1\right)}
\newcommand{\height}[1]{\operatorname{ht}\left(#1\right)}
\newcommand{\emptyseq}{\Lambda}

\newcommand{\N}{\mathbb N}
\newcommand{\Q}{\mathbb Q}
\newcommand{\R}{\mathbb R}
\newcommand{\C}{\mathbb C}
\newcommand{\F}{\mathbb F}




\usepackage{xparse}

\ExplSyntaxOn
\tl_new:N \l_septatrix_env_tl
\NewDocumentCommand \getenv { o m }
{
	\sys_get_shell:nnN { kpsewhich ~ --var-value ~ #2 }
	{ \int_set:Nn \tex_endlinechar:D { -1 } }
	\l_septatrix_env_tl
	\IfNoValueTF {#1}
	{ \tl_use:N \l_septatrix_env_tl }
	{ \tl_set_eq:NN #1 \l_septatrix_env_tl }
}
\tl_const:Nn \c_getenv_par_tl { \par }

\NewDocumentCommand{\ifenvsetTF}{mmm}
{
	\sys_get_shell:nnN { kpsewhich ~ --var-value ~ #1 } { } \l_tmpa_tl
	\tl_if_eq:NNTF \l_tmpa_tl \c_getenv_par_tl { #3 } { #2 }
}
\ExplSyntaxOff

\usepackage{datetime}
% \newdate{date}{06}{09}{2012}
% \date{\displaydate{date}}
\date{\today}
\newcommand{\envOrDefault}[2]{\ifenvsetTF{#1}{\getenv{#1}}{#2}}

\author{\envOrDefault{au4thor}{Holo}}



\usepackage{skak}
\usepackage{relsize}
\usepackage{graphicx}
\usepackage{mathtools}

\usepackage{textcomp}
\usepackage{bbding}

\usepackage{soul}

\newcommand{\flower}{\text{\scalebox{0.75}{\raisebox{-0.7ex}{
				\rotatebox{90}{\textleaf}\hspace{-0.3em}
				\scalebox{0.7}{\textleaf}\hspace{-1.35em}
				\raisebox{1ex}{\scalebox{0.8}{\FiveFlowerOpen}}
}}}}
\title{Exercise 1}
\begin{document}
\maketitle
\begin{cExercise}[][Yuval Paz][author][1]
	\begin{cPart}[][Yuval Paz][author]
		
		Let $A\subseteq \Omega$ with probability $0$, and $B\subseteq \Omega$ any event with some probability $\alpha$.
		
		Let $B'=B\setminus A$, then $\P(A\cup B)=\P(A\cup B')=\P(A)+\P(B')=\P(B')$.
		
		We also have $\P(B)=\P((B\cap A)\cup B')=\P(B\cap A)+\P(B')$
		
		From 1.2, $\P(B\cap A)\le\P(A)\implies \P(B\cap A)=0\implies \P(B)=\P(B')$
		
	\end{cPart}
	\begin{cPart}[][Yuval Paz][author]
		Let $A\subseteq B$, and let $B'=B\setminus A$, then $\P(B)=\P(A\cup B')=\P(A)+\P(B)\ge \P(A)$ as $\P(B)\ge 0$
		
	\end{cPart}
	\begin{cPart}[][Yuval Paz][author]
		Let $(\Omega, \P)$ be any discrete probability space, let $\flower\notin \Omega$, and define $(\Omega\cup\{\flower\}, \P^{\flower})$ be discrete probability space defined as: $\P^{\flower}(A)=\P(A\setminus\{\flower\})$.
		
		Clearly this is a probability space ($\P^{\flower}(\Omega\cup\{\flower\})=\P(\Omega)=1$, and given any countable set of disjointed subsets of $\Omega\cup\{\flower\}$, at most one of them contains $\flower$, removing the flower from this specific set and looking at the $\sigma$-additivity of $\P$ gives the result)
		
		It is also discrete, as if $p$ is a discrete probability function inducing $\P$, then $p^{\flower}$ defined as $p$ on $\Omega$ and $p(\flower)=0$ will induce $\P^{\flower}$.
		
		In this probability space, let $A\subseteq\Omega$, then $A\subsetneq A\cup\{\flower\}$ but $\P^{\flower}(A)=\P^{\flower}(A\cup\{\flower\})$
	\end{cPart}
	\begin{cPart}[][Yuval Paz][author]
		If $\P(A\cap B)=\alpha\in[0,1]$, the only way for the inequality to fail is for $\P(A)+\P(B)>1+\alpha$
		
		Now let $A',B'$ defined as in 1.1 and 1.2, then we have $\P(A)=\P(A')+\alpha\le 1\implies \P(A)\le 1-\alpha$, and similarly for $B$ so $\P(A)+\P(B)=\P(A')+\P(B')+2\alpha>1+\alpha\implies \P(A')+\P(B')+\alpha>1$, but by the definition $\alpha=\P(A\cap B)$, and $A',B',A\cap B$ are all disjoints, so we get that $\P(A'\cup B'\cup (A\cap B))>1$, contradiction.
		
	\end{cPart}
	\begin{cPart}[][Yuval Paz][author]
		Let $A',B'$ be as defined in 1.1 and 1.2.
		
		We have $\P(A)=\P(A')+\P(A\cap B)$ and $\P(B)=\P(B')+\P(A\cap B)$
		
		Notice that $B'\cap A'=\emptyset$, so adding the 2 equations we get $\P(A)+\P(B)=\P(A'\cup B')+2\P(A\cap B)\implies \P(A)+\P(B)-2\P(A\cap B)=\P(A'\cup B')$
		
		But $A'\cup B'$ is exactly $A\Delta B$, so we are done.
	\end{cPart}
\end{cExercise}
\begin{cExercise}[][Yuval Paz][author]
	Let $\P$ be a probability function satisfying the conditions in the question.
	
	Because $\N$ is countable, so every subset of $\N$, so $A=\cup_{n\in A}\{n\}$ is countable union of disjoint sets, hence $\P(A)=\sum_{n\in A}\P(\{n\})$, hence it is enough to show that there is a unique discrete probability function $p$ on $\N$ satisfying $p(n)=3p(n+1)$.
	
	Notice that given 2 such discrete probability functions that agree on a single number must be equal.
	
	Let $p(0)=\alpha$, by definition of discrete probability function we have $\sum_{n\in\N}\alpha/3^n=\alpha\cdot \sum_{n\in\N}1/3^n=1\implies \alpha=\frac1{\sum_{n\in\N}1/3^n}$, hence any 2 discrete probability functions satisfying $p(n)=3p(n+1)$ must have the same value at $0$, but this implies that they are equal.
	
	$\P(\N)$ must be $1$, as $\P$ is a probability function, and (assuming $3\N$ means $\{3n\mid n\in\N\}$) $\P(3\N)=\sum_{n\in\N}\alpha/3^{3n}$
\end{cExercise}


\end{document}




