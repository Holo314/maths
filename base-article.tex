\documentclass[12pt,reqno]{article}
\usepackage[margin=3cm]{geometry}
\usepackage[utf8]{inputenc}


\usepackage{xfp}
\usepackage{natbib}
\usepackage{graphicx}
\usepackage{amsthm}
\usepackage{amsmath}
\usepackage{amssymb}
\usepackage{cases}
\usepackage{microtype}
\usepackage{hyperref}
\usepackage{mathrsfs}
\usepackage{xparse}

\makeatletter
\DeclareRobustCommand\widecheck[1]{{\mathpalette\@widecheck{#1}}}
\def\@widecheck#1#2{%
	\setbox\z@\hbox{\m@th$#1#2$}%
	\setbox\tw@\hbox{\m@th$#1%
		\widehat{%
			\vrule\@width\z@\@height\ht\z@
			\vrule\@height\z@\@width\wd\z@}$}%
	\dp\tw@-\ht\z@
	\@tempdima\ht\z@ \advance\@tempdima2\ht\tw@ \divide\@tempdima\thr@@
	\setbox\tw@\hbox{%
		\raise\@tempdima\hbox{\scalebox{1}[-1]{\lower\@tempdima\box
				\tw@}}}%
	{\ooalign{\box\tw@ \cr \box\z@}}}
\makeatother

\newtheorem{theorem}{Theorem}[section]
\newtheorem{proposition}[theorem]{Proposition}
\newtheorem{corollary}[theorem]{Corollary}
\newtheorem{lemma}[theorem]{Lemma}
\newtheorem{conjecture}[theorem]{Conjecture}
\newtheorem{example}[theorem]{Example}
\theoremstyle{definition}
\newtheorem{definition}[theorem]{Definition}
\theoremstyle{remark}
\newtheorem{remark}[theorem]{Remark}
\theoremstyle{exercise}
\newtheorem{exercise}[section]{Exercise}
\theoremstyle{subExercise}
\newtheorem{subExercise}{Part}[section]
\newtheorem{subSubExercise}{Sub Part}[subExercise]
\numberwithin{equation}{section}
\newcommand{\acl}[2]{\operatorname{acl}^{#2}\left(#1\right)}
\newcommand{\rank}[2]{\operatorname{rank}_{#2}\left(#1\right)}
\newcommand{\dcl}[2]{\operatorname{dcl}^{#2}\left(#1\right)}
\newcommand{\Aut}[1]{\operatorname{Aut}\left(#1\right)}
\newcommand{\Av}[1]{\operatorname{Av}\left(#1\right)}
\newcommand{\comment}[2]{#2}
\newcommand{\cof}[1]{\operatorname{cof}\left(#1\right)}
\renewcommand{\cal}[1]{\mathcal{#1}}
\newcommand{\D}[2][{}]{\text{Diag}_{#1}\left(#2\right)}
\renewcommand{\phi}{\varphi}
\newcommand{\code}[1]{\left\lceil#1\right\rceil}
\setcounter{MaxMatrixCols}{20}
\sloppy

\usepackage{expl3}




\ExplSyntaxOn

\tl_new:N \l_septatrix_env_tl
\NewDocumentCommand \getenv { o m }
{
	\sys_get_shell:nnN { kpsewhich ~ --var-value ~ #2 }
	{ \int_set:Nn \tex_endlinechar:D { -1 } }
	\l_septatrix_env_tl
	\IfNoValueTF {#1}
	{ \l_septatrix_env_tl }
	{ #1 }
}

\NewDocumentCommand{\ifenvsetTF}{mmm}
{
	\sys_get_shell:nnN { kpsewhich ~ --var-value ~ #1 } { } \l_tmpa_tl
	\tl_if_eq:NNTF \l_tmpa_tl \c_getenv_par_tl { #3 } { #2 }
}

\tl_new:N \l_env_tl
\NewDocumentEnvironment{ifEnvEq}{ m m m +b}{
	\sys_get_shell:nnN { kpsewhich ~ --var-value ~ #1 }
		{ \int_set:Nn \tex_endlinechar:D { -1 } }
		\l_env_tl
	
	\cs_generate_variant:Nn \tl_if_eq:nnTF { o }
	\tl_if_eq:onTF { \l_env_tl } { #2 } { \stepcounter{#3} } { #4 }
}{}
\ExplSyntaxOff


% Optional parameters: Title, Value to blacklist, EnvVar, exercise number
\NewDocumentEnvironment{cExercise}{ O{} O{} O{author} O{\fpeval{\value{section}+1}} +b }{
	\setcounter{section}{\fpeval{#4 - 1}}
	\begin{ifEnvEq}{#3}{#2}{section}
		\begin{exercise}
			#1
		\end{exercise}
		#5
	\end{ifEnvEq}
}{}

% Optional parameters: Title, Value to blacklist, EnvVar, exercise number
\NewDocumentEnvironment{cPart}{ O{} O{} O{author} O{\fpeval{\value{subExercise}+1}} +b }{
	\setcounter{subExercise}{\fpeval{#4 - 1}}
	\begin{ifEnvEq}{#3}{#2}{subExercise}
		\begin{subExercise}
			#1
		\end{subExercise}
		#5
	\end{ifEnvEq}
}{}

% Optional parameters: Title, Value to blacklist, EnvVar, exercise number
\NewDocumentEnvironment{cSubPart}{ O{} O{} O{author} O{\fpeval{\value{subSubExercise}+1}} +b }{
	\setcounter{subSubExercise}{\fpeval{#4 - 1}}
	\begin{ifEnvEq}{#3}{#2}{subSubExercise}
		\begin{subSubExercise}
			#1
		\end{subSubExercise}
		#5
	\end{ifEnvEq}
}{}

% Legacy Environment Start
\newcommand{\ex}[2][\fpeval{\value{section}+1}]{\setcounter{section}{\fpeval{#1 - 1}}
	\begin{exercise}
		#2
	\end{exercise}
}
\newcommand{\sub}[2][\fpeval{\value{subExercise}+1}]{\setcounter{subExercise}{\fpeval{#1 - 1}}
	\begin{subExercise}
		#2
	\end{subExercise}
}
\newcommand{\subb}[2][\fpeval{\value{subSubExercise}+1}]{\setcounter{subSubExercise}{\fpeval{#1 - 1}}
	\begin{subSubExercise}
		#2
	\end{subSubExercise}
}
% Legacy Environment End
\newcommand{\cl}[2]{\mbox{cl}_{#2}\left(#1\right)}
\newcommand{\CB}[1]{\operatorname{CB}\left(#1\right)}
\newcommand{\MR}[1]{\operatorname{MR}\left(#1\right)}
\newcommand{\MD}[1]{\operatorname{MD}\left(#1\right)}
\newcommand{\monster}{{\mathfrak C}}
\newcommand{\tp}[1]{\operatorname{tp}\left(#1\right)}
\newcommand{\hull}[3]{\operatorname{Hull}^{#2}_{#3}\left(#1\right)}
\newcommand{\stp}[1]{\operatorname{stp}\left(#1\right)}
\newcommand{\dom}[1]{\operatorname{dom}\left(#1\right)}
\newcommand{\range}[1]{\operatorname{range}\left(#1\right)}
\newcommand{\cnst}[2]{\operatorname{const}_{#1}\left(#2\right)}
\renewcommand{\c}{{\mathfrak c}}
\newcommand{\club}[1]{\operatorname{club}\left(#1\right)}
\newcommand{\Lev}[1]{\operatorname{Lev}\left(#1\right)}
\newcommand{\height}[1]{\operatorname{ht}\left(#1\right)}
\newcommand{\emptyseq}{\Lambda}

\newcommand{\N}{\mathbb N}
\newcommand{\Q}{\mathbb Q}
\newcommand{\R}{\mathbb R}
\newcommand{\C}{\mathbb C}
\newcommand{\F}{\mathbb F}
\renewcommand{\P}{\mathbb P}
\renewcommand{\Bbb}[1]{\mathbb #1}
\newcommand{\incomp}{\operatorname{\bot}}
\newcommand{\comp}{\operatorname{\|}}
\newcommand{\force}{\Vdash}
\newcommand{\Add}[2]{\operatorname{Add}\left(#1,#2\right)}

\usepackage{xparse}


\ExplSyntaxOn
\NewExpandableDocumentCommand \randint { m m }
{ \int_rand:nn { #1 } { #2 } }
\ExplSyntaxOff
\newcounter{cntTherefore}
\newcommand\setTherefore[2]{%
	\csdef{Therefore#1}{#2}}
\newcommand\addTherefore[1]{%
	\stepcounter{cntTherefore}%
	\csdef{Therefore\thecntTherefore}{#1}}
\newcommand\getTherefore[1]{%
	\csuse{Therefore#1}}
\newcommand{\Therefore}{\getTherefore{\randint{1}{\thecntTherefore}} }
\addTherefore{therefore}
\addTherefore{hence}
\addTherefore{which implies}
\addTherefore{consequently}
\addTherefore{it follows that}




\newcommand{\envOrDefault}[2]{\ifenvsetTF{#1}{\getenv{#1}}{#2}}

\newcommand{\envWithDefault}[1]{\envOrDefault{#1}{Holo}}

\def\bonktxt#1{% 
	\quitvmode\hbox{% 
		\pdfliteral{q 1 0 .15 .4 0 0 cm}\rlap{#1}\pdfliteral{Q}\hphantom{#1}% 
		\pdfliteral{q .9063 .4226 -.4226 .9063 -3 6 cm .2 0 0 .2 0 0 cm}\llap\bonktext\pdfliteral{Q}\ % 
		\pdfliteral{q 
			.81914 .57356 -.57356 .81914 -3 4 cm 
			1.7 0 0 1.7 0 0 cm 1 j 1 J .7 w 
			0 0 m 0 .7 l 2 .7 l 2 0 l b 
			0 j .3 w .2 -.4 m .2 1.1 l s 
			1.8 -.4 m 1.8 1.1 l s 
			.8 1.05 m .85 1.15 1.15 1.15 1.2 1.05 c 
			1 0 m 1 -1.4 l s 
			1.1 -1.4 m 1.2 -2 1.1 -3 y s 
			.9 -1.4 m .8 -2 .9 -3 y s 
			Q}\kern3pt\relax 
	}% 
} 
\def\bonktext{bonk!}
\newcommand{\bonk}[1]{\bonktxt{$#1$}}

\usepackage{datetime}
% \newdate{date}{06}{09}{2012}
% \date{\displaydate{date}}
\date{\today}


\author{\envWithDefault{author}}

