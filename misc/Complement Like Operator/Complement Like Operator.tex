\documentclass[12pt,reqno]{article}
\usepackage[margin=3cm]{geometry}
\usepackage[utf8]{inputenc}


\usepackage{xfp}
\usepackage{natbib}
\usepackage{graphicx}
\usepackage{amsthm}
\usepackage{amsmath}
\usepackage{amssymb}
\usepackage{cases}
\usepackage{microtype}
\usepackage{hyperref}
\usepackage{mathrsfs}

\newtheorem{theorem}{Theorem}[section]
\newtheorem{proposition}[theorem]{Proposition}
\newtheorem{corollary}[theorem]{Corollary}
\newtheorem{lemma}[theorem]{Lemma}
\newtheorem{conjecture}[theorem]{Conjecture}
\newtheorem{example}[theorem]{Example}
\theoremstyle{definition}
\newtheorem{definition}[theorem]{Definition}
\theoremstyle{remark}
\newtheorem{remark}[theorem]{Remark}
\theoremstyle{exercise}
\newtheorem{exercise}[section]{Exercise}
\theoremstyle{subExercise}
\newtheorem{subExercise}{Part}[section]
\newtheorem{subSubExercise}{Sub Part}[subExercise]
\numberwithin{equation}{section}
\newcommand{\acl}[2]{\operatorname{acl}^{#2}\left(#1\right)}
\newcommand{\rank}[2]{\operatorname{rank}_{#2}\left(#1\right)}
\newcommand{\dcl}[2]{\operatorname{dcl}^{#2}\left(#1\right)}
\newcommand{\Aut}[1]{\operatorname{Aut}\left(#1\right)}
\newcommand{\Av}[1]{\operatorname{Av}\left(#1\right)}
\newcommand{\comment}[2]{#2}
\newcommand{\cof}[1]{\operatorname{cof}\left(#1\right)}
\renewcommand{\cal}[1]{\mathcal{#1}}
\newcommand{\D}[2][{}]{\text{Diag}_{#1}\left(#2\right)}
\renewcommand{\phi}{\varphi}
\newcommand{\code}[1]{\left\lceil#1\right\rceil}
\setcounter{MaxMatrixCols}{20}
\sloppy

\newcommand{\ex}[2][\fpeval{\value{section}+1}]{\setcounter{section}{\fpeval{#1 - 1}}
	\begin{exercise}
		#2
\end{exercise}}
\newcommand{\sub}[2][\fpeval{\value{subExercise}+1}]{\setcounter{subExercise}{\fpeval{#1 - 1}}
	\begin{subExercise}
		#2
\end{subExercise}}
\newcommand{\subb}[2][\fpeval{\value{subSubExercise}+1}]{\setcounter{subSubExercise}{\fpeval{#1 - 1}}
	\begin{subSubExercise}
		#2
\end{subSubExercise}}
\newcommand{\cl}{\mbox{cl}}
\newcommand{\CB}[1]{\operatorname{CB}\left(#1\right)}
\newcommand{\MR}[1]{\operatorname{MR}\left(#1\right)}
\newcommand{\MD}[1]{\operatorname{MD}\left(#1\right)}
\newcommand{\monster}{{\mathfrak C}}
\newcommand{\tp}[1]{\operatorname{tp}\left(#1\right)}
\newcommand{\hull}[3]{\operatorname{Hull}^{#2}_{#3}\left(#1\right)}
\newcommand{\stp}[1]{\operatorname{stp}\left(#1\right)}
\newcommand{\dom}[1]{\operatorname{dom}\left(#1\right)}
\newcommand{\range}[1]{\operatorname{range}\left(#1\right)}
\newcommand{\cnst}[2]{\operatorname{const}_{#1}\left(#2\right)}
\renewcommand{\c}{{\mathfrak c}}
\newcommand{\club}[1]{\operatorname{club}\left(#1\right)}
\newcommand{\Lev}[1]{\operatorname{Lev}\left(#1\right)}
\newcommand{\height}[1]{\operatorname{ht}\left(#1\right)}
\newcommand{\emptyseq}{\Lambda}

\newcommand{\N}{\mathbb N}
\newcommand{\Q}{\mathbb Q}
\newcommand{\R}{\mathbb R}
\newcommand{\C}{\mathbb C}
\newcommand{\F}{\mathbb F}




\usepackage{xparse}

\ExplSyntaxOn
\tl_new:N \l_septatrix_env_tl
\NewDocumentCommand \getenv { o m }
{
	\sys_get_shell:nnN { kpsewhich ~ --var-value ~ #2 }
	{ \int_set:Nn \tex_endlinechar:D { -1 } }
	\l_septatrix_env_tl
	\IfNoValueTF {#1}
	{ \tl_use:N \l_septatrix_env_tl }
	{ \tl_set_eq:NN #1 \l_septatrix_env_tl }
}
\tl_const:Nn \c_getenv_par_tl { \par }

\NewDocumentCommand{\ifenvsetTF}{mmm}
{
	\sys_get_shell:nnN { kpsewhich ~ --var-value ~ #1 } { } \l_tmpa_tl
	\tl_if_eq:NNTF \l_tmpa_tl \c_getenv_par_tl { #3 } { #2 }
}
\ExplSyntaxOff

\usepackage{datetime}
% \newdate{date}{06}{09}{2012}
% \date{\displaydate{date}}
\date{\today}
\newcommand{\envOrDefault}[2]{\ifenvsetTF{#1}{\getenv{#1}}{#2}}

\author{\envOrDefault{au4thor}{Holo}}



\newcommand{\clf}[2]{\mbox{clf}_{#2}\left(#1\right)}
\newcommand{\ICI}[1]{\mathcal{ICI}\left(#1\right)}
\title{Complement Like Operator}
\begin{document}
	
	\maketitle
	\section{Introduction}
	Given a set $X$, we can uniquely identify the complement operator $^c:\mathcal{P}(X)\to\mathcal{P}(X)$ using 3 properties:
	\begin{enumerate}
		\item{ for all $a\subseteq X$ we have $a^c\cap a=\emptyset$}
		\item{ for all $a\subseteq X$ we have $a^c\cup a=X$}
		\item{ for all $a\subseteq X$ we have $(a^c)^c=a$}
	\end{enumerate}
	We wish to explore "complement like operators", an operator $^*:\mathcal{P}(X)\to\mathcal{P}(X)$ that satisfy only 2 out of those 3 properties:
	
	
	\begin{definition}
		$\blacksquare$-complement is an operator $\mathcal{P}(X)\to\mathcal{P}(X)$ that satisfy only 1 and 2
	\end{definition}
	\begin{definition}
		$\bullet$-complement is an operator $\mathcal{P}(X)\to\mathcal{P}(X)$ that satisfy only 1 and 3
	\end{definition}
	\begin{definition}
		$\star$-complement is an operator $\mathcal{P}(X)\to\mathcal{P}(X)$ that satisfy only 2 and 3
	\end{definition}
	
	As it turns out, there are no $\blacksquare$ operators, so in fact property 1 and 2 alone are enough to identify the complement operator.
	\begin{lemma}\label{lem:1.4}
		There are no $\blacksquare$ operator
	\end{lemma}
	\begin{proof}
		Let $^*$ be $\blacksquare$ operator, By property 1 we have $x\in a^*\subseteq X$ implies $x\notin a$, so $a^*\subseteq a^c$.\\
		Similarly by property 2 we have $x\notin a$ implies $x\in a^*$ so $a^c\subseteq a^*$, hence $a^c=a^*$.
	\end{proof}
	
	So now we only need to consider $\star$ and $\bullet$ operators. The following theorems will justify only considering one of those 2 operators:
	
	\begin{theorem}\label{thm:1.5}
		There is a canonical bijection between the set of all $\bullet$ operators and the set of all $\star$ operators on $X$.
	\end{theorem}
	\begin{proof}
		Let $^*$ be a $\star$ operator, then $^{c*c}$ is a $\bullet$ operator.
		
		
		 Indeed $(A^{c*c})^{c*c}=A^{c*cc*c}=A^{c**c}=A^{cc}=A$ and $x\in A^{c*c}\implies x \notin A^{c*}\implies x \notin A^{cc}=A\implies A\cap A^{c*c}=\emptyset$.
		
		Moreover, this transformation is the inverse of itself: $^{c(c*c)c}=^*$	
	\end{proof}
	From here on we will only consider $\star$-complement operators.
	
	\section{Properties of $\star$-complement operator}
	One can ask, is there exists a $\star$-complement operator? i.e. does there exists a set $X$ with $^*:\mathcal{P}(X)\to\mathcal{P}(X)$ such that $a^*\cup a=X$ and $a^*{}^*=a$ for all $a\subseteq X$ and $^*$ is not the complement operator?
	
	For a Dedekind infinite $X$ we can construct such operator pretty explicitly.
	
	\begin{definition}
		A set $X$ is Dedekind finite set if whenever $Y\subsetneq X$, $|Y|<|X|$.\\
		If $X$ is not Dedekind finite, we call it Dedekind-infinite.
	\end{definition}
	$X$ is Dedekind-infinite if and only if it contains countably infinite subset, the $\impliedby$ direction is trivial, for the other direction, take a bijection $f:X\to X\setminus\{*\}$, and look at the subset $\{f^n(*)\}_{n\in\omega}$ to find an infinite countable subset of $X$.
	
	\begin{lemma}\label{lem:2.2}
		If $X$ is Dedekind finite, then there is a $\star$-complement operator on $X$
	\end{lemma}
	\begin{proof}
		Because $X$ contains a countable infinite subset, we can assume $\mathbb Z\subseteq X$, for $n\in\mathbb Z$ define $A_n=\{x\in\mathbb Z\mid x>n\}$ and let $A=\{A_n\}_{n\in\mathbb Z}$.
		
		The operator $^*:\mathcal P(X)\to\mathcal P(X)$ defined by: $B^*=B^c$ for $B,B^c\notin A$, otherwise $A_n^*=A_{n-1}^c$ and $A_n^{c*}=A_{n+1}$.
		
		Clearly $C^{**}=C$ for all $C\subseteq X$, and because $A_{n-1}\subseteq A_n$ we have $A_n^c\subseteq A_{n-1}^c$ so $C^*\cup C=X$.
	\end{proof}
	
	
%	\newline
	From the construction we can notice how $(A,\subsetneq)\cong(\Bbb{Z},<)$, as we will see soon, this isomorphism is appears in all $\star$-complement operators.
	
	\begin{definition}
		$\cl{A}{f}$ is the closure of $A$ under $f$, $\bigcup_{k\in\omega}\{f^k[A]\}$
	\end{definition}

	\begin{definition}
		For $f$ a bijection, $\clf{A}{f}=\cl{A}{f}\cup \cl{A}{f^{-1}}$
	\end{definition}
	\begin{theorem}\label{thm:2.5}
		If $^{*}$ is $\star$-complement operator on $X$, then for each $a\subseteq X$ we have $\clf{a}{^{*c}}=\{a\}$ or $(\clf{a}{^{*c}},\subsetneq)\cong(\Bbb{Z}, <)$ and there exists at least one $a\subseteq X$ such that the latter holds. In addition, if $a$ is finite or co-finite then $a^*=a^c$.
	\end{theorem}
	\begin{proof}
		Let $^{*c}$ be such operator, and assume $\clf{a}{^{*c}}\ne\{a\}$, because $a^{*}\supsetneq a^c$ such $a$ exists, if $(a^{*c})^{*c}=a^{*c}$ then $(a^{*c})^{*}=a^{*}$ so $a^{*c}=a$, $a^{*c*c}\ne a$ as well because $a^{*c*c}\subsetneq a^{*c}\subsetneq a$, continuing it for both direction will finish the proof of the first part.
		
		Assume that $a$ is finite(resp. co-finite) and $a^*\ne a^c$ then $\clf{a}{^{*c}}$ has a $\subseteq$-minimum(resp. $\subseteq$-maximum), and hence is not isomorphic to $\Bbb{Z}$, contradiction.
	\end{proof}
	In fact, $\clf{a}{^{*c}}$ can also be seen as the equivalence class $[a]_{\sim_*^\star}$ where $a\sim_*^\star b\iff \exists n\in\Bbb{Z}\;(A^{(*c)^n}=B)$, so $\{C\mid \exists a\subseteq X\;(\clf{a}{^{*c}}=C)\}$ is a partition of $\mathcal{P}(X)$.
	\begin{lemma}\label{lem:2.6}
		If $P$ is a partition of $\mathcal{P}(X)$ such that if $p\in P$ then either $|p|=1$ or $(p,\subsetneq)\cong(\Bbb{Z},<)$, and at least one $p \in P$ is the latter, then there exists $\star$-complement operator, $^*$, such that $\{C\mid \exists a\subseteq X\;(\clf{a}{^{*c}}=C)\}=P$.
	\end{lemma}
	\begin{proof}
		If $p\in P$ is such that $|p|=1$ then $a^*=a^c$ for the $a\in p$.
		
		If not, then $p=\{p_k\}_{k\in\Bbb{Z}}=\{\cdots, p_{-1}\subsetneq p_0\subsetneq p_1\cdots\}$, and let $p_k^*=p_{k-1}^c$ as we did in the proof of \hyperref[lem:2.2]{Lemma 2.2}.
	\end{proof}
	
	We classify all $X$ with a $\star$-complement operator.
	
	\begin{theorem}\label{thm:2.7}
		The following are equivalent:
		\begin{enumerate}
			\item{ There exists a $\star$-complement operator on $X$}
			\item{ $\mathcal{P}(X)$ has a $\Bbb{Z}$-chain(ordered by $\subsetneq$)}
			\item{ $X$ is countable union of infinite disjoint sets}
			\item{ $\mathcal{P}(X)$ is Dedekind infinite}
		\end{enumerate}
	\end{theorem}
	\begin{proof}
		(1)$\iff$(2) is clear by \hyperref[lem:2.2]{Lemma 2.2} and \hyperref[thm:2.5]{Theorem 2.5}.
		
		(2)$\implies$(3) let $\{P_j\}_{j\in\Bbb{Z}}$ be a chain of subsets of $X$, define for $j\in\omega$, $C_j=P_j\setminus\bigcup_{0\le k<j}P_k$, then $\{\bigcup_{j\in\omega}C_{\langle k,j\rangle}\}_{k\in\omega}\cup\{X\setminus\bigcup_{i\in \omega}P_i\}$ is a countable family of infinite disjoint sets whose union is $X$, where $\langle\cdot,\cdot\rangle:\omega^2\to\omega$ is a pairing function.
		
		(3)$\implies$(2), let $\{C_i\}_{i\in \omega}$ be countable family of infinite disjoint sets whose union is $X$, , reorder it to $\{D_i\}_{i\in\mathbb Z}$ and let $P_i=\bigcup_{k<i}D_k$ for each $i\in\mathbb Z$.
		
		(3)$\implies$(4) is trivial and (4)$\implies$(3) is due to Tarski\cite{SurLesEnsemblesFinis}:
		Let $X$ be a set such that $\mathcal{P}(X)$ is Dedekind-infinite, then let $(X_i)_{i\in\omega}$ be a sequence of subsets of $X$, and define the function $F:X\to\mathcal{P}(\omega)$:
		
		Let $a\in X$, then define $F(a)_n$ for $n\in\omega$ recursively: let $F(a)_n$ be the minimal $k\in\omega$ such that $\bigcup_{i\le n} X_{F(a)_i}\subsetneq \bigcup_{i< n} X_{F(a)_i}$ and $a\in X_k$, if not such $k$ exists, let $F(a)_n=F(a)_{n-1}$, let $F(a)=\{F(a)_n\}_{n\in\omega}$.
		
		If $F(a)$ is infinite we can use similar method as in the proof of (2)$\implies$(3), if $F(a)$ is finite for all $a$ we will note that $a\sim b\iff F(a)=F(b)$ is a equivalence relation, hence the underline equivalence classes are partition which is infinite and with injection to the set of finite subsets of $\omega$, $\mathcal{P}_{<\omega}(\omega)$.
		
		And because $|\mathcal{P}_{<\omega}(\omega)|=\aleph_0$, so does the partition of $X$, $X$ is a countable union of infinite disjoint sets.
	\end{proof}
	
	\begin{definition}
		$\Delta_1$-finite set is a set that is not disjoint union of 2 infinite sets
	\end{definition}
	\begin{definition}
		Amorphous set is an infinite $\Delta_1$-finite set.
	\end{definition}
	\begin{remark}
		If $X$ is amorphous set, then there is no $\star$-complement operator on $X$, as countably many disjoint subset induce partition of $X$ to infinite sets.
	\end{remark}
	
	\begin{theorem}\label{thm:2.11}
		It is not provable in ZF that there are no amorphous sets.
	\end{theorem}
	\begin{proof}
		The proof can be found at Lévy\cite{TheIndependenceOfVariousDefinitionsOfFiniteness} theorem 11.
	\end{proof}
	
	
	\begin{corollary}
		It is not provable in ZF that there exists a $\star$-complement operator on every infinite set.
	\end{corollary}
	
	\section{$\star$-strong complement operator}
	
	Now that we have shown some properties of $\star$-complement, we can ask "how far" can $\star$-complement be from the complement?
	
	Given an infinite set $X$ with a $\star$-complement operator $^*$ and $a\subseteq X$, we know by \hyperref[thm:2.5]{Theorem 2.5} that if $a$ is finite or co-finite then $a^*=a^c$, can there exists a $\star$-complement where the other direction is also true? i.e. $a^*=a^c$ if and only if $a$ is finite or co-finite?
	
	\begin{definition}
		An operator $^*:\mathcal P(X)\to\mathcal P(X)$ is called strong $\star$'-complement if 
		\begin{enumerate}
			\item{ for all $a\subseteq X$ we have $a^c\cup a=X$}
			\item{ for all $a\subseteq X$ we have $(a^c)^c=a$}
			\item{ for all $a\subseteq X$ we have $a^*=a^c$ if and only if $a$ is finite or co-finite}
		\end{enumerate}
		a strong $\star$'-complement operator is called strong $\star$-complement if it is also a $\star$-complement
	\end{definition}

	
	\begin{remark}\label{rem:3.2}
		A strong $\star$'-complement operators on $X$ is a strong $\star$-complement operators if and only if $X$ is not an amorphous set.
	\end{remark}

	By \hyperref[thm:2.11]{Theorem 2.11}, it is consistent with $ZF$ that there exists strong $\star$'-complement operators that are not strong $\star$-complement operators, but interestingly the existence of such operators implies that there exists infinite sets without strong $\star$'-complement operators at all.
	
	\begin{theorem}\label{thm:3.3}
		If every infinite set can be equipped with a strong $\star$'-complement operator then every strong $\star$'-complement operator is a strong $\star$-complement operator
	\end{theorem}
	\begin{proof}
		by \hyperref[rem:3.2]{Remark 3.2} all we need to prove is that there are no amorphous sets.
		
		Assume the contrary and let $X$ is amorphous, note that $2\times X$ is an infinite set that is not amorphous (indeed $\{0\}\times X, \{1\}\times X$ is a partition of $2\times X$ into 2 disjoint infinite sets), let $^*$ be strong $\star$'-complement operator on $2\times X$, by \hyperref[rem:3.2]{Remark 3.2} this operator is strong $\star$-complement operator, in particular there exists a $\star$-complement operator on $2\times X$.
		
		by \hyperref[thm:2.7]{Theorem 2.7} there exists $A_0,A_1,A_2$ partition of $2\times X$ into $3$ infinite sets, clearly for each $j\in\{0,1,2\}$ there exists $i\in\{0,1\}$ such that $\{i\}\times X\cap A_j$ is infinite, in particular, for some $i\in\{0,1\}$ we have $j\ne k$ such that $\{i\}\times X\cap A_j$ and $\{i\}\times X\cap A_k$ are disjoint infinite subsets of $\{i\}\times X$, but $|\{i\}\times X|=|X|$, so $\{i\}\times X$ is amorphous, contradiction.
	\end{proof}
	
	Similarly to $\star$-complement operator, we can classify the strong $\star$-complement operators using $\Bbb Z$-chains.
	\begin{theorem}\label{thm:3.4}
		The following are equivalent:
		\begin{enumerate}
			\item There exists a strong $\star$-complement operator on $X$
			\item There exists a partition of the infinite co-infinite subsets of $X$ into $\mathbb Z$-chains (ordered by inclusion)
		\end{enumerate}
	\end{theorem}
	\begin{proof}
		(1)$\implies$(2): let $^*$ be the strong $\star$-complement operator. For $a\subseteq X$ infinite co-infinite we have that $\clf{a}{^{*c}}\ne\{a\}$ so by \hyperref[thm:2.5]{Theorem 2.5} $\{[a]_{\sim_*^\star}\mid a\text{ is infinite co-infinite}\}$ is the desired partition.
		
		(2)$\implies$(1): Let $P$ be partition of $\mathcal{P}(X)$ be the partition of the infinite co-infinite subsets of $X$.\\
		If $a\subseteq X$ is finite or co-finite, define $a^*=a^c$, otherwise, there exists $\{A_i\}_{i\in\mathbb Z}\in P$ such that $a=A_j$, so  define $a^*=A_{}$
	\end{proof}

	
	\section{Relation to Axiom of Choice}
	There are several questions we can ask about the $\star$-complement operator and AC:
	\begin{enumerate}
		\item{ Does the existence of a $\star$-strong complement is provable in ZF?}
		\item{ How strong exactly is the axiom "There exists a $\star$-complement operator}{ on the power set of every infinite set"?}
		\item{ How strong exactly is the axiom "There exists a $\star$-strong complement}{ operator on the power set of every infinite set"?}
	\end{enumerate}
	
	Theorem 2.3 is answers (2), that axiom is equivalent to: $$\forall X\;(X\mbox{ is infinite}\implies\mathcal{P(}X)\mbox{ is Dedekind infinite})$$
	
	If we to borrow definitions from Truss\cite{ClassesOfDedekindDiniteCardinals}, then we have the following finiteness definition:
	
	\begin{definition}
		A cardinality $\kappa$ is $\Delta_4$-finite
		there is no surjective function from it to $\omega$.
	\end{definition}
	
	And, if we look at $\mathcal{P}(X)$ has a $\star$-complement operator as a finiteness definition, we have that it is equivalent to $$\omega=\Delta_4$$ easily by theorem 2.3 form (3)
	
	\medskip
	
	%Sets the bibliography style to UNSRT and imports the 
	%bibliography file "samples.bib".
	\bibliographystyle{unsrt}
	\bibliography{bibliography}
\end{document}




