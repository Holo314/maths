\documentclass[12pt,reqno]{article}
\usepackage[margin=3cm]{geometry}
\usepackage[utf8]{inputenc}


\usepackage{xfp}
\usepackage{natbib}
\usepackage{graphicx}
\usepackage{amsthm}
\usepackage{amsmath}
\usepackage{amssymb}
\usepackage{cases}
\usepackage{microtype}
\usepackage{hyperref}
\usepackage{mathrsfs}
\usepackage{xparse}

\makeatletter
\DeclareRobustCommand\widecheck[1]{{\mathpalette\@widecheck{#1}}}
\def\@widecheck#1#2{%
	\setbox\z@\hbox{\m@th$#1#2$}%
	\setbox\tw@\hbox{\m@th$#1%
		\widehat{%
			\vrule\@width\z@\@height\ht\z@
			\vrule\@height\z@\@width\wd\z@}$}%
	\dp\tw@-\ht\z@
	\@tempdima\ht\z@ \advance\@tempdima2\ht\tw@ \divide\@tempdima\thr@@
	\setbox\tw@\hbox{%
		\raise\@tempdima\hbox{\scalebox{1}[-1]{\lower\@tempdima\box
				\tw@}}}%
	{\ooalign{\box\tw@ \cr \box\z@}}}
\makeatother

\newtheorem{theorem}{Theorem}[section]
\newtheorem{proposition}[theorem]{Proposition}
\newtheorem{corollary}[theorem]{Corollary}
\newtheorem{lemma}[theorem]{Lemma}
\newtheorem{conjecture}[theorem]{Conjecture}
\newtheorem{example}[theorem]{Example}
\theoremstyle{definition}
\newtheorem{definition}[theorem]{Definition}
\theoremstyle{remark}
\newtheorem{remark}[theorem]{Remark}
\theoremstyle{exercise}
\newtheorem{exercise}[section]{Exercise}
\theoremstyle{subExercise}
\newtheorem{subExercise}{Part}[section]
\newtheorem{subSubExercise}{Sub Part}[subExercise]
\numberwithin{equation}{section}
\newcommand{\acl}[2]{\operatorname{acl}^{#2}\left(#1\right)}
\newcommand{\rank}[2]{\operatorname{rank}_{#2}\left(#1\right)}
\newcommand{\dcl}[2]{\operatorname{dcl}^{#2}\left(#1\right)}
\newcommand{\Aut}[1]{\operatorname{Aut}\left(#1\right)}
\newcommand{\Av}[1]{\operatorname{Av}\left(#1\right)}
\newcommand{\comment}[2]{#2}
\newcommand{\cof}[1]{\operatorname{cof}\left(#1\right)}
\renewcommand{\cal}[1]{\mathcal{#1}}
\newcommand{\D}[2][{}]{\text{Diag}_{#1}\left(#2\right)}
\renewcommand{\phi}{\varphi}
\newcommand{\code}[1]{\left\lceil#1\right\rceil}
\setcounter{MaxMatrixCols}{20}
\sloppy

\usepackage{expl3}




\ExplSyntaxOn

\tl_new:N \l_septatrix_env_tl
\NewDocumentCommand \getenv { o m }
{
	\sys_get_shell:nnN { kpsewhich ~ --var-value ~ #2 }
	{ \int_set:Nn \tex_endlinechar:D { -1 } }
	\l_septatrix_env_tl
	\IfNoValueTF {#1}
	{ \l_septatrix_env_tl }
	{ #1 }
}

\NewDocumentCommand{\ifenvsetTF}{mmm}
{
	\sys_get_shell:nnN { kpsewhich ~ --var-value ~ #1 } { } \l_tmpa_tl
	\tl_if_eq:NNTF \l_tmpa_tl \c_getenv_par_tl { #3 } { #2 }
}

\tl_new:N \l_env_tl
\NewDocumentEnvironment{ifEnvEq}{ m m m +b}{
	\sys_get_shell:nnN { kpsewhich ~ --var-value ~ #1 }
		{ \int_set:Nn \tex_endlinechar:D { -1 } }
		\l_env_tl
	
	\cs_generate_variant:Nn \tl_if_eq:nnTF { o }
	\tl_if_eq:onTF { \l_env_tl } { #2 } { \stepcounter{#3} } { #4 }
}{}
\ExplSyntaxOff


% Optional parameters: Title, Value to blacklist, EnvVar, exercise number
\NewDocumentEnvironment{cExercise}{ O{} O{} O{author} O{\fpeval{\value{section}+1}} +b }{
	\setcounter{section}{\fpeval{#4 - 1}}
	\begin{ifEnvEq}{#3}{#2}{section}
		\begin{exercise}
			#1
		\end{exercise}
		#5
	\end{ifEnvEq}
}{}

% Optional parameters: Title, Value to blacklist, EnvVar, exercise number
\NewDocumentEnvironment{cPart}{ O{} O{} O{author} O{\fpeval{\value{subExercise}+1}} +b }{
	\setcounter{subExercise}{\fpeval{#4 - 1}}
	\begin{ifEnvEq}{#3}{#2}{subExercise}
		\begin{subExercise}
			#1
		\end{subExercise}
		#5
	\end{ifEnvEq}
}{}

% Optional parameters: Title, Value to blacklist, EnvVar, exercise number
\NewDocumentEnvironment{cSubPart}{ O{} O{} O{author} O{\fpeval{\value{subSubExercise}+1}} +b }{
	\setcounter{subSubExercise}{\fpeval{#4 - 1}}
	\begin{ifEnvEq}{#3}{#2}{subSubExercise}
		\begin{subSubExercise}
			#1
		\end{subSubExercise}
		#5
	\end{ifEnvEq}
}{}

% Legacy Environment Start
\newcommand{\ex}[2][\fpeval{\value{section}+1}]{\setcounter{section}{\fpeval{#1 - 1}}
	\begin{exercise}
		#2
	\end{exercise}
}
\newcommand{\sub}[2][\fpeval{\value{subExercise}+1}]{\setcounter{subExercise}{\fpeval{#1 - 1}}
	\begin{subExercise}
		#2
	\end{subExercise}
}
\newcommand{\subb}[2][\fpeval{\value{subSubExercise}+1}]{\setcounter{subSubExercise}{\fpeval{#1 - 1}}
	\begin{subSubExercise}
		#2
	\end{subSubExercise}
}
% Legacy Environment End
\newcommand{\cl}[2]{\mbox{cl}_{#2}\left(#1\right)}
\newcommand{\CB}[1]{\operatorname{CB}\left(#1\right)}
\newcommand{\MR}[1]{\operatorname{MR}\left(#1\right)}
\newcommand{\MD}[1]{\operatorname{MD}\left(#1\right)}
\newcommand{\monster}{{\mathfrak C}}
\newcommand{\tp}[1]{\operatorname{tp}\left(#1\right)}
\newcommand{\hull}[3]{\operatorname{Hull}^{#2}_{#3}\left(#1\right)}
\newcommand{\stp}[1]{\operatorname{stp}\left(#1\right)}
\newcommand{\dom}[1]{\operatorname{dom}\left(#1\right)}
\newcommand{\range}[1]{\operatorname{range}\left(#1\right)}
\newcommand{\cnst}[2]{\operatorname{const}_{#1}\left(#2\right)}
\renewcommand{\c}{{\mathfrak c}}
\newcommand{\club}[1]{\operatorname{club}\left(#1\right)}
\newcommand{\Lev}[1]{\operatorname{Lev}\left(#1\right)}
\newcommand{\height}[1]{\operatorname{ht}\left(#1\right)}
\newcommand{\emptyseq}{\Lambda}

\newcommand{\N}{\mathbb N}
\newcommand{\Q}{\mathbb Q}
\newcommand{\R}{\mathbb R}
\newcommand{\C}{\mathbb C}
\newcommand{\F}{\mathbb F}
\renewcommand{\P}{\mathbb P}
\renewcommand{\Bbb}[1]{\mathbb #1}
\newcommand{\incomp}{\operatorname{\bot}}
\newcommand{\comp}{\operatorname{\|}}
\newcommand{\force}{\Vdash}
\newcommand{\Add}[2]{\operatorname{Add}\left(#1,#2\right)}

\usepackage{xparse}


\ExplSyntaxOn
\NewExpandableDocumentCommand \randint { m m }
{ \int_rand:nn { #1 } { #2 } }
\ExplSyntaxOff
\newcounter{cntTherefore}
\newcommand\setTherefore[2]{%
	\csdef{Therefore#1}{#2}}
\newcommand\addTherefore[1]{%
	\stepcounter{cntTherefore}%
	\csdef{Therefore\thecntTherefore}{#1}}
\newcommand\getTherefore[1]{%
	\csuse{Therefore#1}}
\newcommand{\Therefore}{\getTherefore{\randint{1}{\thecntTherefore}} }
\addTherefore{therefore}
\addTherefore{hence}
\addTherefore{which implies}
\addTherefore{consequently}
\addTherefore{it follows that}




\newcommand{\envOrDefault}[2]{\ifenvsetTF{#1}{\getenv{#1}}{#2}}

\newcommand{\envWithDefault}[1]{\envOrDefault{#1}{Holo}}

\def\bonktxt#1{% 
	\quitvmode\hbox{% 
		\pdfliteral{q 1 0 .15 .4 0 0 cm}\rlap{#1}\pdfliteral{Q}\hphantom{#1}% 
		\pdfliteral{q .9063 .4226 -.4226 .9063 -3 6 cm .2 0 0 .2 0 0 cm}\llap\bonktext\pdfliteral{Q}\ % 
		\pdfliteral{q 
			.81914 .57356 -.57356 .81914 -3 4 cm 
			1.7 0 0 1.7 0 0 cm 1 j 1 J .7 w 
			0 0 m 0 .7 l 2 .7 l 2 0 l b 
			0 j .3 w .2 -.4 m .2 1.1 l s 
			1.8 -.4 m 1.8 1.1 l s 
			.8 1.05 m .85 1.15 1.15 1.15 1.2 1.05 c 
			1 0 m 1 -1.4 l s 
			1.1 -1.4 m 1.2 -2 1.1 -3 y s 
			.9 -1.4 m .8 -2 .9 -3 y s 
			Q}\kern3pt\relax 
	}% 
} 
\def\bonktext{bonk!}
\newcommand{\bonk}[1]{\bonktxt{$#1$}}

\usepackage{datetime}
% \newdate{date}{06}{09}{2012}
% \date{\displaydate{date}}
\date{\today}


\author{\envWithDefault{author}}



\newcommand{\clf}{\mbox{clf}}
\newcommand{\ICI}{\mathcal{ICI}}
\title{Complement Like Operator}
\begin{document}
	
	\maketitle
	\section{Introduction}
	Given a set $X$, we can uniquely identify the complement operator $^c:\mathcal{P}(X)\to\mathcal{P}(X)$ using 3 properties:
	\begin{enumerate}
		\item{ for all $a\subseteq X$ we have $a^c\cap a=\emptyset$}
		\item{ for all $a\subseteq X$ we have $a^c\cup a=X$}
		\item{ for all $a\subseteq X$ we have $(a^c)^c=a$}
	\end{enumerate}
	We wish to explore "complement like operators", an operator $^*:\mathcal{P}(X)\to\mathcal{P}(X)$ that satisfy only 2 out of those 3 properties:
	
	
	\begin{definition}
		$\blacksquare$-complement is an operator $\mathcal{P}(X)\to\mathcal{P}(X)$ that satisfy only 1 and 2
	\end{definition}
	\begin{definition}
		$\bullet$-complement is an operator $\mathcal{P}(X)\to\mathcal{P}(X)$ that satisfy only 1 and 3
	\end{definition}
	\begin{definition}
		$\star$-complement is an operator $\mathcal{P}(X)\to\mathcal{P}(X)$ that satisfy only 2 and 3
	\end{definition}
	
	\begin{lemma}\label{lem:1.1}
		There are no $\blacksquare$ operator
	\end{lemma}
	\begin{proof}
		We will now show that property 1 and 2 implies property 3:\newline
		Let $^*$ be $\blacksquare$ operator, By property 1 we have $x\in a^*\subseteq X$ implies $x\notin a$, so $a^*\subseteq a^c$.\newline
		Similarly by property 2 we have $x\notin a$ implies $x\in a^*$ so $a^c\subseteq a^*$, hence $a^c=a^*$, thus property 3 also holds.
	\end{proof}
	
	So now we only need to consider $\star$ and $\bullet$ operators. The following theorems will justify only considering one of those 2 operators:
	
	\begin{theorem}\label{thm:1.2}
		There is bijection between the set of all $\bullet$ operators and the set of all $\star$ operators on $\mathcal{P}(X)$.
	\end{theorem}
	\begin{proof}
		Let $^*$ be a $\star$ operator, then $^*\to^{c*c}$.
		
		
		This is indeed a $\bullet$ operator: $(A^{c*c})^{c*c}=A^{c*cc*c}=A^{c**c}=A^{cc}=A$ and $x\in A^{c*c}\implies x \notin A^{c*}\implies x \notin A^{cc}=A\implies A\cap A^{c*c}=\emptyset$.
		
		Moreover, this map is also an inverse of itself: $^{c(c*c)c}=^*$
	\end{proof}
	From here on we will only consider $\star$-complement operators.
	
	\section{Properties of $\star$-complement operator}
	One can ask, is there exists a $\star$-complement operator? i.e. does there exists a set $X$ with $^*:\mathcal{P}(X)\to\mathcal{P}(X)$ such that $a^*\cup a=X$ and $a^*{}^*=a$ for all $a\subseteq X$?
	
%	\newline\newline
	For explicit example consider $X=\Bbb{N}$, now we will construct $A\in \mathcal{P(P(}X))$ like so: 
	\begin{enumerate}
		\item{let $A_0=$the set of even natural numbers}
		\item{If $n$ is a natural number then:}
		$$\qquad A_{n}=A_{n-1}\cup\{\min(x\in\Bbb{N}\mid x\notin A_{n-1})\}$$
		$$\qquad A_{-n}=A_{-n+1}\setminus\{\min(A_{-n+1})\}$$
	\end{enumerate}
	Now $A=\{A_j\}_{j\in\Bbb{Z}}$.
	
	It is easy to see that $A_j\subsetneq A_{j+1}$ for all $j\in\Bbb{Z}$, then we can define $^*:\mathcal{P}(X)\to\mathcal{P}(X)$ as follows:
	$$a^*=\begin{cases}a^c&a\notin A\\A_{j-1}^c&a=A_j\in A\end{cases}$$
	This $^*$ is a $\star$-complement operator.
	
%	\newline
	So, it is worth talking about $\star$-complement operators. The sharp ones may notice how $(A,\subsetneq)\cong(\Bbb{Z},<)$, this relation is in fact the entire classification of $^*$.
	\begin{definition}
		$\cl(A,f)$ is the closure of $A$ under $f$: $\cl(A,f)=\bigcup_{k\in\omega}\{f^k[A]\}$
	\end{definition}
	\begin{definition}
		$\clf(A,f)=\cl(A,f)\cup\cl(A,f^{-1})$
	\end{definition}
	\begin{lemma}
		If $^{*}$ is $\star$-complement operator on $\mathcal{P}(X)$, then for each $a\subseteq X$ we have $\clf(a,^{*c})=\{a\}$ or $(\clf(a,^{*c}),\subsetneq)\cong(\Bbb{Z}, <)$ and there exists at least one $a\subseteq X$ is the latter. In addition, if $a$ is finite or co-finite then $a^*=a^c$
	\end{lemma}
	\begin{proof}
		Let $^{*c}$ be such operator, and assume $\clf(a,^{*c})\ne\{a\}$, because $a^{*}\supsetneq a^c$ we have $a^{*c}\subsetneq a$, if $(a^{*c})^{*c}=a^{*c}$ then $(a^{*c})^{*}=a^{*}$ so $a^{*c}=a$, $a^{*c*c}\ne a$ as well because $a^{*c*c}\subsetneq a^{*c}\subsetneq a$, continuing it for both direction and that finish the proof of the first part(because $^{*c}$ is a bijective map this process will work for inverse as well).
		
		Assume that $a$ is finite(resp. co-finite) and $a^*\ne a^c$ then $\clf(a,^{*c})$ has a minimum(resp. maximum), and hence is not isomorphic to $\Bbb{Z}$
	\end{proof}
	In fact, $\clf(a,^{*c})$ can also be seen as the equivalence class $[a]_{\sim_*^\star}$ where $a\sim_*^\star b\iff \exists n\in\Bbb{Z}\;(A^{(*c)^n}=B)$, so $\{C\mid \exists a\subseteq X\;(\clf(a,^{*c})=C)\}$ is a partition of $\mathcal{P}(X)$.
	\begin{lemma}
		If $P$ is a partition of $\mathcal{P}(X)$ such that if $p\in P$ then either $|p|=1$ or $(p,\subsetneq)\cong(\Bbb{Z},<)$, and at least one $p \in P$ is the latter, then there exists $\star$-complement operator, $^*$, such that $\{C\mid \exists a\subseteq X\;(\clf(a,^{*c})=C)\}=P$.
	\end{lemma}
	\begin{proof}
		If $p\in P$ is such that $|p|=1$ then $a^*=a^c$ for the $a\in p$.
		
		If not, then $p=\{p_k\}_{k\in\Bbb{Z}}=\{\cdots, p_{-1}\subsetneq p_0\subsetneq p_1\cdots\}$, so let $p_k^*=p_{k-1}^c$.
	\end{proof}
	
	Now we ets $X$ we have $\star$-complement operator on $\mathcal{P}(X)$.
	
	\begin{theorem}
		The following are equivalent:
		\begin{enumerate}
			\item{ There exists a $\star$-complement operator on $\mathcal{P}(X)$}
			\item{ There exists $\Bbb{Z}$-chain to $\mathcal{P}(X)$(ordered by $\subsetneq$)}
			\item{ $X$ is countable union of infinite disjoint sets}
			\item{ $\mathcal{P}(X)$ is Dedekind infinite}
		\end{enumerate}
	\end{theorem}
	\begin{proof}
		(1)$\iff$(2) is clear by lemma 2.1 and lemma 2.2 
		
		Assume (2) let $\{P_j\}_{j\in\Bbb{Z}}$ be a chain of subsets of $X$, then define $C_j=P_j\setminus\bigcup_{k<j}P_k$, then $\{\bigcup_{j\in\omega}C_{\langle k,j\rangle}\}_{k\in\omega}\cup\{X\setminus\bigcup_{i\in \omega}P_i\}$ is a countable family of infinite disjoint sets whose union is $X$, where $\langle\cdot,\cdot\rangle:\omega^2\to\omega$ is a pairing function.
		
		Assume (3), let $\{C_i\}_{i\in\Bbb{Z}}$ be countable family of infinite disjoint sets whose union is $X$, let $P_i=\bigcup_{k<i}C_k$ to get (2).
		
		(3)$\implies$(4) is trivial and (4)$\implies$(3) is due to Tarski\cite{SurLesEnsemblesFinis}:
		Let $X$ be a set such that $\mathcal{P}(X)$ is Dedekind-infinite, then let $(X_i)_{i\in\omega}$ be a sequence of subsets of $X$, and define the function $F:X\to\mathcal{P}(\omega)$:
		
		Let $a\in X$, then define $F(a)_n$ for $n\in\omega$ recursively: let $F(a)_n$ be the minimal $k\in\omega$ such that $\bigcup_{i\le n} X_{F(a)_i}\subsetneq \bigcup_{i< n} X_{F(a)_i}$ and $a\in X_k$, if not such $k$ exists, let $F(a)_n=F(a)_{n-1}$, let $F(a)=\{F(a)_n\}_{n\in\omega}$.
		
		If $F(a)$ is infinite we can use similar method as in the proof of (2)$\implies$(3), if $F(a)$ is finite for all $a$ we will note that $a\sim b\iff F(a)=F(b)$ is a equivalence relation, hence the underline equivalence classes are partition which is infinite and with injection to the set of finite subsets of $\omega$, $\mathcal{P}_{<\omega}(\omega)$.
		
		And because $|\mathcal{P}_{<\omega}(\omega)|=\aleph_0$, so does the partition of $X$, $X$ is a countable union of infinite disjoint sets.
	\end{proof}
	
	\begin{definition}
		$\Delta_1$-finite set is a set that is not disjoint union of 2 infinite sets
	\end{definition}
	\begin{definition}
		Amorphous set is an infinite $\Delta_1$-finite set.
	\end{definition}
	
	\begin{theorem}
		It is not provable in ZF that there are no amorphous sets.
	\end{theorem}
	\begin{proof}
		The proof can be found at Lévy\cite{TheIndependenceOfVariousDefinitionsOfFiniteness} theorem 11.
	\end{proof}
	
	
	\begin{corollary}
		It is not provable in ZF that if $X$ is infinite set, then $\mathcal{P}(X)$ has a $\star$-complement operator.
	\end{corollary}
	
	\section{$\star$-strong complement operator}
	
	Now that we have shown some properties of $\star$-complement, we can ask "how far" can $\star$-complement be from the complement?
	
	By lemma 2.1 we know that $a$ is finite or co-finite implies any $\star$-complement of it must be the complement, can there exists a $\star$-complement where the other direction is also true? I.e. the $\star$ complement of $a$ is the complement if and only if $a$ is finite or co-finite?
	
	Assuming some form of choice, in particular, assuming $\Bbb{R}$ can be well ordered, we can construct such operator:
	\begin{gather*}
		\mbox{Let $\ICI(X)$ be the set of infinite co-infinite subsets of $X$, let $<$ be}\\
		\mbox{well ordering of $\ICI(\Bbb N)$ of order type $|\Bbb R|$, one can show that}\\
		A_{\subsetneq}=\{B\in\ICI(\Bbb N)\mid A\subsetneq B\}\mbox{ has the cardinality of $\Bbb R$ for all $A\in\ICI(\Bbb{N})$}\\
		\mbox{Now assume we defined $A^*$ for all $\ICI(\Bbb N)\ni A<B$, then let}\\
		B^*=\min(C\in B^c_{\subsetneq}\setminus\{A,A^*\}_{A<B} )\mbox{, we can do it because the set of $A<B$}\\
		\mbox{is of cardinality $<\Bbb R$.}\\
		\mbox{For $A\notin \ICI(\Bbb N)$ let $A^*=A^c$, and we are done.}
	\end{gather*}
	The above suggest that there is a point to define $\star$-strong complement.
	\begin{definition}
		An operator $^*:\mathcal{P}(X)\to \mathcal{P}(X)$ is a $\star$-strong complement if it is a $\star$-complement and for all $a\subseteq X$ we have $a^*=a^c\iff a\notin \ICI(X)$.
	\end{definition}
	
	One point we should be careful about in this definition is the fact that $^*$ is not the complement operator because it is $\star$-complement, and not because of $a^*=a^c\iff a\notin \ICI(X)$.\newline
	It is possible to define the following:
	\begin{definition}
		An operator $^*:\mathcal{P}(X)\to \mathcal{P}(X)$ is a $\star$-strong like complement if: 
		\begin{enumerate}
			\item{ for all $a\subseteq X$ we have $a^c\cup a=X$}
			\item{ for all $a\subseteq X$ we have $(a^c)^c=a$}
			\item{ for all $a\subseteq X$ we have $a^*=a^c\iff a\notin \ICI(X)$}
		\end{enumerate}
	\end{definition}
	And contrary to what one may think, this definition is not equivalent to $\star$-strong complement in general, in fact we have the following:
	\begin{lemma}
		Every $\star$-strong like complement on $\mathcal{P}(X)$ is a $\star$-complement on $\mathcal{P}(X)$ if and only if $X$ is not amorphous set. 
	\end{lemma}
	But interestingly enough the following also hold:
	\begin{lemma}
		For all infinite sets $X$ there is $\star$-strong like complement on $\mathcal{P}(X)$ implies that for all infinite sets $X$ there is $\star$-strong complement on $\mathcal{P}(X)$
	\end{lemma}
	So even so the existence of a $\star$-strong like complement does not implies the existence of $\star$-strong complement, if all sets has $\star$-strong like complement then all sets has $\star$-strong complement.
	
	Proof of lemma 3.2:
	\begin{proof}
		By lemma 3.1 we only need to show that there are no amorphous sets.
		
		Let $X$ be amorphous, then $2\times X$ is not amorphous, by assumption we have $\star$-strong like complement on $\mathcal{P}(2\times X)$, by lemma 3.1 we have $\star$-strong complement, by theorem 2.3 we have that $2\times X=A_0\cup A_1\cup A_2$ for $A_0,A_1,A_2$ are infinite disjoint sets.
		
		We have $A_0\cap\{i\}\times X$ is infinite for some $i\in\{0,1\}$, then both $A_1\cap\{i\}\times X$ and $A_2\cap\{i\}\times X$ are finite, hence both $A_1\cap\{1-i\}\times X$ and $A_2\cap\{1-i\}\times X$ are infinite, so we get $\pi_1\left(A_1\cap\{1-i\}\times X\right), \pi_1\left(A_2\cap\{1-i\}\times X\right)$ are 2 infinite disjoint subsets of $X$, which is contradiction of $X$ being amorphous.
	\end{proof}
	
%	\newline
	\begin{theorem}
		The following are equivalent:
		\begin{gather*}
			1.\mbox{ There exists a $\star$-strong complement operator on $\mathcal{P}(X)$}\\
			2.\mbox{ There exists a partition of $\ICI(X)$ to $\Bbb{Z}$-chains(ordered by $\subsetneq$)}
		\end{gather*}
	\end{theorem}
	\begin{proof}
		(1)$\implies$(2): By lemma 2.1 we have $\clf(a,^{*c})$ is either $\Bbb Z$-chain or $\{a\}$ for all $a\in\ICI(X)$, if $\clf(a,^{*c})=\{a\}$ then $a^{*c}=a\implies a^*=a^c\implies a\notin \ICI(X)$.
		
		(2)$\implies$(1): Let $P$ be partition of $\mathcal{P}(X)$ be the partition of $\ICI(X)$ plus singletons to the finite and co-finite elements. Let $^*$ be the operator from lemma 2.2, if $^*$ is not $\star$-strong complement, then take $a\in\ICI(X)$ such that $a^*=a^c$, then $\clf(a,^{*c})=\{a\}$, this set is in the partition of $\ICI(X)$ so $(\{a\},\subsetneq)\cong(\Bbb Z,<)$, contradiction.
	\end{proof}
	
	\section{Relation to Axiom of Choice}
	There are several questions we can ask about the $\star$-complement operator and AC:
	\begin{enumerate}
		\item{ Does the existence of a $\star$-strong complement is provable in ZF?}
		\item{ How strong exactly is the axiom "There exists a $\star$-complement operator}{ on the power set of every infinite set"?}
		\item{ How strong exactly is the axiom "There exists a $\star$-strong complement}{ operator on the power set of every infinite set"?}
	\end{enumerate}
	
	Theorem 2.3 is answers (2), that axiom is equivalent to: $$\forall X\;(X\mbox{ is infinite}\implies\mathcal{P(}X)\mbox{ is Dedekind infinite})$$
	
	If we to borrow definitions from Truss\cite{ClassesOfDedekindDiniteCardinals}, then we have the following finiteness definition:
	
	\begin{definition}
		A cardinality $\kappa$ is $\Delta_4$-finite
		there is no surjective function from it to $\omega$.
	\end{definition}
	
	And, if we look at $\mathcal{P}(X)$ has a $\star$-complement operator as a finiteness definition, we have that it is equivalent to $$\omega=\Delta_4$$ easily by theorem 2.3 form (3)
	
	\medskip
	
	%Sets the bibliography style to UNSRT and imports the 
	%bibliography file "samples.bib".
	\bibliographystyle{unsrt}
	\bibliography{bibliography}
\end{document}




