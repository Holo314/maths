\documentclass{beamer}
\hypersetup{pdfpagemode=FullScreen}
\usepackage[utf8]{inputenc}
\usepackage{utopia}
\usepackage{natbib}
\usepackage{graphicx}
\usepackage{amsthm}
\usepackage{amsmath}
\usepackage{amssymb}
\usepackage{cases}
\usepackage{microtype}
\usepackage{hyperref}
\usepackage{MnSymbol}
\usepackage{wasysym}
\theoremstyle{definition}
\theoremstyle{remark}
\newtheorem{remark}[theorem]{Remark}
\usepackage[utf8]{inputenc}
\usepackage{catchfile}


\usetheme{Frankfurt}
\usecolortheme{seagull}
\AtBeginSection[]{
	\begin{frame}
		\vfill
		\centering
		\begin{beamercolorbox}[sep=8pt,center,shadow=true,rounded=true]{title}
			\usebeamerfont{title}\secname\par%
		\end{beamercolorbox}
		\vfill
	\end{frame}
}

\newcommand{\F}{\mathcal{F}}
\newcommand{\U}{\mathcal{U}}


\usepackage{xparse}

\ExplSyntaxOn
\tl_new:N \l_septatrix_env_tl
\NewDocumentCommand \getenv { o m }
{
	\sys_get_shell:nnN { kpsewhich ~ --var-value ~ #2 }
	{ \int_set:Nn \tex_endlinechar:D { -1 } }
	\l_septatrix_env_tl
	\IfNoValueTF {#1}
	{ \tl_use:N \l_septatrix_env_tl }
	{ \tl_set_eq:NN #1 \l_septatrix_env_tl }
}
\tl_const:Nn \c_getenv_par_tl { \par }

\NewDocumentCommand{\ifenvsetTF}{mmm}
{
	\sys_get_shell:nnN { kpsewhich ~ --var-value ~ #1 } { } \l_tmpa_tl
	\tl_if_eq:NNTF \l_tmpa_tl \c_getenv_par_tl { #3 } { #2 }
}
\ExplSyntaxOff

\usepackage{datetime}
% \newdate{date}{06}{09}{2012}
% \date{\displaydate{date}}
\date{\today}
\newcommand{\envOrDefault}[2]{\ifenvsetTF{#1}{\getenv{#1}}{#2}}

\author{\envOrDefault{author}{Holo}}




\title[self-referentialism]{The self-referential problem}
\begin{document}
\frame{\titlepage}
\section[Type-Token]{Type-Token Distinction}
\begin{frame}
	Is 'Red' red?\pause\\
	Is '\textcolor{blue}{Red}' red?\pause\\
	Is '\textcolor{red}{Red}' red?\pause\\
	Is '\textcolor{red}{Blue}' red?\pause\\
	\begin{block}{Token (linguistics)}
		A \textbf{token} is an instance of word
	\end{block}\pause
	\begin{block}{Type (linguistics)}
		A \textbf{type} is an abstract idea that a \textit{token} refers to
	\end{block}
\end{frame}

\begin{frame}
	\begin{block}{Token (linguistics)}
		A \textbf{token} is an instance of word
	\end{block}
	\begin{block}{Type (linguistics)}
		A \textbf{type} is an abstract idea that a \textit{token} refers to
	\end{block}\pause
	"A rose is a rose is a rose."\pause\\
	The sentence has 3 types\pause: "a"\pause, "is"\pause and "rose"\pause\\
	The sentence contains 3 instances of the type "a"\pause, 2 instances of the type "is"\pause and 3 instances of the type "rose".
\end{frame}

\begin{frame}
	Is 'Red' red?\pause\\
	The token 'Red' is not red\pause, the type is \textit{also not red}.\pause\\
	Is '\textcolor{red}{Red}' red?\pause\\
	The token \textit{is} red\pause, this token refers to the same type as before, so it is also not red.\pause \newline\newline
	When asking "is 'A' B?" and the question doesn't make sense, the answer is "no".\pause so the type of 'Red', which doesn't have any colour what so ever, is \textbf{not} red.
\end{frame}

\begin{frame}{Examples}
	\pause
	The '(token) palindrome' is not a palindrome.\pause\\
	The '(token) civic' is a palindrome.\pause\\
	The '(type) cat' is not "three letters long"\pause\\
	The '(token) writable' is writable.\pause\\
	The '(type) writable' is not writable.\pause\newline\newline
	From now on, unless specify otherwise, when asking "is 'A' B?", we will talk about the \textit{type} of 'A' (Newhard variation).
\end{frame}

\section[Bucketing]{Partitioning the English language}
\begin{frame}
	\begin{block}{The Partition Claim}
		Given a property "B" we can check every word in the English language and put it in one of 2 buckets: "is B", "is not B".
	\end{block}\pause
	Every word in the English language \textbf{is not} "red".\pause\\
	Every word in the English language \textbf{is} "word"$^*$.\pause\\
	Some words in the English language are "nouns" and some are not.\pause\\
	Some words in the English language are "offensive" and some are not.\pause\newline\newline\newline
	
	$^*$ {\small{we can also talk about "sentences" and not only words, in which case not everything will be a "word"}}
\end{frame}

\begin{frame}
	\begin{block}{The Partition Claim}
		Given a property "B" we can check every word in the English language and put it in one of 2 buckets: "is B", "is not B".\\
		Alternative form: Every word is either \textbf{B} or not
	\end{block}\pause
	"Every word is either \textbf{red} or not"\pause\\
	"Every word is either a \textbf{word} or not"\pause\\
	"Every word is either a \textbf{noun} or not"\pause\\
	"Every word is either \textbf{offensive} or not"\pause
	\begin{block}{Grelling–Nelson paradox}
 		"\textbf{The Partition Claim}" is false
	\end{block}
\end{frame}

\section[Auto-Hetero Words]{Autological words and Heterological words}
\begin{frame}
	\begin{block}{Autological}
		A word is \textbf{autological} if it describes itself
	\end{block}\pause
	"word" is an autological word. "noun" is an autological word.\pause
	\begin{block}{Heterological}
		A word is \textbf{heterological} if it does not describes itself. That is, a word heterological if it is not autological
	\end{block}\pause
	"red" is heterological, "offensive" is heterological.
\end{frame}

\begin{frame}
	\begin{block}{The Partition Claim for Autological and Heterological words}
		Every word is either autological or heterological.
	\end{block}\pause
	We will show that the above statement is \textbf{false}, that is, there are words that are neither autological nor heterological.\pause\\
	Similarly we will find a word that \textit{can be} \textbf{both} autological and heterological.
\end{frame}

\section[Paradox]{Grelling–Nelson paradox\\\small Also known as Weyl's paradox and Grelling's paradox}
\begin{frame}{Heterological is not heterological}
	Assume \textbf{"Heterological" is heterological.}\pause\\
	Remember that heterological means "not autological"\pause\\
	\textbf{"Heterological" is not autological.}\pause\\
	But "autological" means "describes itself", so the above is actually\pause\\
	\textbf{"Heterological" is not heterological.}\pause\\
	But a word can't be \textbf{both} heterological and not heterological, so "heterological" cannot be heterological.
\end{frame}

\begin{frame}{Heterological is not autological}
	Assume \textbf{"Heterological" is autological.}\pause\\
	Remember that autological means "describes itself", so we have\pause\\
	\textbf{"Heterological" is heterological.}\pause\\
	But we already saw that this is impossible, so heterological can't be autological.
\end{frame}

\begin{frame}{Autological can be both autological and heterological}
	Unlike heterological, where it can be neither heterological nor autological, autological can be both.\pause\\
	By that we mean that if we claim that "autological" is autological, we will not find any holes that will imply otherwise\pause, but similarly if we claim that "autological" is heterological we won't find any holes in our logic.\pause\\
	This is the \textit{dual} version of the paradox
\end{frame}

\begin{frame}{Trying to resolve the paradox}
	One might try to resolve the paradox by redefining "heterological" to "words that are not autological and are not the word 'heterological'"\pause, this will just shift the problem to "nonautological".\pause\\
	In fact, any way of trying to resolve it by excluding some cases will just shift the problem into different words.\pause\\
	\begin{block}{The Partition Claim}
		Every word is either \textbf{B} or not
	\end{block}
	Cannot hold for every "B".
\end{frame}

\begin{frame}{The source of the paradox}
	The reason the paradox exists is because we are trying to ask self-referential questions: "is \textit{heterological} heterological?".\pause\\
	This paradox is not limited to linguistics\pause, it is called:\\
	\begin{itemize}
		\item \textit{Russell's Paradox} in set theory\pause
		\item \textit{Curry's Paradox} in formal logic\pause
		\item \textit{The Halting Problem} in computer science\pause
		\item \textit{Girard's paradox} in type theory\pause
		\item \textit{Curry's Paradox} (again) in lambda calculus\pause
		\item ...\pause
	\end{itemize}
	Pretty much any subject that tries to be "foundational" (be the 'building blocks' of everything else) have a variation of the paradox.\pause\\
	In formal settings (like maths or computer science) we resolve the paradox by restricting the formal rules we are using.
\end{frame}
\end{document}